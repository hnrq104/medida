\section{Lista 4 (04/09/2025)}

Listagem de problemas:
\begin{enumerate}
    \item Exercício \ref{prob:l4:1} : \checkmark
    \item Exercício \ref{prob:l4:2} : \Frowny
    \item Exercício \ref{prob:l4:3} : \Frowny
    \item Exercício \ref{prob:l4:4} : \Frowny
    \item Exercício \ref{prob:l4:5} : \Frowny
\end{enumerate}

\begin{problem}
    \label{prob:l4:1}
\end{problem}
\begin{proof}
    Sabemos que para qualquer $E \subset \R$, $\lambda^*(E) \leq \lambda^*(E \cap A) + \lambda^*(E - A)$.
    Queremos mostrar então que a outra desigualdade vale, i.e. $\lambda^*(E) \geq \lambda^*(E \cap A) + \lambda^*(E - A)$.
    
    Se $\lambda^*(E) = \infty$, não há nada a fazer. Suponha que $\lambda^*(E) < \infty$ e dado $\eps > 0$
    encontre um aberto $V$ com $E \subset V$ tal que 
    $$\lambda^*(V) < \lambda^*(E) + \eps .$$
    Note crucialmente que a condição que $A$ satisfaz significa que $A \in M_F$
    e sendo $V$ aberto, então $V \in M_F$ também. Logo, $\lambda^*(V) = \lambda^*(V \cap A) + \lambda^*(V - A)$.
    Portanto, temos 
    $$\lambda^*(E) + \eps > \lambda^*(V) = \lambda^*(V \cap A) + \lambda^*(V - A) \geq \lambda^*(E \cap A) + \lambda^*(E - A).$$
    Fazendo $\eps \to 0$, terminamos a demonstração.
\end{proof}

\begin{problem}
    \label{prob:l4:2}
\end{problem}
\begin{proof}
    Vamos seguir a dica do Rudin para esse exercício. Seja $\{K_\alpha\}$ a coleção de todos 
    os subconjuntos compactos de $X$ com $\mu(K_\alpha) = 1$. Essa coleção não é vazia pois $X$ está nela. 
    Defina o compacto (interseção de compactos)
    $$K = \bigcap_\alpha K_\alpha.$$
    Vamos mostrar que $K$ satisfaz as propriedades exigidas. 
    
    Naturalmente, se houvesse subcompacto próprio  $H \subsetneq K$
    com $\mu(H) = 1$, então teriamos que $K \subset H \subset X$ e $K = H$, absurdo. Como $\mu(K) \leq \mu(X) \leq 1$, 
    falta só mostrar que $\mu(K) \geq 1$.
    % ,faremos isso provando que se $V \supset K$ é um aberto, então $\mu(V) \geq 1$.
    Seja $V$ aberto com $K \subset V$, então $X - V$ é um compacto e em particular
    $$X - V \subset X - K = X \cap \bigg(\bigcup_\alpha K_\alpha^c\bigg) \subset \bigcup_\alpha K_\alpha^c.$$
    Tomando uma subcobertura finita, notamos que
    $$X - V \subset X \cap \bigcup_{i=1}^{n} K_{\alpha_i}^c = \bigcup_{i=1}^{n} (X - K_{\alpha_i})$$
    Portanto, 
    $$\mu(X - V) \leq \sum_{i=1}^{n} \mu(X - K_{\alpha_i}) = 0$$
    e temos $1 = \mu(X \cap V) \leq \mu(V)$. Como isso vale para qualquer $V$ aberto que contém $K$, tomando
    ínfimos, temos que $\mu(K) \geq 1$.
\end{proof}

\begin{problem}
    \label{prob:l4:3}
\end{problem}
Eu tinha uma resposta super complicada para essa pergunta, por sorte o Bruno - aluno de Doutorado - 
viu a questão e respondeu de maneira muito mais simples.
\begin{proof}
    Vamos mostrar que esses são os fechos de abertos limitados da reta. Uma inclusão 
    é óbvia, pois para toda $f \in C_c(\R)$, $\text{supp } f = \overline{f^{-1}(\R - \{0\})}$
    é o fecho do aberto limitado $f^{-1}(\R - \{0\})$. Agora para todo aberto $A$ limitado em $\R$, vamos mostrar 
    que existe uma função contínua $f$ com $A = f^{-1}(\R - \{0\})$, seguirá que $\overline{A} = \text{supp }f$.
    Como estamos na reta, escreva $A$ como união enumerável de intervalos disjuntos (suas componentes conexas)
    $$A = \bigcup_{n=1} (a_n, b_n)$$
    sendo $A$ limitado, $\inf a_n > \infty$ e $\sup b_n < \infty$. Então, defina $f$ contínua sendo
    $$f(x) = \begin{cases}
        0 & \text{se } x \not \in A\\
        (b_n - x)(x - a_n) & \text{se } x \in (a_n, b_n)
    \end{cases}$$
    Claramente, como $A$ é limitado, o suporte de $f$ é compacto e $f^{-1}(\R - \{0\}) = A$, pois $f \neq 0$ sse $f \in A$.
\end{proof}

\begin{problem}
    \label{prob:l4:4}
\end{problem}

\begin{problem}
    \label{prob:l4:5}
\end{problem}