\clearpage
\section{Lista 4 (04/09/2025)}

Listagem de problemas:
\begin{enumerate}
    \item Exercício \ref{prob:l4:1} : \checkmark
    \item Exercício \ref{prob:l4:2} : \checkmark
    \item Exercício \ref{prob:l4:3} : \checkmark
    \item Exercício \ref{prob:l4:4} : \checkmark
    \item Exercício \ref{prob:l4:5} : \checkmark/\Frowny\,  (Se sortear pode corrigir)
    \begin{enumerate}[label=(\alph*)]
        \item \checkmark
        \item \checkmark
        \item \checkmark
        \item \Frowny /\checkmark (Acho que não está suficientemente detalhada - alguns passos omissos)
        \item \Frowny/\checkmark (Não acredito que seja uma prova de verdade)
    \end{enumerate}
\end{enumerate}

\begin{problem}
    \label{prob:l4:1}
\end{problem}
\begin{proof}
    Sabemos que para qualquer $E \subset \R$, $\lambda^*(E) \leq \lambda^*(E \cap A) + \lambda^*(E - A)$.
    Queremos mostrar então que a outra desigualdade vale, i.e. $\lambda^*(E) \geq \lambda^*(E \cap A) + \lambda^*(E - A)$.
    
    Se $\lambda^*(E) = \infty$, não há nada a fazer. Suponha que $\lambda^*(E) < \infty$ e dado $\eps > 0$
    encontre um aberto $V$ com $E \subset V$ tal que 
    $$\lambda^*(V) < \lambda^*(E) + \eps .$$
    Note crucialmente que a condição que $A$ satisfaz significa que $A \in M_F$
    e sendo $V$ aberto, então $V \in M_F$ também. Logo, $\lambda^*(V) = \lambda^*(V \cap A) + \lambda^*(V - A)$.
    Portanto, temos 
    $$\lambda^*(E) + \eps > \lambda^*(V) = \lambda^*(V \cap A) + \lambda^*(V - A) \geq \lambda^*(E \cap A) + \lambda^*(E - A).$$
    Fazendo $\eps \to 0$, terminamos a demonstração.
\end{proof}

\begin{problem}
    \label{prob:l4:2}
\end{problem}
\begin{proof}
    Vamos seguir a dica do Rudin para esse exercício. Seja $\{K_\alpha\}$ a coleção de todos 
    os subconjuntos compactos de $X$ com $\mu(K_\alpha) = 1$. Essa coleção não é vazia pois $X$ está nela. 
    Defina o compacto (interseção de compactos)
    $$K = \bigcap_\alpha K_\alpha.$$
    Vamos mostrar que $K$ satisfaz as propriedades exigidas. 
    
    Naturalmente, se houvesse subcompacto próprio  $H \subsetneq K$
    com $\mu(H) = 1$, então teriamos que $K \subset H \subset X$ e $K = H$, absurdo. Como $\mu(K) \leq \mu(X) = 1$, 
    falta só mostrar que $\mu(K) \geq 1$.
    % ,faremos isso provando que se $V \supset K$ é um aberto, então $\mu(V) \geq 1$.
    Seja $V$ aberto com $K \subset V$, então $X - V$ é um compacto e em particular
    $$X - V \subset X - K = X \cap \bigg(\bigcup_\alpha K_\alpha^c\bigg) \subset \bigcup_\alpha K_\alpha^c.$$
    Tomando uma subcobertura finita, notamos que
    $$X - V \subset X \cap \bigcup_{i=1}^{n} K_{\alpha_i}^c = \bigcup_{i=1}^{n} (X - K_{\alpha_i})$$
    Portanto, 
    $$\mu(X - V) \leq \sum_{i=1}^{n} \mu(X - K_{\alpha_i}) = 0$$
    e temos $1 = \mu(X \cap V) \leq \mu(V)$. Como isso vale para qualquer $V$ aberto que contém $K$, tomando
    ínfimos, temos que $\mu(K) \geq 1$.
\end{proof}

\begin{problem}
    \label{prob:l4:3}
\end{problem}
Eu tinha uma resposta super complicada para essa pergunta, por sorte o Bruno - aluno de Doutorado - 
viu a questão e respondeu de maneira muito mais simples.
\begin{proof}
    Vamos mostrar que esses são os fechos de abertos limitados da reta. Uma inclusão 
    é óbvia, pois para toda $f \in C_c(\R)$, $\text{supp } f = \overline{f^{-1}(\R - \{0\})}$
    é o fecho do aberto limitado $f^{-1}(\R - \{0\})$. Agora para todo aberto $A$ limitado em $\R$, vamos mostrar 
    que existe uma função contínua $f$ com $A = f^{-1}(\R - \{0\})$, seguirá que $\overline{A} = \text{supp }f$.
    Como estamos na reta, escreva $A$ como união enumerável de intervalos disjuntos (suas componentes conexas)
    $$A = \bigcup_{n=1} (a_n, b_n)$$
    sendo $A$ limitado, $\inf a_n > \infty$ e $\sup b_n < \infty$. Então, defina $f$ contínua sendo
    $$f(x) = \begin{cases}
        0 & \text{se } x \not \in A\\
        (b_n - x)(x - a_n) & \text{se } x \in (a_n, b_n)
    \end{cases}$$
    Claramente, como $A$ é limitado, o suporte de $f$ é compacto e $f^{-1}(\R - \{0\}) = A$, pois $f(x) \neq 0$ sse $x \in A$.
\end{proof}

\begin{problem}
    \label{prob:l4:4}
\end{problem}
Essa questão é fácil mas é bem longa, dá uma preguiça miserável escrevê-la. Vamos seperá-la em partes.
\begin{prop}
    $\rho : \R^2 \times \R^2 \to \R^+$ definida por
    $$\rho((x_1, y_1) , (x_2,y_2)) = \begin{cases}
        |y_1 - y_2| & \text{se } x_1 = x_2\\
        1 + |y_1 - y_2| & \text{se } x_1 \neq x_2\\
    \end{cases}$$
    é uma métrica de $\R^2$.
\end{prop}
\begin{proof}
    Temos que verificar as propriedades usuais - todas são claras, mas farei a fim de completude. 
    Não negatividade e separabilidade segue de que 
    $$\rho((a,b),(x,y)) = 0 \iff a = x \land b = y$$
    e $\rho$ é positiva. Simetria segue diretamente da definição e de que $|y_1 - y_2| = |y_2 - y_1|$. Para desigualdade 
    triangular, temos que analisar dois casos, sejam $p_1 = (x_1,y_1)$, $p_2 = (x_2,y_2)$, $p_3 = (x_3,y_3)$ três pontos de $\R^2$, então
    \begin{enumerate}
        \item Se $x_1 = x_3$, então $\rho(p_1,p_3) = |y_1 - y_3| \leq |y_1 - y_2| + |y_2 - y_3| \leq \rho(p_1,p_2) + \rho(p_2,p_3)$.
        \item Se $x_1 \neq x_3$, então já temos que $x_1 \neq x_2$ ou $x_2 \neq x_3$, em ambos os casos
        $$\rho(p_1,p_3) = 1 + |y_1 - y_3| \leq 1 + |y_1 - y_2| + |y_2 - y_3| \leq \rho(p_1,p_2) + \rho(p_2,p_3).$$ 
    \end{enumerate}
\end{proof}
\begin{prop}
    O espaço $(\R^2, \rho)$ é localmente compacto.
\end{prop}
\begin{proof}
    Para todo ponto $p = (x,y)$, a bola aberta de raio $1/2$ ao redor de $p$ é $B_{(p,1/2)} = \{x\} \times (y-1/2, y+1/2)$ é aberto. Vamos mostrar que
    seu fecho, $B_{[p,1/2]} = \{x\} \times [y-1/2, y+1/2]$, é compacto. Para isso, vamos usar somente que $[y-1/2,y+1/2]$ é compacto em $\R$.
    Seja $\{U_\alpha\}_\alpha$ uma cobertura aberta de $B_{[p,1/2]}$.
    Então
        $$B_{[p,1/2]} \subset (\{x\}\times \R) \cap \bigcup_\alpha U_\alpha$$
    Fixando o eixo $e_1 = x$ e tomando projeções $\pi_2$ na segunda coordenada, vemos que
    $$
        B_{[p,1/2]} \subset \bigcup_\alpha U_\alpha \iff [y-1/2,y+1/2] \subset \bigcup_\alpha \pi_2((\{x\} \times \R) \cap U_\alpha).
    $$
    Mas é óbvio pela definição de $\rho$ (que condiz com a métrica usual de $\R$ se a primeira coordenada está fixa) 
    que $\pi_2((\{x\} \times \R) \cap U_\alpha)$ são conjuntos abertos em $\R$. Logo, como $[y-1/2, y+1/2]$ é compacto,
    obtemos um conjunto finito de indices $\{\alpha_i\}_{i=1}^{N}$, tal que
    $$
    [y-1/2,y+1/2] \subset \bigcup_{i=1}^N \pi_2((\{x\} \times \R) \cap U_{\alpha_i})
    $$
    Claramente, $B_{[p,1/2]} \subset \bigcup_{i=1}^N U_{\alpha_i}$. Como a cobertura aberta foi 
    tomada arbitráriamente, provamos que $B_{[p,1/2]}$ é compacto.
\end{proof}

\begin{prop}
    \label{prob:l4:finite_compact}
    Para $f \in C_c((\R^2,\rho))$, existem no máximo finitos $x_1, \dots x_n$ tais quais $f(x,y) \neq 0$ para algum $y$. 
\end{prop}
\begin{proof}
    Antes de provar o resultado, perceba que, dado $x \in \R$ fixo, o conjunto 
    $$\{x\} \times \R = \bigcup_{y \in \R} B_{((x,y) , 1/2)}$$
    é união de abertos e portanto aberto do nosso espaço métrico.
    Agora o resultado segue quase que imediatamente por contradição, 
    suponha que $f$ de suporte compacto não satisfaz a condição. Então 
    existem infinitos $x_\alpha$ com $f(x_\alpha,y)$ diferente de $0$ para algum $y$.
    Em particular, vale a seguinte inclusão natural
    \begin{equation}
        \text{supp f} \subset \bigcup_\alpha \{x_\alpha\} \times \R
    \end{equation}
    pois, se $x_\beta \not \in \{x_\alpha\}_\alpha$, então, para qualquer $y,z \in \R$ e $\alpha$, $\rho( (x_\beta, y) , (x_\alpha, z)) >= 1$. 
    Logo, $(x_\beta, y)$ não é valor de aderência de nenhuma sequência de $f^{-1}(\R - \{0\})$. Note 
    que a cobertura em (12) óbviamente não admite subcobertura finita, absurdo.
\end{proof}

\begin{prop}
    (Exercício) Defina, para $f \in C_c((\R^2,\rho))$ o funcional linear
    $$\Lambda(f) = \sum_{j=1}^{n} \int_\R f(x_j,y) dy.$$
    Seja $\mu$ a medida recuperada pelo TRR. Se $E$ é o eixo-$x$ então $\mu(E) = \infty$
    e $\mu(K) = 0$ para todo compacto $K \subset E$.
\end{prop}

\begin{proof}
    Primeiramente temos que mostrar que o conjunto é mensurável. Vamos fazer algo melhor e mostrar que ele 
    é boreliano. Note que ele é interseção enumerável de abertos
    $$\R \times \{0\} = \bigcap_{n \in \N} \bigcup_{x \in \R} B_{((x,0), 1/n)}.$$
    Portanto, faz sentido tomar $\mu$ sobre ele. Vamos mostrar que para qualquer aberto $V$ que contém o eixo $x$,
    $\mu(V) = \infty$, para isso, por definição, basta exibir uma sequência de funções $f_n \prec V$ com $\Lambda(f_n) \to \infty$.

    Falta uma ideia boa para essa parte, mostrar que $V$ é gordo em infinitos pontos. Defina os conjuntos $A_n$ onde $V$ é um pouco espesso
    $$A_n = \{x \in \R ; \{x\} \times (-1/n, 1/n) \subset V\}$$
    Note que, como $V$ é aberto e contém $\R \times \{0\}$, claramente todo ponto $x \in \R$ pertence a algum $A_n$, i.e
    $$\R \subset \bigcup_{n \in \N} A_n$$
    Como $\R$ é não enumerável e temos na direita uma união enumerável, precisamos que algum $A_n$ seja não enumerável e portanto
    infinito. Seja $A_M$ então um dos conjuntos infinitos e escolha infinitos pontos  distintos $(x_k)_k \in A_M$.
    Agora estamos prontos para definir nossa sequência de funções de suporte compacto em $V$. Seja $G$ função real 
    com $[-1/2M, 1/2M] \prec G \prec (-1/M, 1/M)$, naturalmente
    $$\int_\R G dy \geq \frac{1}{M}$$
    Defina $f_n$ sendo
    $$f_n(x,y) = \begin{cases}
        G(y) & \text{se } x = x_k \text{ para algum } k \leq n\\ 
        0 & \text{c.c}
    \end{cases}.$$
    Claramente $f_n \prec V$, no entanto, 
    $$\Lambda(f_n) = \sum_{i=1}^{n} \int_\R f_n(x_m, y) dy = \sum_{i=1}^{n} \int_\R G(y) dy \geq \frac{n}{M} \to \infty$$
    quando $n \to \infty$. Como consequência, $\mu(V) = \infty$. Portanto para qualquer aberto $V \supset E$, $\mu(V) = \infty$,
    logo $\mu(E) = \infty$.

    Para mostrar que para todo compacto $K \subset E$, $\mu(K) = 0$, vamos construir outra sequência $(f_n)$, dessa vez
    com $K \prec f_n$, mas que $\Lambda(f_n) \to 0$. Lembrando da prova da proposição \ref{prob:l4:finite_compact} e notamos 
    que $K$ é finito. Escrevendo $K = \{(x_1,0), \dots, (x_M,0)\}$ fica claro o que devemos fazer. Seja $g_n$ uma função real contínua 
    com $[-1/2n, 1/2n] \prec g_n \prec (-1/n,-1/n)$, análogamente ao caso anterior definimos 
    $$f_n(x,y) = \begin{cases}
        g_n(y) & \text{se } x = x_k \text{ para algum } k\\
        0 & \text{c.c}
    \end{cases}.$$
    É imediato que $K \prec f_n$, mas note que 
    $$\Lambda(f_n) = \sum_{k = 1}^{M} \int_\R f_n(x_k, y)dy = \sum_{k=1}^{M} \int_\R g_n(y)dy \leq \frac{2M}{n} \to 0$$
    quando $n \to 0$, portanto $\mu(K) \leq 0$.
\end{proof}

\begin{problem}
    \label{prob:l4:5}
\end{problem}
\begin{enumerate}[label=(\alph*)]
    \item \begin{proof}
        Fazendo duas iterações temos 
        $$C_1 = [0,1/3] \cup [2/3, 1]$$
        $$C_2 = [0,1/9] \cup [2/9, 1/3] \cup [2/3,7/9] \cup [8/9,1]$$
        Indutivamente, se $C_n = \bigcup_m^{M} [a_m, b_m]$ é união finita de intervalos fechados
        disjuntos, então, por definição
        $$C_{n+1} = \bigcup_m^{M} \bigg[a_m, \frac{2a_m + b_m}{3}\bigg] \cup \bigg[ \frac{a_m + 2b_m}{3}, b_m \bigg]$$
        também é união finita de intervalos fechados disjuntos.

        Claramente os $C_n$ são compactos encaixados não vazios, portanto $C = \bigcap_n C_n$ é um compacto não vazio. 
    \end{proof}

    \item \begin{proof}
        Vamos mostrar por indução que depois da $n$-ésima iteração, todos intervalos tem tamanho igual a $3^{-n}$. Note que isso 
        implica o resultado, pois, sendo $C$ interseção de todos os $C_n$, não contém nenhum intervalo de medida positiva.
        É claro para $n = 0$. Como na $n$-ésima iteração removemos 
        o terço do meio de todos os intervalos que sobraram em $C_{n-1}$ (todos de tamanho $3^{-(n-1)}$), só 
        sobram intervalos de tamanho $3^{-n}$, o que demonstra o passo indutivo. 
    \end{proof}

    \item \begin{proof}
        Basta notar que ao remover os terços do intervalos fechados de $C_n$, temos que $\mu(C_{n+1}) = (2/3)\mu(C_n)$, 
        pois de cada intervalo fechado conexo, deixamos somente $2/3$ dele sobrando.
        Portanto, $\mu(C_n) = (2/3)^n \mu(C_0) = (2/3)^n$ e temos 
        $$\forall n \quad \mu(C) \leq \mu(C_n) \leq (2/3)^n$$
        ou seja, $\mu(C) = 0$.
    \end{proof}

    \item \begin{proof}
        Há inúmeras formas de fazer isso, vou fazer a que acredito ser a mais intuitiva - mas não acho que é a mais fácil.
        Note que em cada iteração da construção, nunca removemos os pontos extremos dos intervalos fechados,
        sempre removemos abertos propriamente contidos no meio dos intervalos. Isso quer dizer 
        que em qualquer momento da construção, se $C_n$ é a união disjunta
        $$C_n = \bigcup_m [a_m, a_m + 3^{-n}]$$
        então, para cada $m$, $a_m, 3^{-n} \in C$. Além do mais, em cada iteração, cada intervalo é dividido 
        em um intervalo da "esquerda" e um intervalo da "direita", como visto na letra (a).
        
        Seja $E$ e $D$ símbolos para esquerda e direita, a ideia será construir uma injeção de $\{E,D\}^\N$ para $C$
        usando sequências de pontos extremais dos $C_n$ - em cada passo da construção de $C$ 
        decidimos se escolhemos um ponto esquerdo ou direito. Dada uma sequência $(a_0, a_1, \dots) \in \{E,D\}^\N$, 
        definimos uma sequência $f((a_n)_n) = (c_0, c_1, \dots)$ em $C$  iterativamente. Para o primeiro elemento, temos
        $$c_0 = \begin{cases}
            0 & \text{se } a_0 = E\\
            1 & \text{se } a_0 = D
        \end{cases}$$
        Agora, suponha que já construímos até $c_n$. Se $a_{n} = E$, então no $n$-ésimo passo escolhemos um ponto extremal esquerdo e portanto
        em $C_{n+1}$ temos um conjunto do tipo $[c_{n}, c_{n} + 3^{-n-1}]$. Agora definimos 
        $$c_{n+1} = \begin{cases}
            c_{n} & \text{se } a_{n+1} = E\\
            c_{n} + 3^{-n - 1} & \text{se } a_{n+1} = D\\
        \end{cases}.$$
        Semelhantemente, se $a_{n} = D$, então no passo anterior escolhemos $c_n$ extremal direito e, portanto, 
        em $C_{n+1}$ temos um conjunto do tipo $[c_n - 3^{-n - 1}, c_n]$. Definimos
        $$c_{n+1} = \begin{cases}
            c_{n} - 3^{-n - 1} & \text{se } a_{n+1} = E\\
            c_{n} & \text{se } a_{n+1} = D\\
        \end{cases}$$

        A afirmação é que cada sequência assim definida é convergente em $C$ e sequências distintas convergem em pontos distintos.
        Para mostrar que $(c_1, c_2, \dots)$ converge em $C$, note que todos os elementos pertencem a $C$ e para cada $n$,
        $|c_{n+1} - c_n| \leq 3^{-n - 1}$, portanto 
        $$\sum_{n=0}^{\infty} |c_{n+1} - c_n| \leq \sum_{n=0}^{\infty} 3^{-n - 1} < \infty$$
        Pelo M-teste de Weierstrass, $(c_n)$ converge, como $C$ é fechado, $(c_n)$ converge em um ponto de $C$. 
        
        Vamos mostrar  que sequências diferentes de $E,D$ geram pontos diferentes de $C$. Sejam $a = (a_n)_n \neq b = (b_n)_n$ ambas em $\{E,D\}^\N$,
        seja $n_0$ o primeiro natural tal que $a_{n_0} \neq b_{n_0}$, então no $n_0$-passo, uma sequência escolheu ir para a esquerda 
        e a outra escolheu ir para a direita. Suponha sem perda de generalidade que $a_{n_0} = E$ e $b_{n_0} = D$, então
        se $n_0 = 0$, fica claro que para todo $n > 0$, $f(a)_n \in [0,1/3]$ e $f(b)_n \in [2/3,1]$, logo as sequências associadas não podem convergir no mesmo ponto.
        Semelhantemente, se $n_0 > 0$, então sem perda de generalidade, suponha que $a_{n_0 - 1} = b_{n_0 - 1} = E$, temos que $[f(a)_{n_0} = c_{n_0 - 1}, c_{n_0 - 1} + 3^{-n_0} = f(b)_{n_0}] \subset C_{n_0}$
        e desse momento adiante, para $n > n_0$, 
        $$f(a)_n \in [f(a)_{n_0 + 1}, f(a)_{n_0 + 1} + 3^{-n_0 - 1}]$$
        mas
        $$f(b)_n \in [f(b)_{n_0 + 1} - 3^{-n_0 - 1}, f(b)_{n_0 + 1}]$$
        e $f(b)_{n_0} - f(a)_{n_0} = 3^{- n_0}$, portanto são sempre disjuntos por uma distância $3^{-n_0 - 1}$, isso é as sequências $f(a)_n$ e $f(b)_n$ não podem convergir no mesmo ponto.
        Isso prova a injetividade de $f$ e a não enumerabilidade de $C$.
    \end{proof}
    
    \item
    Antes de fato solucionar a questão, vamos olhar para o que acontece em $C_1$. Podemos escrever $x \in [0,1]$ como 
    $$x = \frac{x_1}{3} + \frac{x_2}{3^2} + \dots + \frac{x_n}{3^n} + \dots$$
    onde cada $x_n$ é igual a $0$, $1$ ou $2$. Na primeira etapa $C_1$, ao remover a terça parte,
    estamos removendo justamente os números $x \in [0,1]$ cuja representação ternária tem $x_1 = 1$, com a exceção
    de $1/3 = 0.1 = 0.0222\dots = 0.0\bar{2}$ que permanece. Na segunda etapa, removemos dos que sobraram os que tem $x_2 = 1$, com exceção
    daqueles cujo último digito não $0$ é $x_2 = 1$ que podem ser escritos como $0.x_1 1 = 0.x_10\bar{2}$.
    Podemos sempre substituir o algarismo $1$ final pela sequência $0222\dots$ e obter o mesmo número, então 
    de forma geral, em cada etapa $n$, estamos removendo os números restantes que em ternário tem $x_n = 1$. Sendo a interseção 
    de todos esses $C_n$, $C$ não possui $1$ em sua representação em base 3.  Vamos tentar formalizar por indução.
    
    \begin{prop}
        $C_n$ é o conjunto de pontos de $ 0.x_1x_2x_3\cdots x_n\dots \in [0,1]$ com $x_1 \dots x_n$ diferentes de $1$.
        Além do mais, $C_n$ pode ser expresso como uma união disjunta
        $$C_n = \bigcup_m^{2^n} [a_m, a_m + 3^{-n}]$$
        onde cada $a_m$ é da forma $0.x_1x_2\dots x_n\bar{0}$ com cada $x_i \in \{0,2\}$ - por isso $2^n$ deles.
    \end{prop}
    \begin{proof}
        A prova, como quase todas as anteriores, será por indução. Claramente $C_1$ satisfaz isso,
        $$C_1 = [0,0.1 = 0.0\bar{2}] \cup [0.2, 1 = 0.\bar{2}]$$
        Suponha que vale para $C_n$, vamos provar que vale para $C_{n+1}$. Basta usar a identidade que vimos 
        na letra (a), precisamente, na etapa $n+1$, teremos que o conjunto
        $$[a_m, a_m+3^{-n}]$$
        se tornará 
        $$[a_m, a_m + 3^{-n - 1}] \cup [a_m + 2\cdot3^{-n - 1}, a_m + 3^{-n}]$$
        Se $a_m$ era da forma $0.x_1\dots x_n0\bar{0}$, conseguimos em $C_{n+1}$, além de $a_m$, um novo $a'_m = 0.x_1\dots x_n2\bar{0}$
        que representa o intervalo direito. Como isso vale para todo $a_m$, na etapa $n+1$ estamos de fato 
        construindo toda as $n+1$ sequências de $0,2$ até o $(n+1)$ digito. Segue naturalmente, olhando para esses intervalos, 
        que removemos exatamente os números contendo o digito $1$ na $(n+1)$ casa decimal, provando a hipótese de indução.
    \end{proof}
    \begin{corollary}
        (Exercício) $C$ é exatamente os números de $[0,1]$ que não contém $1$ em sua expansão ternária.
    \end{corollary}
    \begin{proof}
        Segue do fato de $C$ ser a interseção de todos os $C_n$.  
    \end{proof}
\end{enumerate}