\section{Lista 3 (28/08/2025)}

Listagem de problemas:
\begin{enumerate}
    \item Exercício \ref{prob:l3:1} : \Frowny
    \item Exercício \ref{prob:l3:2} : \Frowny
    \item Exercício \ref{prob:l3:3} : \checkmark
    \item Exercício \ref{prob:l3:4} : \checkmark
    \item Exercício \ref{prob:l3:5} : \checkmark
    \item Exercício \ref{prob:l3:6} : \checkmark
\end{enumerate}

\begin{problem}
    \label{prob:l3:1}
\end{problem}
\begin{proof}
    A resposta dessa pergunta é positiva, mas eu penei um pouco para chegar nessa conclusão. Lembremos 
    que para mostrar que $f$ é Borel mensurável, basta mostrar que, para todo $c \in \R$, a pré-imagem 
    $f^{-1}((c, +\infty))$ é mensurável. Vamos mostrar que essa pré-imagem é uma união enumerável de conjuntos
    Borel mensuráveis em $\R$.

    Seja $A = f^{-1}((c, +\infty))$ e tome $a \in A$, i.e $f(a) > c$, pelas condições de continuidade em $f$, temos três casos possíveis:
    \begin{enumerate}
        \item $f$ é contínua em $a$, logo $\exists \delta_a > 0$ tal que $(a - \delta_a, a + \delta_a) \subset A$
        \item $f$ é contínua a esquerda em $a$, logo $\exists \delta_a > 0$ tal que $(a - \delta_a, a] \subset A$
        \item $f$ é contínua a direita em $a$, logo $\exists \delta_a > 0$ tal que $[a, a + \delta_a) \subset A$
    \end{enumerate}
    Vamos mostrar que $A$ é a união enumerável de seus componentes conexos, como os componentes conexos são intervalos
    da reta, eles são borelianos e portanto $A$ será boreliano. Para isso, basta notar que em cada componente conexo 
    há um racional que determina ele completamente - tome $x$ de um componente, olhe para 
    um racional no intervalinho associado a $x$. Como os racionais são enumeráveis, esses componentes são enumeráveis.
\end{proof}

\begin{problem}
    \label{prob:l3:2}
\end{problem}
\begin{proof}
    \textbf{RESOLVER PARTE POSITIVA E NEGATIVA, ESQUECI DISSO}

    Basta lembrar bem da definição da integral de Riemann para perceber que a de Lebesgue generaliza ela.
    Por Riemann, toda função contínua num compacto é integrável e suas somas inferiores e superiores convergem. Temos
    $$\int_a^b f(x) dx = \lim_{|P|\to 0} L(P,f)$$
    onde $P$ é um pontilhamento do compacto $[a,b]$, $L(f,P)$ é a soma inferior de $f$ por $P$ e $|P|$ é 
    o tamanho do maior intervalo do pontilhamento. Podemos expressar
    $L(P,f)$ como uma soma, se $P$ é $(a = t_0, \dots t_n = b)$, temos 
    $$L(P,f) = \sum_{i = 1}^{n} m_i (t_i - t_{i-1}) = \sum_{i = 1}^{n} m_i \mu([t_{i-1},t_i))$$
    onde $m_i = \inf\{f(x); x \in [t_{i-1}, t_{i})\}$.

    Olhando para essa fórmula é claro perceber que cada pontilhamento $P$ está associado com uma 
    função simples menor ou igual a $f$. A ideia da prova é escolher uma sequência de pontilhamento $(P_n)_n$ (diádicos por exemplo)
    cujo módulo $|P_n|$  tende a 0 e cada pontilhamento é um 
    refinamento do anterior. Dessa forma, eles definirão uma sequência crescentes de funções que convergem para $f$, 
    aplicando o Teorema da Convergência Monotona [\ref{trm:conv_mon}], teremos o resultado.

    Seja $P_0 = \{a,b\}$, definiremos indutivamente uma sequência de refinamentos (os diádicos).
    Dado $P_n = \{t_0 < t_1 < \dots < t_m\}$, cortamos cada intervalo no meio, i.e.
    $$P_{n+1} = P_n \cup \bigg\{\frac{t_i + t_{i+1}}{2} : 0 \leq i < m\bigg\}$$
    Claramente, $|P_n| = (b-a)/2^n \to 0$ e, portanto, pelos teoremas da integral de Riemann,
    \begin{equation}
        \lim_{n \to \infty} L(P_n,f) \to \int_a^b fdx
    \end{equation}
    Agora definimos uma função step $s_n$ associada ao pontilhamento $P_n = \{t_0 < t_1 < \dots < t_m\}$,
    $$s_n(x) = \begin{cases}
        \inf\{f(a): a \in [t_i, t_{i+1})\} & \text{se } x \in [t_i, t_{i+1}]\; \text{para } i < m\\
        \inf\{f(a): a \in [t_{m-1}, t_{m}]\} & \text{se } x \in [t_{m-1}, t_{m}]\\
    \end{cases}$$
    Separar o último caso não é necessário, coloquei somente por clareza. Da forma que estão definidos,
    (1) os $s_n$ são funções simples. Como os $P_n$ são refinamentos, (2) $s_{n} \leq s_{n+1}$ e, além do mais,
    sendo $f$ uniformemente contínua em $[a,b]$, temos que (3) $s_n \to f$ uniformemente. Por [\ref{trm:conv_mon}],
    \begin{equation}
        \lim_{n\to \infty} \int_{[a,b]} s_n d\mu= \int_{[a,b]} f d\mu
    \end{equation}
    Mas por serem funções simples,
    \begin{equation}
        \int_{[a,b]} s_n d\mu = L(P_n, f)
    \end{equation}
    Juntando as equações 9, 10 e 11, obtemos o resultado.
    $$\int_a^b fdx = \lim_{n \to \infty} L(P_n, f) = \lim_{n\to\infty }\int_{[a,b]} s_n d\mu = \int_{[a,b]} fd\mu$$
\end{proof}

\begin{problem}
    \label{prob:l3:3}
\end{problem}
\begin{proof}
    Esse exercício é similar ao \ref{prob:l2:5} da lista anterior, pelo menos a resolução do João. Sejam 
    \begin{align*}
        A_n &= \{x : |f_{n+1}(x) - f_n(x)| \geq \eps_n\}\\
        B_n &= \bigcup_{m \geq n} A_m
    \end{align*}
    Note que, sendo $\mu^*$ uma medida exterior,
    $$\mu^*(B_N) \leq \sum_{m = N}^{\infty} \mu^*(A_m) \to 0$$
    quando $N \to \infty$. Como os $B_n$ são encaixados, isso é o mesmo que dizer $\mu^*(\bigcap_n B_n) = 0$.
    Agora seja $x \not \in \bigcap_n^\infty B_n$, portanto existe $n_0$ tal que 
    $$ x \not \in B_{n_0}, B_{n_0 + 1}, \dots $$
    e, da mesma forma, 
    $$ x \not \in A_{n_0}, A_{n_0 + 1}, \dots$$
    Isso significa que para todo $m > n_0$, $|f_{m+1}(x) - f_{m}(x)| \leq \eps_m$. Como
    $\sum_n \eps_n < \infty$, pelo $M$-teste de Weierstrass, $f_n(x)$ converge. Ou seja, mostramos 
    que $(f_n(x))$ converge em quase todo ponto.
\end{proof}

\begin{problem}
    \label{prob:l3:4}
\end{problem}
Achei que esse problema era um pouco mais fácil do que realmente é. A ideia principal será 
aproximar $f$ por funções borelianas por baixo cujo limite das integrais é a integral de $f$.
Para formalizar isso, precisamos de um lema.

\textbf{Nota posterior:} Só percebi agora que $f$ é supostamente complexa, mas como vimos anteriormente
isso não faz a menor diferença, escreva $f = u^{+} - u^{-} + iv^{+} - iv^{-}$ e aplique o resultado
para cada parte. Vamos supor de agora em diante que $f : \R \to \R^+$.

\begin{lemma}
    \label{lemm:sn_ae}
    Seja $s_n$ uma sequência funções simples, todas menores igual a uma $f: X \to \R^+$ mensurável,
    as quais vale que
    $$\lim_{n \to \infty} \int_X s_n d\mu = \int_X f d\mu$$
    (Tal sequência sempre pode ser obtida pela definição da integral de $f$ e pela aproximação feita em aula). 
    Se definimos $S = \sup_n s_n$ mensurável, então $\mu(\{x : S(x) \neq f(x)\}) = 0$.
\end{lemma}
\begin{proof}
    Sabemos que o conjunto $\{x : S(x) \neq f(x)\}$ é mensurável, então a conclusão do lema faz sentido.
    Além disso, por definição, $S(x) \leq f(x)$. Defina $A_n = \{x : S(x) < f(x) - 1/n\}$, note que
    $$\bigcup_n A_n = \{x : S(x) \neq f(x)\}$$
    Portanto, se para todo $m$, $\mu(A_m) = 0$, valerá que $\mu(\{x : S(x) \neq f(x)\}) = 0$. Agora isso é 
    quase que imediato, já que, se $\mu(A_m) > 0$, então, para todo $n$
    $$\int_{X} f - s_n d\mu > \frac{\mu(A_m)}{m} > 0$$
    Mas isso é absurdo, pois $$\int_X f - s_n d\mu \to 0$$
\end{proof}
Essa é a ferramenta principal. Agora, para a questão, basta procurar uma sequência da mesma forma composta por funções borelianas 
- essa parte é mais difícil. Para isso, vou invocar o Teorema 2.17 do livro do Rudin (adaptado a $\R$).
\begin{theorem}
    Sejam $\mu$ e $M$ a medida e a $\sigma$-álgebra de Lebesgue da reta, elas satisfazem as seguintes propriedades:
    \begin{enumerate}[label=(\alph*)]
        \item Se $E \in M$ e $\eps > 0$, existe um fechado $F$ e um aberto $V$ tal que $F \subset E \subset V$ e $\mu(V - F) < \eps$. 
        \item $\mu$ é uma medida regular de Borel
        \item Se $E \in M$, existem $A$ e $B$ tal que $A$ é $F_\sigma$, $B$ é $G_\delta$, $A \subset E \subset B$, e $\mu(B-A) = 0$.
    \end{enumerate}
\end{theorem}
Agora estamos prontos para resolver o problema.
\begin{proof}
    \textbf{Prova do exercício.} Seja $(s_n)$ uma sequência de funções Lebesgue simples que aproximam
    a integral de $f$ por baixo. Cada $s_n$ pode ser escrita da forma
    $$s_n = \sum_{m=1}^{k} a_m \mathds{1}_{E_m}$$
    Onde os $E_m$ são conjuntos de Lebesgue, pelo Teorema, podemos aproximá-los por borelianos $A_m$ de mesma medida,
    de forma que a função simples boreliana 
    $$\tilde{s}_n = \sum_{m=1}^{k} a_m \mathds{1}_{A_m}$$
    é igual a $s_n$ a.e. Como
    $$\int_X \tilde{s}_n d\mu = \int_X s_n d\mu$$
    então temos que
    $$\lim_{n\to\infty} \int_X \tilde{s}_n d\mu = \int_X fd\mu$$
    Claramente, $\tilde{s}_n \leq s_n \leq f$, logo pelo lema \ref{lemm:sn_ae} $S = \sup \tilde{s}_n$ é idêntica
    a $f$ a.e. 
\end{proof}

\begin{problem}
    \label{prob:l3:5}
\end{problem}
Essa questão é a mais simples e vou tentar transcrever o desenho que soluciona ela em palavras.
A ideia aqui é que escrevamos uma sequência de funções trapezoidais em $[0,1]$ que vão ficando cada vez mais fininhas,
de forma que integrar sobre elas tenda a $0$, mas que cada ponto de $[0,1]$ chegue a valer $0$ e $1$ infinitas vezes.
\begin{proof}
    Vamos construir uma sequência de funções contínuas espertas $(F_n)_n$ em $[0,1]$.
    % $$F_{2^k}(x) = \begin{cases}
    %     1 &\text{se } x \leq 2^{-k}\\
    %     1 - 2^k(x - 2^{-k}) & \text{se } 2^{-k} \leq x \leq 2^{-k + 1}\\
    %     0 & \text{se }  x \geq 2^{-k + 1}
    % \end{cases}$$
    % Semelhantemente, definimos (para a última função do bloco)
    % $$F_{2^{k+1} - 1}(x) = \begin{cases}
    %     1 &\text{se } x \geq  1 - 2^{-k}\\
    %     2^k(x - (1 - 2^{-k})) & \text{se } 1 - 2^{-k} \geq x \geq 1 - 2^{-k + 1}\\
    %     0 & \text{se }  x \leq 1 - 2^{-k + 1}
    % \end{cases}$$
    % Ou mais econômicamente, $F_{2^{k+1} - 1}(x) = F_{2^k}(1 - x)$.
    Para $m \in \{2^k, 2^k + 1, \dots, 2^{k+1} - 1\}$, seja $n = m - 2^k$, definimos
    $$F_{m}(x) = \begin{cases}
        0 & \text{se }  x \leq (n - 1)2^{-k}\\
        2^k(x - (n-1)2^{-k}) & \text{se } (n-1)2^{-k} \leq x \leq n2^{-k}\\
        1 &\text{se } n2^{-k} \leq x \leq (n+1)2^{-k}\\
        1 - 2^k(x - (n+1)2^{-k}) & \text{se } (n+1)2^{-k} \leq x \leq (n+2)2^{-k}\\
        0 & \text{se }  x \geq (n+2)2^{-k}
    \end{cases}$$
    Onde óbiviamente $F_m$ está definida definida dessa forma quando os casos fazem sentido.
    Por exemplo quando $m = 2^k$, $n=0$ o primeiro e o segundo caso não aparecem. Quando 
    $m = 2^{k+1} - 1$, $n = 2^{k} - 1$, o quarto e o último não aparecem. Essas funções são claramente contínuas, 
    são trapézios que vão ficando cada vez menos espessos. Se $2^k \leq m < 2^{k+1}$, uma soma simples 
    sobre funções afins mostra que 
    $$\int_0^1 F_m dx \leq 2^{-k + 1} \to 0$$
    No entanto, é também fácil perceber que se $k > 2$, para qualquer $x \in [0,1]$, existem $2^k \leq N,M < 2^{k+1}$
    tais que $F_N(x) = 0$ e $F_M(x) = 1$. Portanto $F_m(x)$ não converge para nenhum ponto, mesmo que as integrais convirjam. 
\end{proof}

\begin{problem}
    \label{prob:l3:6}
\end{problem}
Esse é o problema mais legal, não acredito que consiguiria fazê-lo sem uma dica da professora Cynthia.
A única função não mensurável que conhecemos até agora é a característica de um conjunto não mensurável,
a ideia é tentar formar essa característica somente no liminf. Vamos fazer isso removendo pontualmente 
o complementar de um conjunto não mensurável infinitas vezes. \textbf{Obs:} A escolha esquista
de $(0,1]$ nos conjuntos a seguir é para facilitar a colagem que precisaremos fazer para construir a função $f$.
\begin{proof}
    Seja $T$ um conjunto não mensurável de $(0,1]$ e $T' = (0,1] - T$ seu complementar em $(0,1]$, note que $T'$ também é não mensurável.
    Vamos definir uma função $g(x,t) : \R\times(0,1] \to \{0,1\}$ que será a nossa ferramenta principal para construir  $f$.
    
    Dado $t$ fixo, se $t \in T'$, seja $A_t = (0,1] - \{t\}$, então definimos 
    $$g(x,t) \begin{cases}
        \mathds{1}_{A_t}(x) & \text{se } t \in T'\\
        \mathds{1}_{(0,1]}(x) & \text{se } t \not \in T'
    \end{cases}$$
    Note que, trivialmente, para todo $t \in (0,1]$, a função $g(t,x)$ -
    com a variável em $x$ - é mensurável, além do mais, sua integral sobre $x$ é claramente $1$. 
    
    Agora a  ideia é de alguma forma colar infinitas cópias de $g$ uma acima da outra.
    Separe $(0,1]$ na união disjunta:
    $$(0,1] = \bigcup_{n=1}^{\infty} (2^{-n}, 2^{-n + 1}]$$
    Definiremos $g_n(x,t) : \R \times (2^{-n},2^{-n + 1}] \to \{0,1\}$ da seguinte forma:
    $$g_n(x, t) = g(x, t2^n - 1)$$
    Por fim, defina $f :\R^2 \to \{0,1\}$ colando as $g_n$.
    $$f(x,t) = \begin{cases}
        \mathds{1}_{(0,1]}(x) & \text{se } t \leq 0 \ \lor \ t > 1\\
        g_n(x,t) & \text{se } t \in (2^{-n}, 2^{-n + 1}] 
    \end{cases}$$
    Afirmo que $f$ satisfaz as propriedades do exercício. Claramente a integral de 
    Lebesgue $\int_\R f(x,t) dx = 1$ para todo $t \in \R$, pois, fixando $t$, nossa 
    função é sempre a indicadora de $(0,1]$ salvo as vezes um único ponto. Para a segunda propriedade,
    vamos querer verificar que $h(x) = \liminf_{t \to 0} f(x,t)$ é justamente $\mathds{1}_T(x)$ que não é mensurável.
    Para isso note que se $x \in T \subset [0,1]$, então $g(x,t) = 1$ para todo $t$ e portanto, $g_n(x,t) = 1$ para qualquer $t$ também.
    Logo $f(x,t) = 1$ para todo $t$ e $h(x) = 1$. Se $x \in (0,1]^c$ então também, trivialmente $f(x,t) = 0$ para todo $t$
    e $h(x) = 0$. Agora, se $x \in T'$, para todo $n$, vale que 
    $$g_n\bigg(x, \frac{x + 1}{2^n}\bigg) = \mathds{1}_{A_x}(x) = 0$$
    e, para qualquer outro $t \in (2^{-n}, 2^{-n + 1}]$,
    $$g_n(x, t) = 1$$
    Em particular, sendo colagem desses valores, vale que para valores arbitrariamente pequenos de $t$ atingimos
    $f(x,t) = 0$ e portanto $h(x) = \liminf_{t \to 0} f(x,t) = 0$. Segue dos casos anteriores que $\liminf_{t \to 0} f(x,t)= \mathds{1}_T(x)$
    que não é mensurável.
\end{proof}

