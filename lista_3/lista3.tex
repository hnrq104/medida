\section{Lista 3 (28/08/2025)}

Listagem de problemas:
\begin{enumerate}
    \item Exercício \ref{prob:l3:1} : \Frowny
    \item Exercício \ref{prob:l3:2} : \Frowny
    \item Exercício \ref{prob:l3:3} : \Frowny
    \item Exercício \ref{prob:l3:4} : \Frowny
    \item Exercício \ref{prob:l3:5} : \Frowny
    \item Exercício \ref{prob:l3:6} : \Frowny
\end{enumerate}

\begin{problem}
    \label{prob:l3:1}
\end{problem}
\begin{proof}
    A resposta dessa pergunta é positiva, mas eu penei um pouco para chegar nessa conclusão. Lembremos 
    que para mostrar que $f$ é Borel mensurável, basta mostrar que, para todo $c \in \R$, a pré-imagem 
    $f^{-1}((c, +\infty))$ é mensurável. Vamos mostrar que essa pré-imagem é uma união enumerável de conjuntos
    Borel mensuráveis em $\R$.

    Seja $A = f^{-1}((c, +\infty))$ e tome $a \in A$, i.e $f(a) > c$, pelas condições de continuidade em $f$, temos três casos possíveis:
    \begin{enumerate}
        \item $f$ é contínua em $a$, logo $\exists \delta_a > 0$ tal que $(a - \delta_a, a + \delta_a) \subset A$
        \item $f$ é contínua a esquerda em $a$, logo $\exists \delta_a > 0$ tal que $(a - \delta_a, a] \subset A$
        \item $f$ é contínua a direita em $a$, logo $\exists \delta_a > 0$ tal que $[a, a + \delta_a) \subset A$
    \end{enumerate}
    Vamos mostrar que $A$ é a união enumerável de seus componentes conexos, como os componentes conexos são intervalos
    da reta, eles são borelianos e portanto $A$ será boreliano. Para isso, basta notar que em cada componente conexo 
    há um racional que determina ele completamente - tome $x$ de um componente, olhe para 
    um racional no intervalinho associado a $x$. Como os racionais são enumeráveis, esses componentes são enumeráveis.
\end{proof}

\begin{problem}
    \label{prob:l3:2}
\end{problem}
\begin{proof}
    \textbf{RESOLVER PARTE POSITIVA E NEGATIVA, ESQUECI DISSO}

    Basta lembrar bem da definição da integral de Riemann para perceber que a de Lebesgue generaliza ela.
    Por Riemann, toda função contínua num compacto é integrável e suas somas inferiores e superiores convergem. Temos
    $$\int_a^b f(x) dx = \lim_{|P|\to 0} L(P,f)$$
    onde $P$ é um pontilhamento do compacto $[a,b]$, $L(f,P)$ é a soma inferior de $f$ por $P$ e $|P|$ é 
    o tamanho do maior intervalo do pontilhamento. Podemos expressar
    $L(P,f)$ como uma soma, se $P$ é $(a = t_0, \dots t_n = b)$, temos 
    $$L(P,f) = \sum_{i = 1}^{n} m_i (t_i - t_{i-1}) = \sum_{i = 1}^{n} m_i \mu([t_{i-1},t_i))$$
    onde $m_i = \inf\{f(x); x \in [t_{i-1}, t_{i})\}$.

    Olhando para essa fórmula é claro perceber que cada pontilhamento $P$ está associado com uma 
    função simples menor ou igual a $f$. A ideia da prova é escolher uma sequência de pontilhamento $(P_n)_n$ (diádicos por exemplo)
    cujo módulo $|P_n|$  tende a 0 e cada pontilhamento é um 
    refinamento do anterior. Dessa forma, eles definirão uma sequência crescentes de funções que convergem para $f$, 
    aplicando o Teorema da Convergência Monotona [\ref{trm:conv_mon}], teremos o resultado.

    Seja $P_0 = \{a,b\}$, definiremos indutivamente uma sequência de refinamentos (os diádicos).
    Dado $P_n = \{t_0 < t_1 < \dots < t_m\}$, cortamos cada intervalo no meio, i.e.
    $$P_{n+1} = P_n \cup \bigg\{\frac{t_i + t_{i+1}}{2} : 0 \leq i < m\bigg\}$$
    Claramente, $|P_n| = (b-a)/2^n \to 0$ e, portanto, pelos teoremas da integral de Riemann,
    \begin{equation}
        \lim_{n \to \infty} L(P_n,f) \to \int_a^b fdx
    \end{equation}
    Agora definimos uma função step $s_n$ associada ao pontilhamento $P_n = \{t_0 < t_1 < \dots < t_m\}$,
    $$s_n(x) = \begin{cases}
        \inf\{f(a): a \in [t_i, t_{i+1})\} & \text{se } x \in [t_i, t_{i+1}]\; \text{para } i < m\\
        \inf\{f(a): a \in [t_{m-1}, t_{m}]\} & \text{se } x \in [t_{m-1}, t_{m}]\\
    \end{cases}$$
    Separar o último caso não é necessário, coloquei somente por clareza. Da forma que estão definidos,
    (1) os $s_n$ são funções simples. Como os $P_n$ são refinamentos, (2) $s_{n} \leq s_{n+1}$ e, além do mais,
    sendo $f$ uniformemente contínua em $[a,b]$, temos que (3) $s_n \to f$ uniformemente. Por [\ref{trm:conv_mon}],
    \begin{equation}
        \lim_{n\to \infty} \int_{[a,b]} s_n d\mu= \int_{[a,b]} f d\mu
    \end{equation}
    Mas por serem funções simples,
    \begin{equation}
        \int_{[a,b]} s_n d\mu = L(P_n, f)
    \end{equation}
    Juntando as equações 9, 10 e 11, obtemos o resultado.
    $$\int_a^b fdx = \lim_{n \to \infty} L(P_n, f) = \lim_{n\to\infty }\int_{[a,b]} s_n d\mu = \int_{[a,b]} fd\mu$$
\end{proof}

\begin{problem}
    \label{prob:l3:3}
\end{problem}
\begin{proof}
    Esse exercício é similar ao \ref{prob:l2:5} da lista anterior, pelo menos a resolução do João. Sejam 
    \begin{align*}
        A_n &= \{x : |f_{n+1}(x) - f_n(x)| \geq \eps_n\}\\
        B_n &= \bigcup_{m \geq n} A_m
    \end{align*}
    Note que, sendo $\mu^*$ uma medida exterior,
    $$\mu^*(B_N) \leq \sum_{m = N}^{\infty} \mu^*(A_m) \to 0$$
    quando $N \to \infty$. Como os $B_n$ são encaixados, isso é o mesmo que dizer $\mu^*(\bigcap_n B_n) = 0$.
    Agora seja $x \not \in \bigcap_n^\infty B_n$, portanto existe $n_0$ tal que 
    $$ x \not \in B_{n_0}, B_{n_0 + 1}, \dots $$
    e, da mesma forma, 
    $$ x \not \in A_{n_0}, A_{n_0 + 1}, \dots$$
    Isso significa que para todo $m > n_0$, $|f_{m+1}(x) - f_{m}(x)| \leq \eps_m$. Como
    $\sum_n \eps_n < \infty$, pelo $M$-teste de Weierstrass, $f_n(x)$ converge. Ou seja, mostramos 
    que $(f_n(x))$ converge em quase todo ponto.
\end{proof}

\begin{problem}
    \label{prob:l3:4}
\end{problem}
Achei que esse problema era um pouco mais fácil do que realmente é. A ideia principal será 
aproximar $f$ por funções borelianas por baixo cujo limite das integrais é a integral de $f$.
Para formalizar isso, precisamos de um lema.

\begin{lemma}
    Seja $s_n$ uma sequência funções simples, todas menores igual a uma $f: X \to \R^+$ mensurável,
    as quais vale que $s_n(x) \to f(x)$ e 
    $$\lim_{n \to \infty} \int_X s_n d\mu = \int_X f d\mu$$
    (Tal sequência sempre pode ser obtida pela definição da integral de $f$ e pela aproximação feita em aula). 
    Se definimos $S = \sup_n s_n$ mensurável, então $\mu(\{x : S(x) \neq f(x)\}) = 0$.
\end{lemma}
\begin{proof}
    Sabemos que o conjunto $\{x : S(x) \neq f(x)\}$ é mensurável, então a conclusão do lema faz sentido.
    Além disso, por definição, $S(x) \leq f(x)$. Defina $A_n = \{x : S(x) < f(x) - 1/n\}$, note que
    $$\bigcup_n A_n = \{x : S(x) \neq f(x)\}$$
    Portanto, se para todo $m$, $\mu(A_m) = 0$, valerá que $\mu(\{x : S(x) \neq f(x)\}) = 0$. Agora isso é 
    quase que imediato, já que, se $\mu(A_m) > 0$, então, para todo $n$
    $$\int_{X} f - s_n d\mu > \frac{\mu(A_m)}{m} > 0$$
    Mas isso é absurdo, pois $$\int_X f - s_n d\mu \to 0$$
\end{proof}
Essa é a ferramenta principal. Agora, para a questão, basta procurar uma sequência da mesma forma composta por funções borelianas 
- essa parte é mais difícil - a forma com que farei isso aqui muito provavelmente não é a ideal.

IDEIA, DADA UMA FUNÇÃO STEP LEBESGUE, APROXIMAR OS CONJUNTOS DE LEBESGUE  POR $\eps/2^N$ EM FECHADOS (BOREL) POR BAIXO
SE A GENTE PARTILHAR A RETA EM INTERVALOS DE $[0,1]$ GANHAMOS
\begin{problem}
    \label{prob:l3:5}
\end{problem}
\begin{proof}
    
\end{proof}

\begin{problem}
    \label{prob:l3:6}
\end{problem}
\begin{proof}
    
\end{proof}

