\section{Lista 3 (28/08/2025)}

Listagem de problemas:
\begin{enumerate}
    \item Exercício \ref{prob:l3:1} : \Frowny
    \item Exercício \ref{prob:l3:2} : \checkmark
    \item Exercício \ref{prob:l3:3} : \checkmark
    \item Exercício \ref{prob:l3:4} : \checkmark
    \item Exercício \ref{prob:l3:5} : \checkmark
    \item Exercício \ref{prob:l3:6} : \checkmark
\end{enumerate}

\begin{problem}
    \label{prob:l3:1}
\end{problem}
\begin{proof}
    A resposta dessa pergunta é positiva, mas eu penei um pouco para chegar nessa conclusão. Lembremos 
    que para mostrar que $f$ é Borel mensurável, basta mostrar que, para todo $c \in \R$, a pré-imagem 
    $f^{-1}((c, +\infty))$ é mensurável. Vamos mostrar que essa pré-imagem é uma união enumerável de conjuntos
    Borel mensuráveis em $\R$.

    Seja $A = f^{-1}((c, +\infty))$ e tome $a \in A$, i.e $f(a) > c$, pelas condições de continuidade em $f$, temos três casos possíveis:
    \begin{enumerate}
        \item $f$ é contínua em $a$, logo $\exists \delta_a > 0$ tal que $(a - \delta_a, a + \delta_a) \subset A$
        \item $f$ é contínua a esquerda em $a$, logo $\exists \delta_a > 0$ tal que $(a - \delta_a, a] \subset A$
        \item $f$ é contínua a direita em $a$, logo $\exists \delta_a > 0$ tal que $[a, a + \delta_a) \subset A$
    \end{enumerate}
    Vamos mostrar que $A$ é a união enumerável de seus componentes conexos, como os componentes conexos são intervalos
    da reta, eles são borelianos e portanto $A$ será boreliano. Para isso, basta notar que em cada componente conexo 
    há um racional que determina ele completamente - tome $x$ de um componente, olhe para 
    um racional no intervalinho associado a $x$. Como os racionais são enumeráveis, esses componentes são enumeráveis.
\end{proof}

\begin{problem}
    \label{prob:l3:2}
\end{problem}
\begin{proof}
    Basta lembrar bem da definição da integral de Riemann para perceber que a de Lebesgue generaliza ela.
    Por Riemann, toda função contínua num compacto é integrável e suas somas inferiores e superiores convergem. Seja $f[a,b] \to \R^+$ contínua, temos
    $$\int_a^b f(x) dx = \lim_{|P|\to 0} L(P,f)$$
    onde $P$ é um pontilhamento do compacto $[a,b]$, $L(f,P)$ é a soma inferior de $f$ por $P$ e $|P|$ é 
    o tamanho do maior intervalo do pontilhamento. Podemos expressar
    $L(P,f)$ como uma soma, se $P$ é $(a = t_0, \dots t_n = b)$, temos 
    $$L(P,f) = \sum_{i = 1}^{n} m_i (t_i - t_{i-1}) = \sum_{i = 1}^{n} m_i \mu([t_{i-1},t_i))$$
    onde $m_i = \inf\{f(x); x \in [t_{i-1}, t_{i})\}$.

    Olhando para essa fórmula é claro perceber que cada pontilhamento $P$ está associado com uma 
    função simples menor ou igual a $f$. A ideia da prova é escolher uma sequência de pontilhamento $(P_n)_n$ (diádicos por exemplo)
    cujo módulo $|P_n|$  tende a 0 e cada pontilhamento é um 
    refinamento do anterior. Dessa forma, eles definirão uma sequência crescentes de funções que convergem para $f$, então,  
    aplicando o Teorema da Convergência Monotona [\ref{trm:conv_mon}], teremos o resultado para funções positivas. Para estender 
    para uma função $g$ com valores reais quaisquer, escrevemos $g = g^+ - g^-$ e usando a linearidade da integral de Lebesgue e Riemann
    teremos o resultado para integrais de $g$ também.

    De agora em diante, seja $f:[a,b] \to \R^+$ uma função contínua positiva.
    Seja $P_0 = \{a,b\}$, definiremos indutivamente uma sequência de refinamentos (os diádicos).
    Dado $P_n = \{t_0 < t_1 < \dots < t_m\}$, cortamos cada intervalo no meio, i.e.
    $$P_{n+1} = P_n \cup \bigg\{\frac{t_i + t_{i+1}}{2} : 0 \leq i < m\bigg\}$$
    Claramente, $|P_n| = (b-a)/2^n \to 0$ e, portanto, pelos teoremas da integral de Riemann,
    \begin{equation}
        \lim_{n \to \infty} L(P_n,f) \to \int_a^b fdx
    \end{equation}
    Agora definimos uma função step $s_n$ associada ao pontilhamento $P_n = \{t_0 < t_1 < \dots < t_m\}$,
    $$s_n(x) = \begin{cases}
        \inf\{f(a): a \in [t_i, t_{i+1})\} & \text{se } x \in [t_i, t_{i+1}]\; \text{para } i < m\\
        \inf\{f(a): a \in [t_{m-1}, t_{m}]\} & \text{se } x \in [t_{m-1}, t_{m}]\\
    \end{cases}$$
    Separar o último caso não é necessário, coloquei somente por clareza. Da forma que estão definidos,
    (1) os $s_n$ são funções simples. Como os $P_n$ são refinamentos, (2) $s_{n} \leq s_{n+1}$ e, além do mais,
    sendo $f$ uniformemente contínua em $[a,b]$, temos que (3) $s_n \to f$ uniformemente. Por [\ref{trm:conv_mon}],
    \begin{equation}
        \lim_{n\to \infty} \int_{[a,b]} s_n d\mu= \int_{[a,b]} f d\mu
    \end{equation}
    Mas por serem funções simples,
    \begin{equation}
        \int_{[a,b]} s_n d\mu = L(P_n, f)
    \end{equation}
    Juntando as equações 9, 10 e 11, obtemos o resultado.
    $$\int_a^b fdx = \lim_{n \to \infty} L(P_n, f) = \lim_{n\to\infty }\int_{[a,b]} s_n d\mu = \int_{[a,b]} fd\mu$$
    Para o caso de $g : [a,b] \to \R$ geral, temos 
    $$\int_{a}^{b} g dx = \int_{a}^{b} g^+ dx - \int_{a}^{b} g^- dx = \int_{[a,b]}g^+ d\mu - \int_{[a,b]}g^- d\mu = \int_{[a,b]}g d\mu$$ 
\end{proof}

\begin{problem}
    \label{prob:l3:3}
\end{problem}
\begin{proof}
    Esse exercício é similar ao \ref{prob:l2:5} da lista anterior, pelo menos a resolução do João. Sejam 
    \begin{align*}
        A_n &= \{x : |f_{n+1}(x) - f_n(x)| \geq \eps_n\}\\
        B_n &= \bigcup_{m \geq n} A_m
    \end{align*}
    Note que, sendo $\mu^*$ uma medida exterior,
    $$\mu^*(B_N) \leq \sum_{m = N}^{\infty} \mu^*(A_m) \to 0$$
    quando $N \to \infty$. Como os $B_n$ são encaixados, isso é o mesmo que dizer $\mu^*(\bigcap_n B_n) = 0$.
    Agora seja $x \not \in \bigcap_n^\infty B_n$, portanto existe $n_0$ tal que 
    $$ x \not \in B_{n_0}, B_{n_0 + 1}, \dots $$
    e, da mesma forma, 
    $$ x \not \in A_{n_0}, A_{n_0 + 1}, \dots$$
    Isso significa que para todo $m > n_0$, $|f_{m+1}(x) - f_{m}(x)| \leq \eps_m$. Como
    $\sum_n \eps_n < \infty$, pelo $M$-teste de Weierstrass, $f_n(x)$ converge. Ou seja, mostramos 
    que $(f_n(x))$ converge em quase todo ponto.
\end{proof}



\begin{problem}
    \label{prob:l3:4}
\end{problem}
% Antes de começar a resolução, vou delinear o percurso que tomaremos.

% Primeiro vamos provar o resultado para funções simples, isso é sempre uma boa ideia porque geralmente é fácil, para isso, precisaremos aproximar corretamente os conjuntos de Lebesgue. O próximo passo será
% estender para funções mensuráveis positivas e limitadas - o primeiro inimigo não trivial. Eliminado os obstáculos anteriores, 
% lançamos-nos contra funções mensuráveis positivas. 
% Daí faltará somente o golpe de misericórdia,
% separar funções complexas $f$ em $u^+ - u^- + iv^+ - iv^-$.

\begin{lemma}
    \label{lemm:borel_approx}
    Seja $E$ um conjunto de Lebesgue em $\R$, existe um boreliano $F_\sigma$
    tal que $F_\sigma \subset E$ e $\mu(E - F_\sigma) = 0$.
\end{lemma}
\begin{proof}
    Escreva $\R = \bigcup_n K_n$ para compactos $K_n$. Dado $E$ conjunto de Lebesgue, afirmo que para 
    qualquer $\eps > 0$, existe um aberto $V \supset E$, tal que $\mu(V - E) < \eps$. Escrevemos 
    $$E = \bigcup_n \mu(E \cap K_n)$$
    Como $\mu(E \cap K_n) < \infty$, pois $K_n$ é compacto, existe aberto $V_n \supset (E \cap K_n)$ com $\mu(V_n - (E \cap K_n)) < \eps/2^n$.
    Logo, tome $V = \bigcup_n V_n$ aberto, temos que $V - E \subset \bigcup_n V_n - (E \cap K_n)$, logo 
    $$\mu(V - E) \leq \sum_{n=1}^{\infty} \mu(V_n - E \cap K_n) < \eps$$
    Isso implica que podemos aproximar por fechados por dentro também, pois aplicando a afirmação 
    para $E^c$, conseguimos um aberto $W \supset E^c$ (com complementar $W^c \subset E$ fechado) tal que
    $$\mu(E - W^c) = \mu(W - E^c) < \eps$$
    Para $n$ natural, tome fechados $F_n \subset E$ com $\mu(E - F_n) < 1/n$. Seja $F_\sigma = \bigcup_n F_n \subset E$,
    então, sendo união enumerável de fechados, $F_\sigma$ é Boreliano e vale que, para todo $n$,
    $$\mu(E - F\sigma) \leq \mu(E - F_n) < 1/n$$
    logo, $\mu(E - F_\sigma) = 0$.
    
\end{proof}
\begin{lemma}
    Seja $s$ uma função simples de Lebesgue, existe uma função simples $h$ de Borel tal que $h \leq s$ e
    $s = h$ a.e.
\end{lemma}
\begin{proof}
    Como $s$ é simples, pode ser escrita da forma
    $$s(x) = \sum_{n = 1}^{N} a_n \mathds{1}_{E_n}(x)$$
    Para cada $E_n$, tome pelo lema anterior, um boreliano $B_n \subset E_n$ com $\mu(E_n - B_n) = 0$. A função
    $$h(x) = \sum_{n=1}^{N} a_n \mathds{1}_{B_n}(x)$$
    Claramente satisfaz que 
    $$\mu(\{x : s(x) \neq h(x)\}) = \mu\bigg(\bigcup_n E_n - B_n \bigg) \leq \sum_{n} \mu(E_n - B_n) = 0$$
\end{proof}

\begin{lemma}
    Seja $f : \R \to \R^+$ limitada de Lebesgue, então existe função de Borel $g$ tal que 
    $f = g$ a.e.
\end{lemma}
\begin{proof}
    Seja $f < M$ e escreva novamente $\R = \bigcup_n K_n$ união de compactos. Em cada compacto $K_m$,
    $$\int_{K_m} fd\mu \leq M\mu(K_m) < \infty$$
    Seja $(s_n) \leq f$ sequência de funções simples $s_n : K_m \to \R^+$ de Lebesgue que aproximam $f$, ou seja:
    $$\lim_{n\to\infty} \int_{K_m} s_n d\mu = \int_{K_m} f d\mu$$
    Agora, para cada $s_n$, pelo lema anterior, encontre $h_n = s_n \leq f$ a.e. com $h_n \leq s_n$ Borel simples. Note que
    $$\lim_{n\to\infty}\int_{K_m} h_n d\mu = \lim_{n\to\infty} \int_{K_m} s_n d\mu = \int_{K_m} fd\mu < \infty$$
    Afirmo que $H = \sup_n h_n \leq f$ Borel mensurável é igual a $f$ em quase todo ponto de $K_m$. Seja $A = \{x : H(x) \neq f(x)\}$, então
    $$A = \bigcup_k \{x: f(x) - H(x) > 1/k\}$$
    Suponha que $\mu(x : f(x) - H(x) > 1/k) = C > 0$ para algum $k$, vale que $\mu(x: f(x) - h_n(x) > 1/k) \geq C $ para todo $n$ e 
    portanto,
    $$\int_{K_m} f - h_n d\mu \geq \frac{C}{k} > 0$$
    o que é absurdo, pois $\int_{K_m} f - h_nd\mu \to 0$. Mostramos que  $\mu(x : f(x) - H(x) > 1/k) = 0$ para todo $k$ e 
    como consequência que $\mu(A) = 0$.

    Conseguimos o resultado para cada compacto $K_m$. Vamos estender para uma função definida em toda a reta.
    Para cada $m$, recupere função boreliana $H_m$ com $H_m \leq f$ e $H_m = f$ a.e em $K_m$ que valha $0$ em $K_m^c$ - nossa construção
    anterior permite fazer essa escolha. Tome $G = \sup_m H_m$, teremos que $G = f$ a.e . Para ver que isso vale, note que
    $G \leq f$ e escreva
    $$Q = \{x : G(x) \neq f(x)\} = \bigcup_m \{x \in K_m : G(x) < f(x)\} \subset \bigcup_m \{x \in K_m : H_m(x) < f(x)\}$$
    $Q$ é união enumerável de conjuntos de medida nula e portanto, tem medida nula.
\end{proof}

\begin{lemma}
    (Exercício)
    Seja $f : \R \to \C$ de Lebesgue, então existe função de Borel $g$ tal que 
    $f = g$ a.e.
\end{lemma}
\begin{proof}
    Vamos começar com funções $f : \R \to \R^+$ que não atingem infinito, $f(x) < \infty$ para todo $x$.
    Defina a sequência de funções Lebesgue $(F_n)_n : \R \to \R^+$ dadas por $F_n = \min(f,n)$. Como essas são
    todas limitadas, para cada $n$, pelo lema anterior, existe boreliana $G_n = F_n$ a.e com $G_n \leq F_n$.
    Defina a boreliana $G = \sup_n G_n \leq f$ (tomando $\sup$ pela milésima vez), temos que $G = F$ a.e .
    Para ver isso, assim como antes, seja 
    $$A = \{x : G(x) \neq f(x)\} = \{x : G(x) < f(x)\} = \bigcup_n \{x : G(x) < f(x) < n\}$$
    Mas, semelhantemente à prova anterior,
    $$\bigcup_n \{x : G(x) < f(x) < n\} \subset \bigcup_n \{x : G_n(x) < F_n(x)\}$$
    $A$ é, portanto, união enumerável de conjuntos de medida nula, logo $\mu(A) = 0$.

    Para dar o golpe de misericórdia nas funções reais positivas, se $\infty \in \{f(x) : x \in \R\}$, então seja $E = f^{-1}(\infty)$.
    Encontre, pelo lema \ref{lemm:borel_approx}, um boreliano $B \subset E$ com $\mu(E - B) = 0$ e uma boreliana 
    $G = f\mathds{1}_{E^c}$ a.e. Então, a função 
    $$H = G\mathds{1}_{E^c}  + \infty \cdot \mathds{1}_{B}$$
    é claramente igual a $f$ a.e.

    Para finalizar, estendendo para funções complexas, seja $f = u^+ - u^- + iv^+ - iv^-$ de Lebesgue,
    aproxime $u^+, u^-, v^+, v^-$ respectivamente por borelianas $a^+, a^-, b^+, b^-$ a.e. Então
    a função $g = a^+ - a^- + ib^+ - ib^-$ satisfaz que 
    $$\{x : f(x) \neq g(x)\} \subset [a^+ \neq u^+] \cup [a^- \neq u^-] \cup [b^+ \neq v^+] \cup [b^- \neq v^-] $$
    que é união de conjuntos de medida nula.
\end{proof}

\begin{problem}
    \label{prob:l3:5}
\end{problem}
Essa questão é a mais simples e vou tentar transcrever o desenho que soluciona ela em palavras.
A ideia aqui é que escrevamos uma sequência de funções trapezoidais em $[0,1]$ que vão ficando cada vez mais fininhas,
de forma que integrar sobre elas tenda a $0$, mas que cada ponto de $[0,1]$ chegue a valer $0$ e $1$ infinitas vezes.
\begin{proof}
    Vamos construir uma sequência de funções contínuas espertas $(F_n)_n$ em $[0,1]$.
    % $$F_{2^k}(x) = \begin{cases}
    %     1 &\text{se } x \leq 2^{-k}\\
    %     1 - 2^k(x - 2^{-k}) & \text{se } 2^{-k} \leq x \leq 2^{-k + 1}\\
    %     0 & \text{se }  x \geq 2^{-k + 1}
    % \end{cases}$$
    % Semelhantemente, definimos (para a última função do bloco)
    % $$F_{2^{k+1} - 1}(x) = \begin{cases}
    %     1 &\text{se } x \geq  1 - 2^{-k}\\
    %     2^k(x - (1 - 2^{-k})) & \text{se } 1 - 2^{-k} \geq x \geq 1 - 2^{-k + 1}\\
    %     0 & \text{se }  x \leq 1 - 2^{-k + 1}
    % \end{cases}$$
    % Ou mais econômicamente, $F_{2^{k+1} - 1}(x) = F_{2^k}(1 - x)$.
    Para $m \in \{2^k, 2^k + 1, \dots, 2^{k+1} - 1\}$, seja $n = m - 2^k$, definimos
    $$F_{m}(x) = \begin{cases}
        0 & \text{se }  x \leq (n - 1)2^{-k}\\
        2^k(x - (n-1)2^{-k}) & \text{se } (n-1)2^{-k} \leq x \leq n2^{-k}\\
        1 &\text{se } n2^{-k} \leq x \leq (n+1)2^{-k}\\
        1 - 2^k(x - (n+1)2^{-k}) & \text{se } (n+1)2^{-k} \leq x \leq (n+2)2^{-k}\\
        0 & \text{se }  x \geq (n+2)2^{-k}
    \end{cases}$$
    Onde óbiviamente $F_m$ está definida definida dessa forma quando os casos fazem sentido.
    Por exemplo quando $m = 2^k$, $n=0$ o primeiro e o segundo caso não aparecem. Quando 
    $m = 2^{k+1} - 1$, $n = 2^{k} - 1$, o quarto e o último não aparecem. Essas funções são claramente contínuas, 
    são trapézios que vão ficando cada vez menos espessos. Se $2^k \leq m < 2^{k+1}$, uma soma simples 
    sobre funções afins mostra que 
    $$\int_0^1 F_m dx \leq 2^{-k + 1} \to 0$$
    No entanto, é também fácil perceber que se $k > 2$, para qualquer $x \in [0,1]$, existem $2^k \leq N,M < 2^{k+1}$
    tais que $F_N(x) = 0$ e $F_M(x) = 1$. Portanto $F_m(x)$ não converge para nenhum ponto, mesmo que as integrais convirjam. 
\end{proof}

\begin{problem}
    \label{prob:l3:6}
\end{problem}
Esse é o problema mais legal, não acredito que consiguiria fazê-lo sem uma dica da professora Cynthia.
A única função não mensurável que conhecemos até agora é a característica de um conjunto não mensurável,
a ideia é tentar formar essa característica somente no liminf. Vamos fazer isso removendo pontualmente 
o complementar de um conjunto não mensurável infinitas vezes. \textbf{Obs:} A escolha esquista
de $(0,1]$ nos conjuntos a seguir é para facilitar a colagem que precisaremos fazer para construir a função $f$.
\begin{proof}
    Seja $T$ um conjunto não mensurável de $(0,1]$ e $T' = (0,1] - T$ seu complementar em $(0,1]$, note que $T'$ também é não mensurável.
    Vamos definir uma função $g(x,t) : \R\times(0,1] \to \{0,1\}$ que será a nossa ferramenta principal para construir  $f$.
    
    Dado $t$ fixo, se $t \in T'$, seja $A_t = (0,1] - \{t\}$, então definimos 
    $$g(x,t) \begin{cases}
        \mathds{1}_{A_t}(x) & \text{se } t \in T'\\
        \mathds{1}_{(0,1]}(x) & \text{se } t \not \in T'
    \end{cases}$$
    Note que, trivialmente, para todo $t \in (0,1]$, a função $g(t,x)$ -
    com a variável em $x$ - é mensurável, além do mais, sua integral sobre $x$ é claramente $1$. 
    
    Agora a  ideia é de alguma forma colar infinitas cópias de $g$ uma acima da outra.
    Separe $(0,1]$ na união disjunta:
    $$(0,1] = \bigcup_{n=1}^{\infty} (2^{-n}, 2^{-n + 1}]$$
    Definiremos $g_n(x,t) : \R \times (2^{-n},2^{-n + 1}] \to \{0,1\}$ da seguinte forma:
    $$g_n(x, t) = g(x, t2^n - 1)$$
    Por fim, defina $f :\R^2 \to \{0,1\}$ colando as $g_n$.
    $$f(x,t) = \begin{cases}
        \mathds{1}_{(0,1]}(x) & \text{se } t \leq 0 \ \lor \ t > 1\\
        g_n(x,t) & \text{se } t \in (2^{-n}, 2^{-n + 1}] 
    \end{cases}$$
    Afirmo que $f$ satisfaz as propriedades do exercício. Claramente,
    $\int_\R f(x,t) dx = 1$ para todo $t \in \R$, pois, fixando $t$, nossa 
    função é sempre a indicadora de $(0,1]$ salvo as vezes um único ponto. Para a segunda propriedade,
    vamos querer verificar que $h(x) = \liminf_{t \to 0} f(x,t)$ é justamente $\mathds{1}_T(x)$ que não é mensurável.
    Para isso note que se $x \in T \subset [0,1]$, então $g(x,t) = 1$ para todo $t$ e portanto, $g_n(x,t) = 1$ para qualquer $t$ também.
    Logo $f(x,t) = 1$ para todo $t$ e $h(x) = 1$. Se $x \in (0,1]^c$ então também, trivialmente $f(x,t) = 0$ para todo $t$
    e $h(x) = 0$. Agora, se $x \in T'$, para todo $n$, vale que 
    $$g_n\bigg(x, \frac{x + 1}{2^n}\bigg) = \mathds{1}_{A_x}(x) = 0$$
    e, para qualquer outro $t \in (2^{-n}, 2^{-n + 1}]$,
    $$g_n(x, t) = 1$$
    Em particular, sendo colagem desses valores, vale que para valores arbitrariamente pequenos de $t$ atingimos
    $f(x,t) = 0$ e portanto $h(x) = \liminf_{t \to 0} f(x,t) = 0$. Segue dos casos anteriores que $\liminf_{t \to 0} f(x,t)= \mathds{1}_T(x)$
    que não é mensurável.
\end{proof}

