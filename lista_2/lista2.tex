\section{Lista 2 (21/08/2025)}

Listagem de problemas:
\begin{enumerate}
    \item Exercício \ref{prob:l2:1} : \checkmark
    \item Exercício \ref{prob:l2:2} : \checkmark
    \item Exercício \ref{prob:l2:3} : \checkmark
    \item Exercício \ref{prob:l2:4} : \checkmark
    \item Exercício \ref{prob:l2:5} : \Frowny
    \item Exercício \ref{prob:l2:6} : \Frowny
    \item Exercício \ref{prob:l2:7} : \Frowny
\end{enumerate}

Para a solução de vários problemas dessa lista, utilizaremos os três principais teoremas vistos em aula até agora. Vamos enunciá-los.

\begin{theorem}
    \label{trm:conv_mon}
    (Convergência Monótona). Dada uma sequência crescente de funções mensuráveis $(f_n)_n$ de $X$ para $[0,\infty]$, satisfazendo:
    \begin{enumerate}[label=(\alph*)]
        \item $0 \leq f_1(x) \leq f_2(x) \leq \dots \leq \infty$ para todo $x \in X$
        \item $f_n(x) \to f(x)$ para todo $x \in X$
    \end{enumerate}
    Então $f$ é mensurável, e
    $$\int_{X} f_n\,d\mu \to \int_{X} f\,d\mu$$
\end{theorem}

\begin{theorem}
    \label{trm:lemma_fatou}
    (Lema de Fatou). Se $f_n : X \to [0,\infty]$ é mensurável, para cada $n$, então
    $$\int_X \big(\liminf_{n\to\infty} f_n\big)\,d\mu \leq \liminf_{n\to\infty} \int_X f_n \,d\mu$$
\end{theorem}

\begin{theorem}
    \label{trm:conv_dom}
    (Convergência Dominada). Se $\{f_n\}$ é uma sequência de funções mensuráveis complexas de $X$ tal que
    $$f(x) = \lim_{n\to\infty} f_n(x)$$
    existe para todo $x \in X$. Se existe uma função $g \in L^1(\mu)$ tal que, para todo $n$,
    $$|f_n(x)| \leq |g(x)|$$
    então $f \in L^1(\mu)$,
    $$\lim_{n\to\infty} \int_X |f_n - f|\,d\mu = 0$$
    e
    $$\lim_{n\to\infty} \int_X f_n \,d\mu = \int_X f \,d\mu$$
    
\end{theorem}

\begin{problem}
    \label{prob:l2:1}
\end{problem}

\begin{proof}
    
    Essa questão parece muito com a de interseção de conjuntos mensuráveis (Teorema 1.19
    Rudin). Se $f_1 \in L^1(\mu)$, como ela é positiva, existe $0 \leq M < \infty$ tal que $\int_X f_1 \, d\mu \leq M$. Defina $g_n$ mensurável por $g_n = f_1 - f_n$. Temos então que
    \begin{enumerate}[label=(\alph*)]
        \item $0 \leq g_1 \leq g_2 \leq \dots \leq \infty$
        \item $g_n(x) \to f_1(x) - f(x)$ para todo $x \in X$.
    \end{enumerate}
    Podemos aplicar convergência monótona [\ref{trm:conv_mon}] para encontrar:
    \begin{align}
    \lim_{n\to\infty} \int_X f_1 - f_n \, d\mu &= \int_X f_1 - f \, d\mu \\
    \lim_{n\to\infty} \bigg(\int_X f_1\,d\mu  - \int_X f_n \, d\mu\bigg) &= \int_X f_1\,d\mu  - \int_Xf \, d\mu \\
    \int_X f_1\,d\mu - \lim_{n\to\infty} \int_X f_n \, d\mu&= \int_X f_1\,d\mu  - \int_Xf \, d\mu \\
    \lim_{n\to\infty} \int_X f_n \, d\mu & = \int_Xf \, d\mu
    \end{align}
    Onde, crucialmente, usamos na última igualdade que $\int_X f_1 \leq M < \infty$.

    Se admitimos $f_1 \not \in L^1(\mu)$, a igualdade pode não valer. Defina $f_n(x) = 1/n$ para todo $x \in \R$.
    Temos que $f_n \to f = 0$, logo $\int_\R f\,d\mu = 0$, mas $\int_\R f_n\,d\mu = \infty$ para todo $n$.
\end{proof}

\begin{problem}
    \label{prob:l2:2}
\end{problem}
\begin{proof}
    
    Podemos usar diretamente o exemplo patológico da questão \ref{prob:l2:1}. Mas afim de fazer um diferente, seja $X = \{0,1\}$ com medida de contáveis. Defina as funções simples (e portanto mensuráveis) $h$ e $g$ dadas por
    \begin{equation*}
    h(x)=
    \begin{cases}
        0 & \text{if } x =0\\
        1 & \text{if } x =1
    \end{cases}
\end{equation*}
e
\begin{equation*}
    g(x)=
    \begin{cases}
        1 & \text{if } x =0\\
        0 & \text{if } x =1
    \end{cases}
\end{equation*}
Seja $\{f_n\}$, tal que $f_n = h$ se $n$ for par, e $f_n = g$ se $n$ for ímpar. Então claramente,
$\liminf_n f_n(x) = 0$ para todo $x$, mas $\int_X f_n\, d\mu = 1$ para todo $n$. Portanto
$$0 = \int_X (\liminf_n f_n)d\mu < \liminf_n \int_X f_nd\mu = 1$$
\end{proof}

\begin{problem}
    \label{prob:l2:3}
\end{problem}
\begin{proof}
    
    Esse problema é bem legal, envolve aproximar a função pontualmente e perceber que podemos aplicar nossos resultados. Antes de mais nada, dado $\alpha > 0$, defina $g_n : X \to [0,\infty]$, por 
    $$g_n(x) =n\log(1 + (f(x)/n)^\alpha) = n [\log(n^\alpha + f(x)^\alpha) - \log(n^\alpha)]$$
    $g_n$ é composição de uma função contínua por uma mensurável $f \geq 0$, é portanto mensurável e
    da forma que está definida, é positiva. $g(x) \in [0,\infty]$.

    Agora, vamos tentar estimar $g_n$. Pelo teorema do valor médio, dado $x$ fixo,
    $$\log(n^\alpha + f(x)^\alpha) - \log(n^\alpha) = \frac{f(x)^\alpha}{y}$$
    para $y \in (n^\alpha, n^\alpha + f(x)^\alpha)$.
    Então, temos
    $$n\frac{f(x)^\alpha}{n^\alpha + f(x)^\alpha} \leq g_n(x) \leq n\frac{f(x)^\alpha}{n^\alpha}$$
    E isso já é suficiente para dois casos do problema. 
    
    Se $\alpha = 1$,
    $$\frac{nf(x)}{n + f(x)} \leq g_n(x) \leq f(x)$$
    Como o lado esquerdo tende a $f(x)$, temos que $g_n(x) \to f(x)$. Além do mais, $g_n(x) \leq f(x) \in L^1(\mu)$, logo, por Convergência Dominada [\ref{trm:conv_dom}], temos que
    
    $$\lim_{n\to\infty}\int_X n\log(1 + (f(x)/n)^\alpha)d\mu = \lim_{n\to\infty}\int_X g_n(x)d\mu = \int_X fd\mu = c$$
    
    Se $\alpha < 1$, de $g_n(x) \geq nf(x)^\alpha / (n^\alpha + f(x)^\alpha) \to \infty$
    temos que
    $$ \infty = \liminf_{n\to\infty} nf(x)^\alpha / (n^\alpha + f(x)^\alpha) \leq \liminf_{n\to\infty} g_n(x)$$
    Usando o lema de Fatou [\ref{trm:lemma_fatou}], 
    $$\int_X \infty d\mu \leq \liminf_{n\to\infty}\int_X g_n d\mu = \liminf_{n \to \infty} \int_X n\log(1 + (f(x)/n)^\alpha) d\mu$$
    Se a medida $\mu$ não for identicamente $0$, então temos o resultado.

    
    Para $\alpha > 1$, terei que usar a dica do João, percebi que só conseguiria usar convergência dominada se $\int_X f^\alpha d\mu < \infty$, (mas não sabemos disso). Então precisamos fazer surgir $f$ sem expoentes na estimativa de $g_n$, para isso consideramos a sequências de desigualdades, válida para $f(x) \geq 0$, $\alpha > 1$.
    $$
        1 + x^\alpha \leq (1 + x)^\alpha \leq (e^x)^\alpha = e^{\alpha x}\\
    $$
    Tomando $\log$'s na equação,
    $$
        \log(1 + x^\alpha) \leq \log((1 + x)^\alpha) \leq \log((e^x)^\alpha) = \alpha x
    $$
    Portanto, $g_n(x) \leq n \alpha (f(x)/n) = \alpha f(x) \in L^1(\mu)$. Agora estamos muito felizes, pois sabemos que pontualmente (para cada $x$ fixo).
    
    $$g_n(x) \leq \frac{f(x)^\alpha}{n^{\alpha -1}} \to 0$$
    Logo, por convergência dominada [\ref{trm:conv_dom}],
    $$\lim_{n \to \infty} \int_X n\log(1 + (f(x)/n)^\alpha) d\mu= \lim_{n\to\infty} \int_X g_nd\mu = \int_X 0d\mu = 0$$
\end{proof}


\begin{problem}
    \label{prob:l2:4}
\end{problem}

\begin{proof}
    Essa questão segue quase imediatamante da série de desigualdades
    \begin{align}
        \lim_{n\to\infty} \bigg|\int_X f_n d\mu - \int_Xfd\mu \bigg| &\leq \lim_{n\to\infty} \int_X |f_n - f| d\mu\\
        &\leq \lim_{n\to\infty} \int_X \sup_x\{|f_n - f|\}d\mu\\
        &= \lim_{n\to\infty} \sup_x\{|f_n - f|\} \mu(X) \to 0
    \end{align}
    Onde em (7) usamos crucialmente que $\mu(X) < \infty$ e a sequência é uniformemente convergente.

    Se $\mu(X) = \infty$, segue exatamente da solução do exercício \ref{prob:l2:1}, com $f_n = 1/n$, $f = 0$, $X = \R$, que a hipótese não pode ser omitida.
\end{proof}

\begin{problem}
    \label{prob:l2:5}
\end{problem}

\begin{problem}
    \label{prob:l2:6}
\end{problem}

\begin{problem}
    \label{prob:l2:7}
\end{problem}