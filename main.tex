\documentclass{article}

\usepackage{amsmath,amssymb,amsthm,bbm,mathtools,calc,verbatim,enumitem,tikz,url,mathrsfs,cite,fullpage,hyperref,bm}
\usepackage{dsfont}
\usepackage{float}
\usepackage{subcaption}
%\usepackage{setspace}
\renewcommand{\baselinestretch}{1.1}
\addtolength{\footskip}{\baselineskip/2}

%\usepackage{showlabels}
\usepackage{comment}
\usepackage[english]{babel}

%No caso do livro o Tu, para decoplar seções de capitulos usamos:
% \usepackage{chngcntr}
% \counterwithout{section}{chapter}

\theoremstyle{definition}
\newtheorem{theorem}{Theorem}[section]
\newtheorem{lemma}[theorem]{Lemma}
\newtheorem{corollary}[theorem]{Corollary}
\newtheorem{prop}[theorem]{Proposition}
\newtheorem{observation}[theorem]{Observation}
\newtheorem{construction}[theorem]{Construction}

\newtheorem{definition}[theorem]{Definition}

\newtheorem{conjecture}[theorem]{Conjecture}
\newtheorem{question}[theorem]{Question}
\newtheorem{obs}[theorem]{Observation}
\newtheorem{claim}[theorem]{Claim}
\newtheorem{fact}[theorem]{Fact}
\newtheorem{problem}{Problem}[section]
\newtheorem{exercise}{Exercise}[section]
\newtheorem{remark}[theorem]{Remark}

% my custom problems
\newtheorem{innercustomexercise}{Exercise}
\newenvironment{customexercise}[1]
  {\renewcommand\theinnercustomexercise{#1}\innercustomexercise}
  {\endinnercustomexercise}

\newenvironment{clmproof}[1]{\begin{proof}[Proof of Claim~\ref{#1}]\let\qednow\qedsymbol\renewcommand{\qedsymbol}{}}{\; \qednow \end{proof}}

\newcommand\N{\mathbb{N}}
\newcommand\R{\mathbb{R}}
\newcommand\Z{\mathbb{Z}}
\newcommand\cA{\mathcal{A}}
\newcommand\cB{\mathcal{B}}
\newcommand\cN{\mathcal{N}}
\newcommand\cP{\mathcal{P}}
\newcommand\cQ{\mathcal{Q}}
\newcommand\cZ{\mathcal{Z}}
\newcommand\rN{\tilde{N}}
\newcommand\cT{\mathcal{T}}
\newcommand\cE{\mathcal{E}}
\def\Pr{\mathbb{P}}
\def\cS{\mathcal{S}}
\newcommand\Ex{\mathbb{E}}
\newcommand\id{\hbox{$1\mkern-6.5mu1$}}
\newcommand\lcm{\operatorname{lcm}}
\newcommand\eps{\varepsilon}
\newcommand{\floor}[1]{\lfloor #1 \rfloor}
\newcommand{\ceil}[1]{\lceil #1 \rceil}
\newcommand{\prob}{\begin{problem} \end{problem}}
\newcommand{\exer}{\begin{exercise} \end{exercise}}
\newcommand{\cexer}[1]{\begin{customprob}{#1}\end{customprob}}


\renewcommand{\leq}{\leqslant}
\renewcommand{\geq}{\geqslant}
\renewcommand{\le}{\leqslant}
\renewcommand{\ge}{\geqslant}
\renewcommand{\to}{\rightarrow}
\renewcommand{\Re}{\re}

\def\ds{\displaystyle}

\def\eps{\varepsilon}
\def\p{\partial}

\def\HH{\mathcal{H}}
\def\E{\mathbb{E}}
\def\C{\mathbb{C}}
\def\cM{\mathcal{M}}
\def\cF{\mathcal{F}}
\def\cI{\mathcal{I}}
\def\R{\mathbb{R}}
\def\bS{\mathbb{S}}
\def\bH{\mathbb{H}}
\def\Z{\mathbb{Z}}
\def\N{\mathbb{N}}
\def\PP{\mathbb{P}}
\def\1{\mathbbm{1}}
\def\l{}
\def\k{\kappa}
\def\w{\omega}
\def\s{\sigma}
\def\t{\theta}
\def\a{\alpha}
\def\g{\gamma}
\def\z{\zeta}
\def\zbar{\bar{z}}
\def\<{\langle}
\def\>{\rangle}
%\def\endproof{{\hfill $\square$} }
\def\Xt{\widetilde{X}}
\def\Pt{\widetilde{P}}

\def\cN{\mathcal{N}}
\def\cC{\mathcal{C}}
\def\cD{\mathcal{D}}
\def\cR{\mathcal{R}}
\def\cB{\mathcal{B}}
\def\cG{\mathcal{G}}
\def\EE{\mathbb{E}}
\def\FF{\mathbb{F}}
\def\T{\mathbb{T}}
\def\cA{\mathcal{A}}
\def\cQ{\mathcal{Q}}
\def\cC{\mathcal{C}}
\def\F{\mathbb{F}}
\def\tm{\tilde{\mu}}
\def\ts{\tilde{\sigma}}
\def\Q{\mathcal{Q}}
\def\vp{\varphi}

\hypersetup{
	colorlinks=true,
	linkcolor=blue,
	urlcolor=blue,
}

\pagestyle{plain}
\author{henrique}
\title{Listas de Medida}

\begin{document}
\maketitle

\tableofcontents
\setcounter{section}{-1}

\section{Introdução e Notação}
Ao decorrer do curso, vou escrever minhas resoluções dos exercícios nesse arquivo. Tem alguns motivos para isso:
\begin{enumerate}
	\item Posso reutilizar resultados passados.
	\item Está tudo organizado se um futuro henrique quiser rever.
	\item Há uma certo senso de completude no final do curso.
\end{enumerate}
Por isso, peço desculpa ao monitor e a professora se não gostarem desse formato, me avisem que eu posso separar os arquivos.
O código fonte pode ser encontrado em \url{https://github.com/hnrq104/medida}.

Eu vou tentar usar uma notação menos esotérica, mas, ás vezes, uma vontade maior se expressa. Por enquanto encontrei os segundos usos no texto:
\begin{enumerate}
	\item $\bigcup_n$ ou $\sum_n$. Quando o intervalo de índices não está específicado, geralmente estou tomando a união ou o somatório
	sobre os naturais positivos.
	\item $[n] = \{1,2,\dots, n\}$ é uma notação de combinatória que uso bastante.
	\item "Observação" é algo que estou com muita preguiça de tentar provar (se estiver correto), 
	espero poder perguntar em monitorias se a prova é necessária.
\end{enumerate}



\section{Lista 1 (15/08/2025)}

Listagem de problemas:
\begin{enumerate}
    \item Exercício \ref{prob:l1:1} : \checkmark
    \item Exercício \ref{prob:l1:2} : \checkmark
    \item Exercício \ref{prob:l1:3} : \checkmark
    \item Exercício \ref{prob:l1:4} : \checkmark
    \item Exercício \ref{prob:l1:5} : \checkmark
    \item Exercício \ref{prob:l1:6} : \checkmark
\end{enumerate}

\begin{problem}
    \label{prob:l1:1}
\end{problem}

Esse problema é muito bonitinho e a resposta é negativa. Para resolvê-lo, precisamos da seguinte observação.
\begin{observation}
    \label{obs:1:enum}
    A coleção de uniões enumeráveis de infinitos conjuntos não vazios disjuntos é não-enumerável. (quase um trava-língua)
\end{observation}
\begin{proof}
    Sejam $\{E_1, E_2, \dots, E_n, \dots\}$ infinitos conjuntos satisfazendo
    \begin{enumerate}
        \item $E_i \neq \varnothing \ \forall i \in \N$ 
        \item $E_i \cap E_j = \varnothing \ \forall i \neq j \in \N$ 
    \end{enumerate}
    A função $f: \{0,1\}^{\N} \to \mathbb{P}(\bigcup_{i \in \N} E_i)$ dada por
    $$f(a_1, a_2, \dots, a_n, \dots) = \bigcup_{i \in \N} B_i$$
    onde $B_i = \varnothing$ se $a_i = 0$ e $B_i = E_i$ se $a_i = 1$ é injetiva. Como $2^{\N}$ é não enumerável, temos o resultado.
\end{proof}
Agora podemos dar continuidade a resolução.
\begin{prop}
    Seja $(X,M)$ uma $\sigma$-algebra infinita, então $M$ é não enumerável.
\end{prop}

O que fiz antes tava errado :( . Segue a solução do João. 
\begin{proof}
    Suponha que $M$ seja enumerável. Para cada $x \in X$, defina os conjuntos minimais $E_x$ de $M$,
    $$E_x := \bigcap\limits_{\{E_k \in M\, ;\, x \in E_k\}} E_k$$
    Como $M$ é enumerável, essas interseções são enumeráveis e portanto pertencem a $M$. 
    
    A ideia da prova é mostrar que os $E_x$ particionam o espaço em conjuntos disjuntos, depois ver que eles geram $M$ e 
    concluir que, como $M$ é infinita, devem existir infinitos deles.


    
    Vamos mostrar que o espaço é particionado em conjuntos disjuntos. Sejam $x,y$ tal que $E_x \neq E_y$, afirmo que
    $x \not \in E_y$. Suponha que $x \in E_y$, então pela definição de $E_x$, $E_x \subseteq E_y$. Do mesmo modo, se 
    $y \in E_x$, então $E_y \subseteq E_x$ e $E_x = E_y$ (contradição). Se $y \not \in E_x$, então $E_y - E_x$ é um conjunto disjunto de $x$ que contém $y$,
    logo $x \not \in E_y$. Para provar que a interseção é vazia, verificamos que se $x \not \in E_y$, então $E_x \subset E_x - E_y$, 
    portanto $E_x \cap E_y = \varnothing$.

    O próximo passo é mostrar que esses conjuntos geram $M$. Afirmo que dado $E \in M$
    $$E = \bigcup\limits_{E_x \subset E} E_x$$
    
    Claramente temos $\bigcup_{E_x \subset E} E_x \subset E$. Para a outra inclusão, seja $x \in E$, então $x \in E_x \subset E$, pois $E$ é um conjunto
    que contém $x$.

    Agora para matar a questão. Suponha que houvessem somente finitos $E_x$, digamos $n$. Haveria somente $2^n$ possíveis uniões desses conjuntos, como
    eles geram $M$ e $M$ é infinita temos uma contradição. Portanto, existem infinitos $E_x$ disjuntos não vazios, $M$ contém todas suas enumeráveis coleções,
    pela observação \ref{obs:1:enum}, $M$ não pode ser contável.


    % Suponha que não existe partição infinita de $X$ em conjuntos disjuntos. Dizemos que uma partição $\{A_1,\dots, A_m\}$ de $X$ é maximal, se não podemos refiná-la, i.e.
    % não existe $\varnothing \neq B \subsetneq A_j$ para algum $A_j$ (Se não poderiamos trocar $A_j$ por $A_j \cap B$ e $A_j - B$, que daria uma partição maior).
    % Seja $C$ uma partição maximal de conjuntos disjuntos não vazios de $M$, cuja existência é garantida pelo Lema de Zorn. Suponha que $|C| = m < \infty$ sendo da seguinte forma:
    % $$C = \{A_1, A_2, \dots, A_m \} \quad A_i \cap A_j = \varnothing \ \forall i\neq j$$
    % $C$ é, portanto, uma partição finita de $X$ em conjuntos disjuntos. Olhemos para todas as possíveis uniões finitas de elementos de $C$ ($2^m$ delas considerando $\varnothing$), 
    % como são finitos, existe um conjunto $B \in M$ diferente delas. Repartindo $B$, temos:
    % $$B = \bigcup_{\{i\ \mid \ A_i \cap B \neq \varnothing \}} B \cap A_i$$
    % Como $B \neq \varnothing$ e escolhemos $B$ a evitar uniões de elementos de $C$ sabemos que
    % $$\varnothing \neq B \subsetneq \bigcup_{\{i\ \mid \ A_i \cap B \neq \varnothing \}} A_i$$
    % Portanto, existe $A_j$ com $A_j - B \neq \varnothing$ e $A_j \cap B \neq \varnothing$. Podemos então trocar $A_j$ por $A_j - B$ e $A_j \cap B$,
    % contradizendo a maximalidade.
\end{proof}

\begin{problem}
\label{prob:l1:2}
\end{problem}

\begin{proof}
    Dada uma sequência de funções mensuráveis $\{f_n\} : X \to [-\infty, \infty]$, sabemos que $I(x) = \liminf_n f_n(x)$ e $S(x) = \limsup_n f_n(x)$ são mensuráveis.
    Além disso, para cada $x \in X$, a sequência $f_n(x)$ converge se e somente se ela não tem valores tendendo para o infinito e $I(x) = S(x)$.
    A partir dessa caracterização, definimos o conjunto $A$ tal que:
    $$A = I^{-1}((-\infty, \infty)) \cap S^{-1}((-\infty, \infty))$$
    Isso é, $A$ é o conjunto de pontos de $X$ tal que a sequência $f_n(x)$ é limitada. Note que, como $I$ e $S$ são mensuráveis, $A$ é interseção
    de conjuntos mensuráveis de $X$, logo é mensurável. Em particular, as funções $\mathds{1}_A$ e $\mathds{1}_{A^c}$ são mensuráveis. Como vimos que somas e multiplicações
    de funções mensuráveis é mensurável, podemos definir uma $H$ mensurável dada por:
    $$H(x) = \mathds{1}_{A^c}(x) + \mathds{1}_A(x) \cdot S(x) - \mathds{1}_A(x) \cdot I(x) $$
    Os pontos em que as $f_n$ convergem é então dado por pelo conjunto mensurável $H^{-1}(\{0\})$. Para confirmar essa afirmação, note que
    se $H(y) = 0$, então $H(y) \neq 1$, logo $y \not \in A^c$. Temos que $y \in A$, $I(y) \in (-\infty,\infty)$ e $S(y) \in (-\infty, \infty)$, 
    logo $S(y) - I(y)$ está bem definido (nenhum dos dois é infinito de mesmo sinal) e, temos, $S(y) = I(y)$, ou seja, a sequência $f_n(y)$ converge.
    Se $H(z) \neq 0$, ou $z \in A^c$, e portanto a sequência $f_n(z)$ não é limitada, ou $S(z) \neq I(z)$ e portanto, a sequência não converge.  
\end{proof}

\begin{problem}
    \label{prob:l1:3}
\end{problem}

\begin{prop}
    $\mathcal{M}$ é $\sigma$-álgebra. Isso é, satisfaz:
    \begin{enumerate}
        \item $X \in \mathcal{M}$
        \item $E \in \mathcal{M} \Rightarrow E^c \in \mathcal{M}$
        \item $\{E_1, E_2, \dots, E_n, \dots\} \subset \mathcal{M} \Rightarrow \bigcup_{i \in \N} E_i \in \mathcal{M}$
    \end{enumerate}
\end{prop}
\begin{proof}
    (1). $X^c = \varnothing$ enumerável, logo $X \in \mathcal{M}$. (2). Por construção. (3). Dados
    contáveis conjuntos $C = \{E_1, E_2, \dots\}$ em $\mathcal{M}$, separe-os em incontáveis ($A$) e contáveis ($B$) de forma que:
    $$\{E_1, E_2, \dots\} = A \cup B =  \{E_i: E_i \ \text{incontável}\} \cup \{E_j: E_j \ \text{contável}\}$$
    Seja então $H = \bigcup_{i \in \N} E_i = \bigcup_{A_i \in A} A_i \cup  \bigcup_{B_i \in B} B_i$. Note que se $A$ não é vazio, i.e. contém
    ao menos um elemento $A_j$, então $H^c \subset (A_j)^c$ que é contável. Se $A$ é vazio, então $H = \bigcup_{B_i \in B} B_i$ é uma união enumerável de conjuntos
    contáveis, logo $H$ é contável.
\end{proof}
\begin{prop}
    $\mu$ é uma medida em $\mathcal{M}$.
\end{prop}
\begin{proof}
    Como $\varnothing$ é contável, $\mu(\varnothing) = 0$, além disso, $\mu(E) \in \{0,1\} \subset [0,\infty]$.
    Então, basta mostrar que, dada uma coleção disjunta $C = \{E_1, E_2, \dots\} \subset \mathcal{M}$,
    $$\sum_{E_i \in C} \mu(E_i) = \mu\big( \bigcup_{E_i \in C} E_i \big) $$
    Como anteriormente escreva $C = A \cup B$, onde $A$ são os conjuntos incontáveis e $B$ são os contáveis.
    Se $A$ for vazio, todos os conjuntos $E_i$ são contáveis, então a união deles é contável e temos 
    que os dois lados da equação são $0$. Se $A$ possui um conjunto $E_j$, ele obrigatóriamente é o único em 
    $A$, pois, como os $E_i$ são disjuntos, todos os outros $E_i$'s estão contidos em $(E_j)^c$ que é enumerável.
    Portanto, o somatório da esquerda possui somente um valor diferente de $0$, vulgo $\mu(E_j) = 1$ e a união
    da direita contém $E_j$ não enumerável, portanto vale $1$ também.
\end{proof}

\begin{problem}
    \label{prob:l1:4}
\end{problem}

Vou supor de antemão que as medidas $\mu_1$ e $\mu_2$ são positivas, há um passo em que precisaremos dessa hipótese.
\begin{prop}
    $\mu(E) = \inf \{\mu_1(E \cap F) +  \mu_2(E - F) \, : \, F \in \mathcal{M}\}$ é uma medida positiva.
\end{prop}
\begin{proof}
    (1) Sendo ínfimo de valores positivos, claramante $\mu(E) \in [0,\infty]$. (2) $\mu(\varnothing) \leq \mu_1(\varnothing) + \mu_2(\varnothing) = 0$ . (3) Considere em $\mathcal{M}$ uma sequência qualquer de conjuntos 
    disjuntos $(E_n)_{n \in \N}$. Queremos mostrar que:
    $$\mu \bigg(\bigcup_{n} E_n \bigg) = \sum_{n} \mu(E_n)$$

    Considere
    \begin{align*}
        \mu \bigg(\bigcup_{n} E_n \bigg) &= \inf \bigg\{\mu_1\bigg( \bigcup_n E_n \cap F \bigg) + \mu_2\bigg( \bigcup_n E_n - F \bigg) \, : \, F \in \mathcal{M} \bigg\}\\
        &= \inf \bigg\{ \sum_n \mu_1(E_n \cap F) + \sum_n \mu_2(E_n - F) \, : \, F \in \mathcal{M} \bigg\}\\
        &= \inf \bigg\{ \sum_n ( \mu_1(E_n \cap F) + \mu_2(E_n - F) ) \, : \, F \in \mathcal{M} \bigg\}
    \end{align*}
    Onde usamos na segunda igualdade o fato de que somatórios de valores positivos podem ser rearranjados (e portanto a hipótese de que $\mu_1$ e $\mu_2$ são positivas).
    Agora note que para todo $F \in \mathcal{M}$ e qualquer $E_i$ temos
    $$\inf\{\mu_1(E_i \cap \tilde{F}) + \mu_2(E_i - \tilde{F}) \, : \, \tilde{F} \in \mathcal{M}\} \leq \mu_1(E_i \cap F) + \mu_2(E_2 - F)$$
    Logo, termo a termo,
    $$\sum_n \inf\{\mu_1(E_n \cap \tilde{F}) + \mu_2(E_n - \tilde{F}) \, : \, \tilde{F} \in \mathcal{M}\} \leq \sum_n \mu_1(E_n \cap F) + \mu_2(E_n - F)$$
    i.e.
    $$\sum_{n} \mu(E_n) \leq \sum_n \mu_1(E_n \cap F) + \mu_2(E_n - F)$$
    Como vale para todo $F$, temos, tomando ínfimos
    $$\sum_{n} \mu(E_n) \leq \mu \bigg(\bigcup_{n} E_n \bigg)$$
    Falta provar que $\mu\big(\bigcup_{n} E_n \big) \leq \sum_{n} \mu(E_n) $. Ou, mais sorreteiramente, que para 
    todo $\eps > 0$,
    $$ \mu\big(\bigcup_{n} E_n \big) \leq \bigg(\sum_{n} \mu(E_n)\bigg) + \eps = \sum_{n} (\mu(E_n) + \eps/2^n )$$
    Para cada $n$, existe $F_n \in \mathcal{M}$ tal que $\mu(E_n) \leq \mu_1(E_n \cap F_n) + \mu_2(E_n - F_n) + \eps/2^n$.
    Tome $F = \bigcup_n (F_i \cap E_i)$. Então,
    \begin{align*}
        \mu\big(\bigcup_n E_n \big) &\leq \mu_1\big( \bigcup_n E_n \cap F\big) +\mu_2\big( \bigcup_n E_n - F\big)\\
        &= \sum_n \mu_1(E_n \cap F) + \mu_2(E_n - F)\\
        &= \sum_n \mu_1(E_n \cap F_n) + \mu_2(E_n - F_n)\\
        &\leq \sum_{n} (\mu(E_n) + \eps/2^n)\\
        &= \sum_{n} \mu(E_n) + \eps
    \end{align*}
    Onde na segunda igualdade usamos o fato de que os $E_n$ são disjuntos entre si e na segunda desigualdade, a definição dos $F_n$.
    Como isso vale para todo $\eps > 0$, tomando $\eps \to 0$, encontramos $ \big(\bigcup_{n} E_n \big) = \sum_{n} \mu(E_n)$.
\end{proof}

\begin{prop}
    $\mu$ é a maior medida menor que $\mu_1$ e $\mu_2$.
\end{prop}
\begin{proof}
    Para todo $E \in \mathcal{M}$, $\mu(E) \leq \mu_1(E \cap X) + \mu_2(E - X) = \mu_1(E)$, 
    semelhantemente, $\mu(E) \leq \mu_1(E \cap \varnothing) + \mu_2(E - \varnothing) = \mu_2(E)$. Portanto,
    $\mu(E) \leq \min(\mu_1(E), \mu_2(E))$. Agora seja $\tilde{\mu}$ qualquer medida também menor que $\mu_1$ e $\mu_2$.
    Então, para todo $F$,
    $$\tilde{\mu}(E) = \tilde{\mu}(E\cap F) + \tilde{\mu}(E - F) \leq \mu_1(E \cap F) + \mu_2(E - F)$$
    Como isso vale para qualquer $F$, tomando ínfimos, temos
    $$\tilde{\mu}(E) \leq \mu(E)$$
\end{proof}

\begin{problem}
    \label{prob:l1:5}
\end{problem}

Será útil para a letra (b) duas proposições importantes.
\begin{prop} \label{prop:borel_induzido}
    Seja $(X,\mathcal{T})$ espaço topológico e $\mathcal{B}_X$ sua $\sigma$-álgebra de Borel. Se $Y \in \mathcal{B}_X$ é um conjunto mensurável, então na topologia induzida $(Y,\mathcal{T} \cap Y)$, a $\sigma$-álgebra de Borel $\mathcal{B}_Y$ coincide com o conjunto $\{E \cap Y \, : \, E \in \mathcal{B}_X\}$.
\end{prop}
\begin{proof} 
    Vamos provar primeiro que $\mathcal{B}_Y \subseteq \{E \cap Y \, : \, E \in \mathcal{B}_X\}$.
    Então basta mostrar que o segundo conjunto é uma $\sigma$-álgebra que contem os abertos.
    Ele claramente contem os abertos de $Y$, pois esses são $Y \cap U$ para $U$ aberto de $X$ que são mensuráveis. Falta verificar as propriedades de $\sigma$-álgebra. (1) $Y$ pertence ao conjunto, pois $Y = Y \cap Y$ e $Y \in \mathcal{B}_X$. (2) Se $A \cap Y$ é um elemento, então
    $(A \cap Y)^c_Y = Y - (A \cap Y) = Y \cap A^c$ também pertence, pois $A^c \in \mathcal{B}_X$. Sejam $(A_1\cap Y, A_2 \cap Y, \dots)$ elementos do conjunto, então $\bigcup_n (A_n \cap Y) = (\bigcup_n A_n) \cap Y$ pertence também. Isso finaliza a primeira parte.

    Falta mostrar que $\{E \cap Y \, : \, E \in \mathcal{B}_X\} \subseteq \mathcal{B}_Y$, isso não foi trivial para mim (tive que rever a prova do João na internet); Essa proposição é equivalente a $\{E \in \mathcal{B}_X \, : \, E \cap Y \in \mathcal{B}_Y\} = \mathcal{B}_X$, que segue diretamente do fato que o conjunto da esquerda é uma $\sigma$-álgebra que contém os abertos de $X$. Vamos provar as propriedades: (1) $X \in \mathcal{B}_X$ e $X \cap Y = Y \in \mathcal{B}_Y$, logo $X$ pertence ao conjunto. (2) Se $E \in \mathcal{B}_X$ é tal que $E \cap Y \in \mathcal{B}_Y$ então $E^c \in \mathcal{B}_X$ tem $E^c \cap Y = Y - E \in \mathcal{B}_Y$.
    (3) $\bigcup_n E_n$ é tal que $E_n \cap Y \in \mathcal{B}_Y$, então $\bigcup_n E_n \cap Y = \bigcup_n (E_n \cap Y) \in \mathcal{B}_Y$. Portanto, o conjunto que definimos é uma $\sigma$-álgebra. Falta verificar que contém os abertos de $X$, mas segue trivialmente do fato que os abertos de $Y$ são justamente $U \cap Y \in \mathcal{B}_Y$.
\end{proof}
    O próximo é bem óbvio, estou inserindo por completude. (Mas é meio chato de provar).
\begin{prop} \label{prop:borel_homeomorphism}
    Se $(X,\mathcal{T})$ e $(Y,\mathcal{S})$ são espaços topológicos homeomorfos por um mapa $f:X \to Y$, então vale que $\mathcal{B}_Y = \{f(E_x) \, : \, E_x \in \mathcal{B}_X\}$ 
    % e $\mathcal{B}_X = \{f^{-1}(E_y) \, : \, E_y \in \mathcal{B}_Y\}$ 
    onde $\mathcal{B}_X$ e $\mathcal{B}_Y$ são as $\sigma$-álgebras de Borel em $X$ e $Y$ respectivamente. 
\end{prop}

\begin{proof}
    Seja $\mathcal{M} = \{f(E_x) \, : \, E_x \in \mathcal{B}_X\}$. $\mathcal{M}$ claramente contém
    os abertos de $Y$ pois se $U \subset Y$ é aberto, $f^{-1}(U)$ é aberto pertencente a $\mathcal{B}_X$, logo $U = f(f^{-1}(U)) \in \mathcal{M}$. Vamos mostrar que é $\sigma$-álgebra.
    (1) $Y = f(X) \in \mathcal{M}$. (2) $f(E_x) \in \mathcal{M} \Rightarrow (f(E_x))^c = f(E_x^c) \in \mathcal{M}$. (3) $\bigcup_n f(E_x^n) = f(\bigcup_n E_x^n) \in \mathcal{M}$. Portanto mostramos que $\mathcal{B}_Y \subseteq \mathcal{M}$. Agora para mostrar que $\mathcal{M} \in \mathcal{B}_Y$ usamos mensurabilidade, sendo $f^{-1}$ contínua, ela é mensurável entre $\sigma$-álgebras de Borel, logo se $A = f^{-1}(E_x) \in \mathcal{M}$, então, como $E_x \in \mathcal{B}_X$, $A \in \mathcal{B}_Y$. E terminamos a demonstração. 
\end{proof}
Agora as letras (a) e (b) saem quase que de graça.
\begin{enumerate}[label=(\alph*)]
    \item \begin{proof}
        Translações $f:\R^d \to \R^d$ tal que $f(x) = f(x) + a$ para algum $a \in \R^d$ são homeomorfismo de $\R^d$ para si próprio. Por \ref{prop:borel_homeomorphism}, se $E \in \mathcal{B}^d$ então $f(E) = E + a \in \mathcal{B}^d$.
    \end{proof}
    \item \begin{proof}
        Vamos fazer para seções horizontais, a prova para seções verticais é análoga.
        Para $y \in \R$ e $E$ Borel de $\R^2$, definimos $E_y = E \cap (\R \times \{y\})$ boreliano. Note que 
        $\R \times \{y\} = \bigcap_n \R \times \{a - 1/n, a + 1/n\}$ é Borel de $\R^2$. Pela proposição
        \ref{prop:borel_induzido}, $\{E_y \, : \, E \in \mathcal{B}^2\}$ é a $\sigma$-álgebra de Borel induzida por $\R \times \{y\}$, mas esse conjunto é trivialmente homeomorfo a reta $\R$ com 
        a projeção na primeira coordenada. Portanto, por \ref{prop:borel_homeomorphism}, as seções horizontais definidas na questão são borelianos da reta.
    \end{proof}
\end{enumerate}




\begin{problem}
    \label{prob:l1:6}
\end{problem}

Essa questão é bem divertida, estende dupla contagem para medidas.

\begin{prop}
    (a) Os conjuntos $H_k$ são mensuráveis.
\end{prop}
\begin{proof}
    Como cada $E_i$ é mensurável, definimos as funções mensuráveis $(f_n)_{n \in \N}$ por:
    $$f_n(x) = \sum_{j = 1}^{n} \mathds{1}_{E_j}(x)$$
    Então $0 \leq f_1 \leq \dots \leq f_n \leq f_{n+1} \leq \dots \leq \infty$ é uma sequência crescente
    mensurável, e portanto: 
    $$F(x) = \sup_n f_n(x) = \lim_n f_n(x) = \#\{n \, : \, x \in E_n \}$$
    é uma função mensurável. Temos que $H_k = F^{-1}([k,\infty])$ é um conjunto mensurável.
\end{proof}

Agora vem a parte difícil. Para mostrar a letra (b), esqueçamos $(E_n)_{n \in \N}$ por enquanto, foquemos
em $(E_n)_{n \in [N]}$ finitos.

\begin{definition}
    Dada uma sequência finita $(E_n)_{n \in [N]}$ de conjuntos de $\mathcal{M}$. Sejam $H_k^{(N)}$ da seguinte forma:
    $$H_k^{(N)} = \{x \in X \, : \, \#\{n \, : \, x \in E_n\} \geq k\}$$
    A mesma definição dos $H_k$, mas para uma coleção finita de no máximo $N$ conjuntos.
\end{definition}

\begin{observation}
    Temos propriedades simples, que independem de $N$ e da coleção escolhida:
    \begin{enumerate}
        \item Exatamente como na letra (a), $H_k^{(N)}$ é mensurável.
        \item $H_0^{(N)} = X$
        \item $H_{k+1}^{(N)} \subseteq H_{k}^{(N)}$
        \item $H_{N+1}^{(N)} = \varnothing$, pois nenhum elemento pertence em mais que $N$ conjuntos.
    \end{enumerate}
\end{observation}

Para qualquer sequência infinita $(E_n)_{n\in\N}$ definimos os $H_k^{(N)}$ para os primeiros $N$ conjuntos da sequência.

\begin{lemma}
    \label{lemm:hk}
    Seja $(E_n)_{n \in \N}$ mensuráveis. Para todo $N \in \N$, vale:
    $$\sum_{k=1}^{N} \mu(H_k^{(N)}) = \sum_{k=1}^{N} \mu(E_k)$$
\end{lemma}
\begin{proof}
    Vamos seguir por indução. Para $N = 1$, temos de graça que $E_1 = H_1^{(1)}$, logo $\mu(H_1^{(1)}) = \mu(E_1)$.
    Suponha que o resultado vale para $N$ e olhemos para o caso $N+1$.
    \begin{align*}
        \sum_{n=1}^{N+1} \mu(E_n) &= \mu(E_{N+1}) + \sum_{n=1}^{N} \mu(E_n)\\
        &= \mu(E_{N+1}) + \sum_{n=1}^{N} \mu(H_n^{(N)})
    \end{align*}
    Onde usamos a hipótese de indução na segunda igualdade.

    Note que $H_k^{(N+1)} = (H_k^{(N)} - E_{N+1}) \cup (H_{k-1}^{(N)} \cap E_{N+1})$. Pois se $x \in X$ aparece em $k$ conjuntos de $(E_n)_{n \in [N+1]}$,
    ou ele aparece em $k$ dos primeiros $N$ conjuntos, ou aparece em $E_{N+1}$ e pelo menos $k-1$ outros dos primeiros $N$.
    Para aproveitar dessa observação, podemos reescrever o somatório

    \begin{align*}
        \mu(E_{N+1}) + \sum_{n=1}^{N} \mu(H_n^{(N)}) &= \mu(E_{N+1}) + \sum_{n=1}^{N} \mu(H_n^{(N)} - E_{N+1}) + \mu(H_n^{(N)} \cap E_{N+1})\\
        &= \mu(E_{N+1}) + \sum_{n=1}^{N+1} \mu(H_n^{(N)} - E_{N+1}) + \mu(H_n^{(N)} \cap E_{N+1})\\
    \end{align*}
    Já que $H_{N+1}^{(N)} = \varnothing$. Agora escrevemos $\mu(E_{N+1}) = \mu(H_0^{(N)} \cap E_{N+1}) = \mu(X \cap E_{N+1})$ e reindexamos 
    cada termo da direita no somatório, obtendo
    \begin{align*}
        \sum_{n=1}^{N+1} \mu(E_n) &= \sum_{n=1}^{N+1} \mu(H_n^{(N)} - E_{N+1}) + \mu(H_{n-1}^{(N)} \cap E_{N+1})\\
        &= \sum_{k=1}^{N+1} \mu(H_k^{(N+1)})
    \end{align*}
    Provando o passo indutivo.
\end{proof}

Estamos quase finalizados, sentimos até vontade de passar o limite em \ref{lemm:hk} e obter o resultado, mas isso por si só não é suficiente.
\begin{prop}
    $\sum_{k=1}^{\infty} \mu(H_k) = \sum_{k=1}^{\infty} \mu(E_k)$
\end{prop}
\begin{proof}
    Tomando limites em $N$ no Lema \ref{lemm:hk}, temos que
    $$\lim_{N \to \infty} \sum_{k=1}^{N} \mu(H_k^{(N)}) = \sum_{k=1}^{\infty} \mu(E_k)$$
    Para obter o resultado, vamos mostrar que
    $$\lim_{N \to \infty} \sum_{k=1}^{N} \mu(H_k^{(N)}) = \sum_{k=1}^{\infty} \mu(H_k)$$
    Note que, pela definição dos $H_k^{(N)}$, temos uma sequência crescente $H_k^{(1)} \subseteq H_k^{(2)} \subseteq \dots \subseteq H_k$, tal que
    $$\bigcup_{n=1}^{\infty} H_k^{(n)} = H_k$$
    Por conta das inclusões $H_N^{(N)} \subseteq H_N$ e $\mu$ ser uma medida positiva, temos, termo a termo,
    $\mu(H_N^{(N)}) \leq \mu(H_N)$. Portanto, já temos um lado da igualdade.
    $$\lim_{N \to \infty} \sum_{k=1}^{N} \mu(H_k^{(N)}) \leq \sum_{k=1}^{\infty} \mu(H_k)$$
    Para o outro lado, observamos que como $H_N^{(N)} \to H_N$ são mensuráveis, $\lim_{n\to\infty} \mu(H_N^{(N)}) = \mu(H_N)$.
    Portanto, para cada $M > 0$, 
    $$\lim_{N \to \infty} \sum_{k=1}^{N} \mu(H_k^{(N)}) \geq \lim_{N \to \infty} \sum_{k=1}^{M} \mu(H_k^{N}) = \sum_{k=1}^{M} \mu(H_k)$$
    Como isso vale para todo $M$, $\lim_{N \to \infty} \sum_{k=1}^{N} \mu(H_k^{(N)}) \geq \sum_{k=1}^{\infty} \mu(H_k)$.
\end{proof}

\end{document}
