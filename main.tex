\documentclass{article}

\usepackage{amsmath,amssymb,amsthm,bbm,mathtools,calc,verbatim,enumitem,tikz,url,mathrsfs,cite,fullpage,hyperref,bm, marvosym}
\usepackage{dsfont}
\usepackage{float}
\usepackage{subcaption}
%\usepackage{setspace}
\renewcommand{\baselinestretch}{1.1}
\addtolength{\footskip}{\baselineskip/2}

%\usepackage{showlabels}
\usepackage{comment}
\usepackage[english]{babel}

%No caso do livro o Tu, para decoplar seções de capitulos usamos:
% \usepackage{chngcntr}
% \counterwithout{section}{chapter}

\theoremstyle{definition}
\newtheorem{theorem}{Theorem}[section]
\newtheorem{lemma}[theorem]{Lemma}
\newtheorem{corollary}[theorem]{Corollary}
\newtheorem{prop}[theorem]{Proposition}
\newtheorem{observation}[theorem]{Observation}
\newtheorem{construction}[theorem]{Construction}

\newtheorem{definition}[theorem]{Definition}

\newtheorem{conjecture}[theorem]{Conjecture}
\newtheorem{question}[theorem]{Question}
\newtheorem{obs}[theorem]{Observation}
\newtheorem{claim}[theorem]{Claim}
\newtheorem{fact}[theorem]{Fact}
\newtheorem{problem}{Problem}[section]
\newtheorem{exercise}{Exercise}[section]
\newtheorem{remark}[theorem]{Remark}

% my custom problems
\newtheorem{innercustomexercise}{Exercise}
\newenvironment{customexercise}[1]
  {\renewcommand\theinnercustomexercise{#1}\innercustomexercise}
  {\endinnercustomexercise}

\newenvironment{clmproof}[1]{\begin{proof}[Proof of Claim~\ref{#1}]\let\qednow\qedsymbol\renewcommand{\qedsymbol}{}}{\; \qednow \end{proof}}

\newcommand\N{\mathbb{N}}
\newcommand\R{\mathbb{R}}
\newcommand\Z{\mathbb{Z}}
\newcommand\cA{\mathcal{A}}
\newcommand\cB{\mathcal{B}}
\newcommand\cN{\mathcal{N}}
\newcommand\cP{\mathcal{P}}
\newcommand\cQ{\mathcal{Q}}
\newcommand\cZ{\mathcal{Z}}
\newcommand\rN{\tilde{N}}
\newcommand\cT{\mathcal{T}}
\newcommand\cE{\mathcal{E}}
\def\Pr{\mathbb{P}}
\def\cS{\mathcal{S}}
\newcommand\Ex{\mathbb{E}}
\newcommand\id{\hbox{$1\mkern-6.5mu1$}}
\newcommand\lcm{\operatorname{lcm}}
\newcommand\eps{\varepsilon}
\newcommand{\floor}[1]{\lfloor #1 \rfloor}
\newcommand{\ceil}[1]{\lceil #1 \rceil}
\newcommand{\prob}{\begin{problem} \end{problem}}
\newcommand{\exer}{\begin{exercise} \end{exercise}}
\newcommand{\cexer}[1]{\begin{customprob}{#1}\end{customprob}}


\renewcommand{\leq}{\leqslant}
\renewcommand{\geq}{\geqslant}
\renewcommand{\le}{\leqslant}
\renewcommand{\ge}{\geqslant}
\renewcommand{\to}{\rightarrow}
\renewcommand{\Re}{\re}

\def\ds{\displaystyle}

\def\eps{\varepsilon}
\def\p{\partial}

\def\HH{\mathcal{H}}
\def\E{\mathbb{E}}
\def\C{\mathbb{C}}
\def\cM{\mathcal{M}}
\def\cF{\mathcal{F}}
\def\cI{\mathcal{I}}
\def\R{\mathbb{R}}
\def\bS{\mathbb{S}}
\def\bH{\mathbb{H}}
\def\Z{\mathbb{Z}}
\def\N{\mathbb{N}}
\def\PP{\mathbb{P}}
\def\1{\mathbbm{1}}
\def\l{}
\def\k{\kappa}
\def\w{\omega}
\def\s{\sigma}
\def\t{\theta}
\def\a{\alpha}
\def\g{\gamma}
\def\z{\zeta}
\def\zbar{\bar{z}}
\def\<{\langle}
\def\>{\rangle}
%\def\endproof{{\hfill $\square$} }
\def\Xt{\widetilde{X}}
\def\Pt{\widetilde{P}}

\def\cN{\mathcal{N}}
\def\cC{\mathcal{C}}
\def\cD{\mathcal{D}}
\def\cR{\mathcal{R}}
\def\cB{\mathcal{B}}
\def\cG{\mathcal{G}}
\def\EE{\mathbb{E}}
\def\FF{\mathbb{F}}
\def\T{\mathbb{T}}
\def\cA{\mathcal{A}}
\def\cQ{\mathcal{Q}}
\def\cC{\mathcal{C}}
\def\F{\mathbb{F}}
\def\tm{\tilde{\mu}}
\def\ts{\tilde{\sigma}}
\def\Q{\mathcal{Q}}
\def\vp{\varphi}

\hypersetup{
	colorlinks=true,
	linkcolor=blue,
	urlcolor=blue,
}

\pagestyle{plain}
\author{henrique}
\title{Listas de Medida}

\begin{document}
\maketitle

\tableofcontents
\setcounter{section}{-1}

\section{Introdução e Notação}
Ao decorrer do curso, vou escrever minhas resoluções dos exercícios nesse arquivo. Tem alguns motivos para isso:
\begin{enumerate}
	\item Posso reutilizar resultados passados.
	\item Está tudo organizado se um futuro henrique quiser rever.
	\item Há uma certo senso de completude no final do curso.
\end{enumerate}
Por isso, peço desculpa ao monitor e a professora se não gostarem desse formato, me avisem que eu posso separar os arquivos.
O código fonte pode ser encontrado em \url{https://github.com/hnrq104/medida}.

Eu vou tentar usar uma notação menos esotérica, mas, ás vezes, uma vontade maior se expressa. Por enquanto encontrei os segundos usos no texto:
\begin{enumerate}
	\item $\bigcup_n$ ou $\sum_n$. Quando o intervalo de índices não está específicado, geralmente estou tomando a união ou o somatório
	sobre os naturais positivos.
	\item $[n] = \{1,2,\dots, n\}$ é uma notação de combinatória que uso bastante.
	\item "Observação" é algo que estou com muita preguiça de tentar provar (se estiver correto), 
	espero poder perguntar em monitorias se a prova é necessária.
	\item "a.e" significa "almost everywhere", geralmente sou contra anglicanismos descenessários, mas
	esse já está encravado em meu vocabulário.
\end{enumerate}


%LISTA 1
\section{Lista 1 (15/08/2025)}

Listagem de problemas:
\begin{enumerate}
    \item Exercício \ref{prob:l1:1} : \checkmark
    \item Exercício \ref{prob:l1:2} : \checkmark
    \item Exercício \ref{prob:l1:3} : \checkmark
    \item Exercício \ref{prob:l1:4} : \checkmark
    \item Exercício \ref{prob:l1:5} : \checkmark
    \item Exercício \ref{prob:l1:6} : \checkmark
\end{enumerate}

\begin{problem}
    \label{prob:l1:1}
\end{problem}

Esse problema é muito bonitinho e a resposta é negativa. Para resolvê-lo, precisamos da seguinte observação.
\begin{observation}
    \label{obs:1:enum}
    A coleção de uniões enumeráveis de infinitos conjuntos não vazios disjuntos é não-enumerável. (quase um trava-língua)
\end{observation}
\begin{proof}
    Sejam $\{E_1, E_2, \dots, E_n, \dots\}$ infinitos conjuntos satisfazendo
    \begin{enumerate}
        \item $E_i \neq \varnothing \ \forall i \in \N$ 
        \item $E_i \cap E_j = \varnothing \ \forall i \neq j \in \N$ 
    \end{enumerate}
    A função $f: \{0,1\}^{\N} \to \mathbb{P}(\bigcup_{i \in \N} E_i)$ dada por
    $$f(a_1, a_2, \dots, a_n, \dots) = \bigcup_{i \in \N} B_i$$
    onde $B_i = \varnothing$ se $a_i = 0$ e $B_i = E_i$ se $a_i = 1$ é injetiva. Como $2^{\N}$ é não enumerável, temos o resultado.
\end{proof}
Agora podemos dar continuidade a resolução.
\begin{prop}
    Seja $(X,M)$ uma $\sigma$-algebra infinita, então $M$ é não enumerável.
\end{prop}

O que fiz antes tava errado :( . Segue a solução do João. 
\begin{proof}
    Suponha que $M$ seja enumerável. Para cada $x \in X$, defina os conjuntos minimais $E_x$ de $M$,
    $$E_x := \bigcap\limits_{\{E_k \in M\, ;\, x \in E_k\}} E_k$$
    Como $M$ é enumerável, essas interseções são enumeráveis e portanto pertencem a $M$. 
    
    A ideia da prova é mostrar que os $E_x$ particionam o espaço em conjuntos disjuntos, depois ver que eles geram $M$ e 
    concluir que, como $M$ é infinita, devem existir infinitos deles.


    
    Vamos mostrar que o espaço é particionado em conjuntos disjuntos. Sejam $x,y$ tal que $E_x \neq E_y$, afirmo que
    $x \not \in E_y$. Suponha que $x \in E_y$, então pela definição de $E_x$, $E_x \subseteq E_y$. Do mesmo modo, se 
    $y \in E_x$, então $E_y \subseteq E_x$ e $E_x = E_y$ (contradição). Se $y \not \in E_x$, então $E_y - E_x$ é um conjunto disjunto de $x$ que contém $y$,
    logo $x \not \in E_y$. Para provar que a interseção é vazia, verificamos que se $x \not \in E_y$, então $E_x \subset E_x - E_y$, 
    portanto $E_x \cap E_y = \varnothing$.

    O próximo passo é mostrar que esses conjuntos geram $M$. Afirmo que dado $E \in M$
    $$E = \bigcup\limits_{E_x \subset E} E_x$$
    
    Claramente temos $\bigcup_{E_x \subset E} E_x \subset E$. Para a outra inclusão, seja $x \in E$, então $x \in E_x \subset E$, pois $E$ é um conjunto
    que contém $x$.

    Agora para matar a questão. Suponha que houvessem somente finitos $E_x$, digamos $n$. Haveria somente $2^n$ possíveis uniões desses conjuntos, como
    eles geram $M$ e $M$ é infinita temos uma contradição. Portanto, existem infinitos $E_x$ disjuntos não vazios, $M$ contém todas suas enumeráveis coleções,
    pela observação \ref{obs:1:enum}, $M$ não pode ser contável.


    % Suponha que não existe partição infinita de $X$ em conjuntos disjuntos. Dizemos que uma partição $\{A_1,\dots, A_m\}$ de $X$ é maximal, se não podemos refiná-la, i.e.
    % não existe $\varnothing \neq B \subsetneq A_j$ para algum $A_j$ (Se não poderiamos trocar $A_j$ por $A_j \cap B$ e $A_j - B$, que daria uma partição maior).
    % Seja $C$ uma partição maximal de conjuntos disjuntos não vazios de $M$, cuja existência é garantida pelo Lema de Zorn. Suponha que $|C| = m < \infty$ sendo da seguinte forma:
    % $$C = \{A_1, A_2, \dots, A_m \} \quad A_i \cap A_j = \varnothing \ \forall i\neq j$$
    % $C$ é, portanto, uma partição finita de $X$ em conjuntos disjuntos. Olhemos para todas as possíveis uniões finitas de elementos de $C$ ($2^m$ delas considerando $\varnothing$), 
    % como são finitos, existe um conjunto $B \in M$ diferente delas. Repartindo $B$, temos:
    % $$B = \bigcup_{\{i\ \mid \ A_i \cap B \neq \varnothing \}} B \cap A_i$$
    % Como $B \neq \varnothing$ e escolhemos $B$ a evitar uniões de elementos de $C$ sabemos que
    % $$\varnothing \neq B \subsetneq \bigcup_{\{i\ \mid \ A_i \cap B \neq \varnothing \}} A_i$$
    % Portanto, existe $A_j$ com $A_j - B \neq \varnothing$ e $A_j \cap B \neq \varnothing$. Podemos então trocar $A_j$ por $A_j - B$ e $A_j \cap B$,
    % contradizendo a maximalidade.
\end{proof}

\begin{problem}
\label{prob:l1:2}
\end{problem}

\begin{proof}
    Dada uma sequência de funções mensuráveis $\{f_n\} : X \to [-\infty, \infty]$, sabemos que $I(x) = \liminf_n f_n(x)$ e $S(x) = \limsup_n f_n(x)$ são mensuráveis.
    Além disso, para cada $x \in X$, a sequência $f_n(x)$ converge se e somente se ela não tem valores tendendo para o infinito e $I(x) = S(x)$.
    A partir dessa caracterização, definimos o conjunto $A$ tal que:
    $$A = I^{-1}((-\infty, \infty)) \cap S^{-1}((-\infty, \infty))$$
    Isso é, $A$ é o conjunto de pontos de $X$ tal que a sequência $f_n(x)$ é limitada. Note que, como $I$ e $S$ são mensuráveis, $A$ é interseção
    de conjuntos mensuráveis de $X$, logo é mensurável. Em particular, as funções $\mathds{1}_A$ e $\mathds{1}_{A^c}$ são mensuráveis. Como vimos que somas e multiplicações
    de funções mensuráveis é mensurável, podemos definir uma $H$ mensurável dada por:
    $$H(x) = \mathds{1}_{A^c}(x) + \mathds{1}_A(x) \cdot S(x) - \mathds{1}_A(x) \cdot I(x) $$
    Os pontos em que as $f_n$ convergem é então dado por pelo conjunto mensurável $H^{-1}(\{0\})$. Para confirmar essa afirmação, note que
    se $H(y) = 0$, então $H(y) \neq 1$, logo $y \not \in A^c$. Temos que $y \in A$, $I(y) \in (-\infty,\infty)$ e $S(y) \in (-\infty, \infty)$, 
    logo $S(y) - I(y)$ está bem definido (nenhum dos dois é infinito de mesmo sinal) e, temos, $S(y) = I(y)$, ou seja, a sequência $f_n(y)$ converge.
    Se $H(z) \neq 0$, ou $z \in A^c$, e portanto a sequência $f_n(z)$ não é limitada, ou $S(z) \neq I(z)$ e portanto, a sequência não converge.  
\end{proof}

\begin{problem}
    \label{prob:l1:3}
\end{problem}

\begin{prop}
    $\mathcal{M}$ é $\sigma$-álgebra. Isso é, satisfaz:
    \begin{enumerate}
        \item $X \in \mathcal{M}$
        \item $E \in \mathcal{M} \Rightarrow E^c \in \mathcal{M}$
        \item $\{E_1, E_2, \dots, E_n, \dots\} \subset \mathcal{M} \Rightarrow \bigcup_{i \in \N} E_i \in \mathcal{M}$
    \end{enumerate}
\end{prop}
\begin{proof}
    (1). $X^c = \varnothing$ enumerável, logo $X \in \mathcal{M}$. (2). Por construção. (3). Dados
    contáveis conjuntos $C = \{E_1, E_2, \dots\}$ em $\mathcal{M}$, separe-os em incontáveis ($A$) e contáveis ($B$) de forma que:
    $$\{E_1, E_2, \dots\} = A \cup B =  \{E_i: E_i \ \text{incontável}\} \cup \{E_j: E_j \ \text{contável}\}$$
    Seja então $H = \bigcup_{i \in \N} E_i = \bigcup_{A_i \in A} A_i \cup  \bigcup_{B_i \in B} B_i$. Note que se $A$ não é vazio, i.e. contém
    ao menos um elemento $A_j$, então $H^c \subset (A_j)^c$ que é contável. Se $A$ é vazio, então $H = \bigcup_{B_i \in B} B_i$ é uma união enumerável de conjuntos
    contáveis, logo $H$ é contável.
\end{proof}
\begin{prop}
    $\mu$ é uma medida em $\mathcal{M}$.
\end{prop}
\begin{proof}
    Como $\varnothing$ é contável, $\mu(\varnothing) = 0$, além disso, $\mu(E) \in \{0,1\} \subset [0,\infty]$.
    Então, basta mostrar que, dada uma coleção disjunta $C = \{E_1, E_2, \dots\} \subset \mathcal{M}$,
    $$\sum_{E_i \in C} \mu(E_i) = \mu\big( \bigcup_{E_i \in C} E_i \big) $$
    Como anteriormente escreva $C = A \cup B$, onde $A$ são os conjuntos incontáveis e $B$ são os contáveis.
    Se $A$ for vazio, todos os conjuntos $E_i$ são contáveis, então a união deles é contável e temos 
    que os dois lados da equação são $0$. Se $A$ possui um conjunto $E_j$, ele obrigatóriamente é o único em 
    $A$, pois, como os $E_i$ são disjuntos, todos os outros $E_i$'s estão contidos em $(E_j)^c$ que é enumerável.
    Portanto, o somatório da esquerda possui somente um valor diferente de $0$, vulgo $\mu(E_j) = 1$ e a união
    da direita contém $E_j$ não enumerável, portanto vale $1$ também.
\end{proof}

\begin{problem}
    \label{prob:l1:4}
\end{problem}

Vou supor de antemão que as medidas $\mu_1$ e $\mu_2$ são positivas, há um passo em que precisaremos dessa hipótese.
\begin{prop}
    $\mu(E) = \inf \{\mu_1(E \cap F) +  \mu_2(E - F) \, : \, F \in \mathcal{M}\}$ é uma medida positiva.
\end{prop}
\begin{proof}
    (1) Sendo ínfimo de valores positivos, claramante $\mu(E) \in [0,\infty]$. (2) $\mu(\varnothing) \leq \mu_1(\varnothing) + \mu_2(\varnothing) = 0$ . (3) Considere em $\mathcal{M}$ uma sequência qualquer de conjuntos 
    disjuntos $(E_n)_{n \in \N}$. Queremos mostrar que:
    $$\mu \bigg(\bigcup_{n} E_n \bigg) = \sum_{n} \mu(E_n)$$

    Considere
    \begin{align*}
        \mu \bigg(\bigcup_{n} E_n \bigg) &= \inf \bigg\{\mu_1\bigg( \bigcup_n E_n \cap F \bigg) + \mu_2\bigg( \bigcup_n E_n - F \bigg) \, : \, F \in \mathcal{M} \bigg\}\\
        &= \inf \bigg\{ \sum_n \mu_1(E_n \cap F) + \sum_n \mu_2(E_n - F) \, : \, F \in \mathcal{M} \bigg\}\\
        &= \inf \bigg\{ \sum_n ( \mu_1(E_n \cap F) + \mu_2(E_n - F) ) \, : \, F \in \mathcal{M} \bigg\}
    \end{align*}
    Onde usamos na segunda igualdade o fato de que somatórios de valores positivos podem ser rearranjados (e portanto a hipótese de que $\mu_1$ e $\mu_2$ são positivas).
    Agora note que para todo $F \in \mathcal{M}$ e qualquer $E_i$ temos
    $$\inf\{\mu_1(E_i \cap \tilde{F}) + \mu_2(E_i - \tilde{F}) \, : \, \tilde{F} \in \mathcal{M}\} \leq \mu_1(E_i \cap F) + \mu_2(E_2 - F)$$
    Logo, termo a termo,
    $$\sum_n \inf\{\mu_1(E_n \cap \tilde{F}) + \mu_2(E_n - \tilde{F}) \, : \, \tilde{F} \in \mathcal{M}\} \leq \sum_n \mu_1(E_n \cap F) + \mu_2(E_n - F)$$
    i.e.
    $$\sum_{n} \mu(E_n) \leq \sum_n \mu_1(E_n \cap F) + \mu_2(E_n - F)$$
    Como vale para todo $F$, temos, tomando ínfimos
    $$\sum_{n} \mu(E_n) \leq \mu \bigg(\bigcup_{n} E_n \bigg)$$
    Falta provar que $\mu\big(\bigcup_{n} E_n \big) \leq \sum_{n} \mu(E_n) $. Ou, mais sorreteiramente, que para 
    todo $\eps > 0$,
    $$ \mu\big(\bigcup_{n} E_n \big) \leq \bigg(\sum_{n} \mu(E_n)\bigg) + \eps = \sum_{n} (\mu(E_n) + \eps/2^n )$$
    Para cada $n$, existe $F_n \in \mathcal{M}$ tal que $\mu(E_n) \leq \mu_1(E_n \cap F_n) + \mu_2(E_n - F_n) + \eps/2^n$.
    Tome $F = \bigcup_n (F_i \cap E_i)$. Então,
    \begin{align*}
        \mu\big(\bigcup_n E_n \big) &\leq \mu_1\big( \bigcup_n E_n \cap F\big) +\mu_2\big( \bigcup_n E_n - F\big)\\
        &= \sum_n \mu_1(E_n \cap F) + \mu_2(E_n - F)\\
        &= \sum_n \mu_1(E_n \cap F_n) + \mu_2(E_n - F_n)\\
        &\leq \sum_{n} (\mu(E_n) + \eps/2^n)\\
        &= \sum_{n} \mu(E_n) + \eps
    \end{align*}
    Onde na segunda igualdade usamos o fato de que os $E_n$ são disjuntos entre si e na segunda desigualdade, a definição dos $F_n$.
    Como isso vale para todo $\eps > 0$, tomando $\eps \to 0$, encontramos $ \big(\bigcup_{n} E_n \big) = \sum_{n} \mu(E_n)$.
\end{proof}

\begin{prop}
    $\mu$ é a maior medida menor que $\mu_1$ e $\mu_2$.
\end{prop}
\begin{proof}
    Para todo $E \in \mathcal{M}$, $\mu(E) \leq \mu_1(E \cap X) + \mu_2(E - X) = \mu_1(E)$, 
    semelhantemente, $\mu(E) \leq \mu_1(E \cap \varnothing) + \mu_2(E - \varnothing) = \mu_2(E)$. Portanto,
    $\mu(E) \leq \min(\mu_1(E), \mu_2(E))$. Agora seja $\tilde{\mu}$ qualquer medida também menor que $\mu_1$ e $\mu_2$.
    Então, para todo $F$,
    $$\tilde{\mu}(E) = \tilde{\mu}(E\cap F) + \tilde{\mu}(E - F) \leq \mu_1(E \cap F) + \mu_2(E - F)$$
    Como isso vale para qualquer $F$, tomando ínfimos, temos
    $$\tilde{\mu}(E) \leq \mu(E)$$
\end{proof}

\begin{problem}
    \label{prob:l1:5}
\end{problem}

Será útil para a letra (b) duas proposições importantes.
\begin{prop} \label{prop:borel_induzido}
    Seja $(X,\mathcal{T})$ espaço topológico e $\mathcal{B}_X$ sua $\sigma$-álgebra de Borel. Se $Y \in \mathcal{B}_X$ é um conjunto mensurável, então na topologia induzida $(Y,\mathcal{T} \cap Y)$, a $\sigma$-álgebra de Borel $\mathcal{B}_Y$ coincide com o conjunto $\{E \cap Y \, : \, E \in \mathcal{B}_X\}$.
\end{prop}
\begin{proof} 
    Vamos provar primeiro que $\mathcal{B}_Y \subseteq \{E \cap Y \, : \, E \in \mathcal{B}_X\}$.
    Então basta mostrar que o segundo conjunto é uma $\sigma$-álgebra que contem os abertos.
    Ele claramente contem os abertos de $Y$, pois esses são $Y \cap U$ para $U$ aberto de $X$ que são mensuráveis. Falta verificar as propriedades de $\sigma$-álgebra. (1) $Y$ pertence ao conjunto, pois $Y = Y \cap Y$ e $Y \in \mathcal{B}_X$. (2) Se $A \cap Y$ é um elemento, então
    $(A \cap Y)^c_Y = Y - (A \cap Y) = Y \cap A^c$ também pertence, pois $A^c \in \mathcal{B}_X$. Sejam $(A_1\cap Y, A_2 \cap Y, \dots)$ elementos do conjunto, então $\bigcup_n (A_n \cap Y) = (\bigcup_n A_n) \cap Y$ pertence também. Isso finaliza a primeira parte.

    Falta mostrar que $\{E \cap Y \, : \, E \in \mathcal{B}_X\} \subseteq \mathcal{B}_Y$, isso não foi trivial para mim (tive que rever a prova do João na internet); Essa proposição é equivalente a $\{E \in \mathcal{B}_X \, : \, E \cap Y \in \mathcal{B}_Y\} = \mathcal{B}_X$, que segue diretamente do fato que o conjunto da esquerda é uma $\sigma$-álgebra que contém os abertos de $X$. Vamos provar as propriedades: (1) $X \in \mathcal{B}_X$ e $X \cap Y = Y \in \mathcal{B}_Y$, logo $X$ pertence ao conjunto. (2) Se $E \in \mathcal{B}_X$ é tal que $E \cap Y \in \mathcal{B}_Y$ então $E^c \in \mathcal{B}_X$ tem $E^c \cap Y = Y - E \in \mathcal{B}_Y$.
    (3) $\bigcup_n E_n$ é tal que $E_n \cap Y \in \mathcal{B}_Y$, então $\bigcup_n E_n \cap Y = \bigcup_n (E_n \cap Y) \in \mathcal{B}_Y$. Portanto, o conjunto que definimos é uma $\sigma$-álgebra. Falta verificar que contém os abertos de $X$, mas segue trivialmente do fato que os abertos de $Y$ são justamente $U \cap Y \in \mathcal{B}_Y$.
\end{proof}
    O próximo é bem óbvio, estou inserindo por completude. (Mas é meio chato de provar).
\begin{prop} \label{prop:borel_homeomorphism}
    Se $(X,\mathcal{T})$ e $(Y,\mathcal{S})$ são espaços topológicos homeomorfos por um mapa $f:X \to Y$, então vale que $\mathcal{B}_Y = \{f(E_x) \, : \, E_x \in \mathcal{B}_X\}$ 
    % e $\mathcal{B}_X = \{f^{-1}(E_y) \, : \, E_y \in \mathcal{B}_Y\}$ 
    onde $\mathcal{B}_X$ e $\mathcal{B}_Y$ são as $\sigma$-álgebras de Borel em $X$ e $Y$ respectivamente. 
\end{prop}

\begin{proof}
    Seja $\mathcal{M} = \{f(E_x) \, : \, E_x \in \mathcal{B}_X\}$. $\mathcal{M}$ claramente contém
    os abertos de $Y$ pois se $U \subset Y$ é aberto, $f^{-1}(U)$ é aberto pertencente a $\mathcal{B}_X$, logo $U = f(f^{-1}(U)) \in \mathcal{M}$. Vamos mostrar que é $\sigma$-álgebra.
    (1) $Y = f(X) \in \mathcal{M}$. (2) $f(E_x) \in \mathcal{M} \Rightarrow (f(E_x))^c = f(E_x^c) \in \mathcal{M}$. (3) $\bigcup_n f(E_x^n) = f(\bigcup_n E_x^n) \in \mathcal{M}$. Portanto mostramos que $\mathcal{B}_Y \subseteq \mathcal{M}$. Agora para mostrar que $\mathcal{M} \in \mathcal{B}_Y$ usamos mensurabilidade, sendo $f^{-1}$ contínua, ela é mensurável entre $\sigma$-álgebras de Borel, logo se $A = f^{-1}(E_x) \in \mathcal{M}$, então, como $E_x \in \mathcal{B}_X$, $A \in \mathcal{B}_Y$. E terminamos a demonstração. 
\end{proof}
Agora as letras (a) e (b) saem quase que de graça.
\begin{enumerate}[label=(\alph*)]
    \item \begin{proof}
        Translações $f:\R^d \to \R^d$ tal que $f(x) = f(x) + a$ para algum $a \in \R^d$ são homeomorfismo de $\R^d$ para si próprio. Por \ref{prop:borel_homeomorphism}, se $E \in \mathcal{B}^d$ então $f(E) = E + a \in \mathcal{B}^d$.
    \end{proof}
    \item \begin{proof}
        Vamos fazer para seções horizontais, a prova para seções verticais é análoga.
        Para $y \in \R$ e $E$ Borel de $\R^2$, definimos $E_y = E \cap (\R \times \{y\})$ boreliano. Note que 
        $\R \times \{y\} = \bigcap_n \R \times \{a - 1/n, a + 1/n\}$ é Borel de $\R^2$. Pela proposição
        \ref{prop:borel_induzido}, $\{E_y \, : \, E \in \mathcal{B}^2\}$ é a $\sigma$-álgebra de Borel induzida por $\R \times \{y\}$, mas esse conjunto é trivialmente homeomorfo a reta $\R$ com 
        a projeção na primeira coordenada. Portanto, por \ref{prop:borel_homeomorphism}, as seções horizontais definidas na questão são borelianos da reta.
    \end{proof}
\end{enumerate}




\begin{problem}
    \label{prob:l1:6}
\end{problem}

Essa questão é bem divertida, estende dupla contagem para medidas.

\begin{prop}
    (a) Os conjuntos $H_k$ são mensuráveis.
\end{prop}
\begin{proof}
    Como cada $E_i$ é mensurável, definimos as funções mensuráveis $(f_n)_{n \in \N}$ por:
    $$f_n(x) = \sum_{j = 1}^{n} \mathds{1}_{E_j}(x)$$
    Então $0 \leq f_1 \leq \dots \leq f_n \leq f_{n+1} \leq \dots \leq \infty$ é uma sequência crescente
    mensurável, e portanto: 
    $$F(x) = \sup_n f_n(x) = \lim_n f_n(x) = \#\{n \, : \, x \in E_n \}$$
    é uma função mensurável. Temos que $H_k = F^{-1}([k,\infty])$ é um conjunto mensurável.
\end{proof}

Agora vem a parte difícil. Para mostrar a letra (b), esqueçamos $(E_n)_{n \in \N}$ por enquanto, foquemos
em $(E_n)_{n \in [N]}$ finitos.

\begin{definition}
    Dada uma sequência finita $(E_n)_{n \in [N]}$ de conjuntos de $\mathcal{M}$. Sejam $H_k^{(N)}$ da seguinte forma:
    $$H_k^{(N)} = \{x \in X \, : \, \#\{n \, : \, x \in E_n\} \geq k\}$$
    A mesma definição dos $H_k$, mas para uma coleção finita de no máximo $N$ conjuntos.
\end{definition}

\begin{observation}
    Temos propriedades simples, que independem de $N$ e da coleção escolhida:
    \begin{enumerate}
        \item Exatamente como na letra (a), $H_k^{(N)}$ é mensurável.
        \item $H_0^{(N)} = X$
        \item $H_{k+1}^{(N)} \subseteq H_{k}^{(N)}$
        \item $H_{N+1}^{(N)} = \varnothing$, pois nenhum elemento pertence em mais que $N$ conjuntos.
    \end{enumerate}
\end{observation}

Para qualquer sequência infinita $(E_n)_{n\in\N}$ definimos os $H_k^{(N)}$ para os primeiros $N$ conjuntos da sequência.

\begin{lemma}
    \label{lemm:hk}
    Seja $(E_n)_{n \in \N}$ mensuráveis. Para todo $N \in \N$, vale:
    $$\sum_{k=1}^{N} \mu(H_k^{(N)}) = \sum_{k=1}^{N} \mu(E_k)$$
\end{lemma}
\begin{proof}
    Vamos seguir por indução. Para $N = 1$, temos de graça que $E_1 = H_1^{(1)}$, logo $\mu(H_1^{(1)}) = \mu(E_1)$.
    Suponha que o resultado vale para $N$ e olhemos para o caso $N+1$.
    \begin{align*}
        \sum_{n=1}^{N+1} \mu(E_n) &= \mu(E_{N+1}) + \sum_{n=1}^{N} \mu(E_n)\\
        &= \mu(E_{N+1}) + \sum_{n=1}^{N} \mu(H_n^{(N)})
    \end{align*}
    Onde usamos a hipótese de indução na segunda igualdade.

    Note que $H_k^{(N+1)} = (H_k^{(N)} - E_{N+1}) \cup (H_{k-1}^{(N)} \cap E_{N+1})$. Pois se $x \in X$ aparece em $k$ conjuntos de $(E_n)_{n \in [N+1]}$,
    ou ele aparece em $k$ dos primeiros $N$ conjuntos, ou aparece em $E_{N+1}$ e pelo menos $k-1$ outros dos primeiros $N$.
    Para aproveitar dessa observação, podemos reescrever o somatório

    \begin{align*}
        \mu(E_{N+1}) + \sum_{n=1}^{N} \mu(H_n^{(N)}) &= \mu(E_{N+1}) + \sum_{n=1}^{N} \mu(H_n^{(N)} - E_{N+1}) + \mu(H_n^{(N)} \cap E_{N+1})\\
        &= \mu(E_{N+1}) + \sum_{n=1}^{N+1} \mu(H_n^{(N)} - E_{N+1}) + \mu(H_n^{(N)} \cap E_{N+1})\\
    \end{align*}
    Já que $H_{N+1}^{(N)} = \varnothing$. Agora escrevemos $\mu(E_{N+1}) = \mu(H_0^{(N)} \cap E_{N+1}) = \mu(X \cap E_{N+1})$ e reindexamos 
    cada termo da direita no somatório, obtendo
    \begin{align*}
        \sum_{n=1}^{N+1} \mu(E_n) &= \sum_{n=1}^{N+1} \mu(H_n^{(N)} - E_{N+1}) + \mu(H_{n-1}^{(N)} \cap E_{N+1})\\
        &= \sum_{k=1}^{N+1} \mu(H_k^{(N+1)})
    \end{align*}
    Provando o passo indutivo.
\end{proof}

Estamos quase finalizados, sentimos até vontade de passar o limite em \ref{lemm:hk} e obter o resultado, mas isso por si só não é suficiente.
\begin{prop}
    $\sum_{k=1}^{\infty} \mu(H_k) = \sum_{k=1}^{\infty} \mu(E_k)$
\end{prop}
\begin{proof}
    Tomando limites em $N$ no Lema \ref{lemm:hk}, temos que
    $$\lim_{N \to \infty} \sum_{k=1}^{N} \mu(H_k^{(N)}) = \sum_{k=1}^{\infty} \mu(E_k)$$
    Para obter o resultado, vamos mostrar que
    $$\lim_{N \to \infty} \sum_{k=1}^{N} \mu(H_k^{(N)}) = \sum_{k=1}^{\infty} \mu(H_k)$$
    Note que, pela definição dos $H_k^{(N)}$, temos uma sequência crescente $H_k^{(1)} \subseteq H_k^{(2)} \subseteq \dots \subseteq H_k$, tal que
    $$\bigcup_{n=1}^{\infty} H_k^{(n)} = H_k$$
    Por conta das inclusões $H_N^{(N)} \subseteq H_N$ e $\mu$ ser uma medida positiva, temos, termo a termo,
    $\mu(H_N^{(N)}) \leq \mu(H_N)$. Portanto, já temos um lado da igualdade.
    $$\lim_{N \to \infty} \sum_{k=1}^{N} \mu(H_k^{(N)}) \leq \sum_{k=1}^{\infty} \mu(H_k)$$
    Para o outro lado, observamos que como $H_N^{(N)} \to H_N$ são mensuráveis, $\lim_{n\to\infty} \mu(H_N^{(N)}) = \mu(H_N)$.
    Portanto, para cada $M > 0$, 
    $$\lim_{N \to \infty} \sum_{k=1}^{N} \mu(H_k^{(N)}) \geq \lim_{N \to \infty} \sum_{k=1}^{M} \mu(H_k^{N}) = \sum_{k=1}^{M} \mu(H_k)$$
    Como isso vale para todo $M$, $\lim_{N \to \infty} \sum_{k=1}^{N} \mu(H_k^{(N)}) \geq \sum_{k=1}^{\infty} \mu(H_k)$.
\end{proof}

%LISTA 2
\section{Lista 2 (21/08/2025)}

Listagem de problemas:
\begin{enumerate}
    \item Exercício \ref{prob:l2:1} : \checkmark
    \item Exercício \ref{prob:l2:2} : \checkmark
    \item Exercício \ref{prob:l2:3} : \checkmark
    \item Exercício \ref{prob:l2:4} : \checkmark
    \item Exercício \ref{prob:l2:5} : \checkmark
    \item Exercício \ref{prob:l2:6} : \checkmark
    \item Exercício \ref{prob:l2:7} : \checkmark
\end{enumerate}

Para a solução de vários problemas dessa lista, utilizaremos os três principais teoremas vistos em aula até agora. Vamos enunciá-los.

\begin{theorem}
    \label{trm:conv_mon}
    (Convergência Monótona). Dada uma sequência crescente de funções mensuráveis $(f_n)_n$ de $X$ para $[0,\infty]$, satisfazendo:
    \begin{enumerate}[label=(\alph*)]
        \item $0 \leq f_1(x) \leq f_2(x) \leq \dots \leq \infty$ para todo $x \in X$
        \item $f_n(x) \to f(x)$ para todo $x \in X$
    \end{enumerate}
    Então $f$ é mensurável, e
    $$\int_{X} f_n\,d\mu \to \int_{X} f\,d\mu$$
\end{theorem}

\begin{theorem}
    \label{trm:lemma_fatou}
    (Lema de Fatou). Se $f_n : X \to [0,\infty]$ é mensurável, para cada $n$, então
    $$\int_X \big(\liminf_{n\to\infty} f_n\big)\,d\mu \leq \liminf_{n\to\infty} \int_X f_n \,d\mu$$
\end{theorem}

\begin{theorem}
    \label{trm:conv_dom}
    (Convergência Dominada). Se $\{f_n\}$ é uma sequência de funções mensuráveis complexas de $X$ tal que
    $$f(x) = \lim_{n\to\infty} f_n(x)$$
    existe para todo $x \in X$. Se existe uma função $g \in L^1(\mu)$ tal que, para todo $n$,
    $$|f_n(x)| \leq |g(x)|$$
    então $f \in L^1(\mu)$,
    $$\lim_{n\to\infty} \int_X |f_n - f|\,d\mu = 0$$
    e
    $$\lim_{n\to\infty} \int_X f_n \,d\mu = \int_X f \,d\mu$$
    
\end{theorem}

\begin{problem}
    \label{prob:l2:1}
\end{problem}

\begin{proof}
    
    Essa questão parece muito com a de interseção de conjuntos mensuráveis (Teorema 1.19
    Rudin). Se $f_1 \in L^1(\mu)$, como ela é positiva, existe $0 \leq M < \infty$ tal que $\int_X f_1 \, d\mu \leq M$. Defina $g_n$ mensurável por $g_n = f_1 - f_n$. Temos então que
    \begin{enumerate}[label=(\alph*)]
        \item $0 \leq g_1 \leq g_2 \leq \dots \leq \infty$
        \item $g_n(x) \to f_1(x) - f(x)$ para todo $x \in X$.
    \end{enumerate}
    Podemos aplicar convergência monótona [\ref{trm:conv_mon}] para encontrar:
    \begin{align}
    \lim_{n\to\infty} \int_X f_1 - f_n \, d\mu &= \int_X f_1 - f \, d\mu \\
    \lim_{n\to\infty} \bigg(\int_X f_1\,d\mu  - \int_X f_n \, d\mu\bigg) &= \int_X f_1\,d\mu  - \int_Xf \, d\mu \\
    \int_X f_1\,d\mu - \lim_{n\to\infty} \int_X f_n \, d\mu&= \int_X f_1\,d\mu  - \int_Xf \, d\mu \\
    \lim_{n\to\infty} \int_X f_n \, d\mu & = \int_Xf \, d\mu
    \end{align}
    Onde, crucialmente, usamos na segunda* igualdade que $\int_X f_1 \leq M < \infty$.

    Se admitimos $f_1 \not \in L^1(\mu)$, a igualdade pode não valer. Defina $f_n(x) = 1/n$ para todo $x \in \R$.
    Temos que $f_n \to f = 0$, logo $\int_\R f\,d\mu = 0$, mas $\int_\R f_n\,d\mu = \infty$ para todo $n$.
\end{proof}

\begin{problem}
    \label{prob:l2:2}
\end{problem}
\begin{proof}
    
    Podemos usar diretamente o exemplo patológico da questão \ref{prob:l2:1}. Mas afim de fazer um diferente, seja $X = \{0,1\}$ com medida de contáveis. Defina as funções simples (e portanto mensuráveis) $h$ e $g$ dadas por
    \begin{equation*}
    h(x)=
    \begin{cases}
        0 & \text{if } x =0\\
        1 & \text{if } x =1
    \end{cases}
\end{equation*}
e
\begin{equation*}
    g(x)=
    \begin{cases}
        1 & \text{if } x =0\\
        0 & \text{if } x =1
    \end{cases}
\end{equation*}
Seja $\{f_n\}$, tal que $f_n = h$ se $n$ for par, e $f_n = g$ se $n$ for ímpar. Então claramente,
$\liminf_n f_n(x) = 0$ para todo $x$, mas $\int_X f_n\, d\mu = 1$ para todo $n$. Portanto
$$0 = \int_X (\liminf_n f_n)d\mu < \liminf_n \int_X f_nd\mu = 1$$
\end{proof}

\begin{problem}
    \label{prob:l2:3}
\end{problem}
\begin{proof}
    

    Esse problema é bem legal, envolve aproximar a função pontualmente e perceber que podemos aplicar nossos resultados. Antes de mais nada, dado $\alpha > 0$, defina $g_n : X \to [0,\infty]$, por 
    $$g_n(x) =n\log(1 + (f(x)/n)^\alpha) = n [\log(n^\alpha + f(x)^\alpha) - \log(n^\alpha)]$$
    $g_n$ é composição de uma função contínua por uma mensurável $f \geq 0$, é portanto mensurável e
    da forma que está definida, é positiva. $g(x) \in [0,\infty]$.

    Agora, vamos tentar estimar $g_n$. Pelo teorema do valor médio, dado $x$ fixo,
    $$\log(n^\alpha + f(x)^\alpha) - \log(n^\alpha) = \frac{f(x)^\alpha}{y}$$
    para $y \in (n^\alpha, n^\alpha + f(x)^\alpha)$.
    Então, temos
    $$n\frac{f(x)^\alpha}{n^\alpha + f(x)^\alpha} \leq g_n(x) \leq n\frac{f(x)^\alpha}{n^\alpha}$$
    E isso já é suficiente para dois casos do problema.
  
    Uma observação antes de brincar com as integrais é que como $\int_X f d\mu \in (0,\infty)$, notamos duas coisas. 
    \begin{enumerate}
        \item  O conjunto onde $f$ vale infinito tem medida nula.
        \item O conjunto onde $f > 0$ tem medida positiva.
    \end{enumerate}
    O item (1) vale pois, caso contrário, $\int_X fd\mu$ valeria infinito. Da mesma forma, (2) vale pois, caso contrário,
    como $f$ é positiva, a integral valeria $0$. Disso segue que podemos supor, sem perda de generalidade, que $f$ é estritamente 
    positiva e não assume valores infinitos em $X$ (onde $f$ vale $0$ não muda as integrais definidas). Agora, feita essa observação, 
    podemos dar continuidade ao resultado.

    
    Se $\alpha = 1$,
    $$\frac{nf(x)}{n + f(x)} \leq g_n(x) \leq f(x)$$
    Como o lado esquerdo tende a $f(x)$, temos que $g_n(x) \to f(x)$. Além do mais, $g_n(x) \leq f(x) \in L^1(\mu)$, logo, por Convergência Dominada [\ref{trm:conv_dom}], temos que
    
    $$\lim_{n\to\infty}\int_X n\log(1 + (f(x)/n)^\alpha)d\mu = \lim_{n\to\infty}\int_X g_n(x)d\mu = \int_X fd\mu = c$$
    
    Se $\alpha < 1$, de $g_n(x) \geq nf(x)^\alpha / (n^\alpha + f(x)^\alpha) \to \infty$
    temos que
    $$ \infty = \liminf_{n\to\infty} nf(x)^\alpha / (n^\alpha + f(x)^\alpha) \leq \liminf_{n\to\infty} g_n(x)$$
    Usando o lema de Fatou [\ref{trm:lemma_fatou}], 
    $$\infty = \int_X \infty d\mu \leq \liminf_{n\to\infty}\int_X g_n d\mu = \liminf_{n \to \infty} \int_X n\log(1 + (f(x)/n)^\alpha) d\mu$$
    que é o resultado esperado.
    
    Para $\alpha > 1$, terei que usar a dica do João, percebi que só conseguiria usar convergência dominada se $\int_X f^\alpha d\mu < \infty$, (mas não sabemos disso). 
    Então precisamos fazer surgir $f$ sem expoentes na estimativa de $g_n$, para isso consideramos a sequências de desigualdades, válida para $t \geq 0$, $\alpha > 1$.
    $$
        1 + t^\alpha \leq (1 + x)^\alpha \leq (e^x)^\alpha = e^{\alpha x}\\
    $$
    Onde a primeira desigualdade sai, como observado pelo João, imediatamante de 
    $$\bigg(\frac{1}{1 + t}\bigg)^\alpha + \bigg(\frac{t}{1 + t}\bigg)^\alpha \leq 1$$
    Tomando $\log$'s na equação,
    $$
        \log(1 + t^\alpha) \leq \log((1 + t)^\alpha) \leq \log((e^t)^\alpha) = \alpha t
    $$
    Portanto, $g_n(x) \leq n \alpha (f(x)/n) = \alpha f(x) \in L^1(\mu)$. Agora estamos muito felizes, pois sabemos que pontualmente (para cada $x$ fixo).
    
    $$g_n(x) \leq \frac{f(x)^\alpha}{n^{\alpha -1}} \to 0$$
    Logo, por convergência dominada [\ref{trm:conv_dom}],
    $$\lim_{n \to \infty} \int_X n\log(1 + (f(x)/n)^\alpha) d\mu= \lim_{n\to\infty} \int_X g_nd\mu = \int_X 0d\mu = 0$$
\end{proof}


\begin{problem}
    \label{prob:l2:4}
\end{problem}

\begin{proof}
    Essa questão segue quase imediatamante da série de desigualdades
    \begin{align}
        \lim_{n\to\infty} \bigg|\int_X f_n d\mu - \int_Xfd\mu \bigg| &\leq \lim_{n\to\infty} \int_X |f_n - f| d\mu\\
        &\leq \lim_{n\to\infty} \int_X \sup_x\{|f_n - f|\}d\mu\\
        &= \lim_{n\to\infty} \sup_x\{|f_n - f|\} \mu(X) \to 0
    \end{align}
    Onde em (7) usamos crucialmente que $\mu(X) < \infty$ e a sequência é uniformemente convergente.

    Se $\mu(X) = \infty$, segue exatamente da solução do exercício \ref{prob:l2:1}, com $f_n = 1/n$, $f = 0$, $X = \R$, que a hipótese não pode ser omitida.
\end{proof}

\begin{problem}
    \label{prob:l2:5}
\end{problem}

\begin{proof}
    Minha intuição Riemanianna me matou nessa questão, tenho que abandoná-la. Suponha que o resultado seja falso, i.e. $f \in L^1(\mu)$, mas existe $\eps > 0$, tal que 
    para todo $\delta > 0$ existe um mensurável $E_\delta$ com $\mu(E_\delta) < \delta$, mas 
    $$\int_{E_\delta} |f| d\mu > \eps$$
    Então, escolhemos uma sequência de $(E_n)_n$, com $\mu(E_n) < 2^{-n}$ e $\int_{E_n} |f| d\mu > \eps$. Note que a união dos $E_n$ tem medida finita.
    $$A = \bigcup_n E_n \quad \Rightarrow \quad \mu(A) = \mu \big(\bigcup_n E_n \big) \leq \sum_n \mu(E_n) \leq 2$$
    E se definirmos $A_m = \bigcup_{n\geq m} E_n$, achamos uma sequência decrescente de conjuntos de medida finita: $A = A_1 \supset A_2 \supset \dots$.
    Além do mais,
    $$\mu(A_m) \leq \sum_{n \geq m} \mu(E_n) \leq \sum_{n \geq m} 2^{-n} = 2^{-m + 1} \to 0$$
    Quando $m \to \infty$. Portanto, $\mu(\bigcap_m A_m) \to 0$.

    A ideia da prova agora é mostrar que a integral sobre esse conjunto é um limite sobre integrais todas maiores que $\eps$, mas então teríamos
    que a integral sobre um conjunto de medida nula maior que $0$, absurdo. Para isso, defina, para cada $m$
    $$f_m(x) = |f(x)| \mathds{1}_{A_m}(x)$$
    funções mensuráveis, decrescentes e todas dominadas por $|f| \in L^1(\mu)$. Chamando as interseções dos $A_m$ de $B$,
    temos que $f_m \to |f| \cdot \mathds{1}_B$. Pelo teorema da convergência dominada [\ref{trm:conv_dom}], temos que 
    $$\lim_{m\to\infty} \int_X |f|\cdot \mathds{1}_{A_m} d\mu = \int_X |f| \cdot \mathds{1}_B d\mu = \int_B |f| d\mu$$
    Mas, por hipótese, $\int_X |f|\cdot \mathds{1}_{A_m} d\mu \geq \int_{E_m} |f| d\mu > \eps$. Logo o limite da esquerda deve ser maior ou igual a $\eps > 0$,
    mas a integral da direita - sobre um conjunto $B$ de medida nula - deveria ser $0$.
\end{proof}

\begin{problem}
    \label{prob:l2:6}
\end{problem}
\begin{proof}
    A resolução dessa questão está no livro, cuja prova repetirei aqui. Note no entanto que ela segue diretamente da questão \ref{prob:l1:6} da lista anterior, pois provamos 
    que 
    $$\sum_{k=1}^{\infty} \mu(E_k) = \sum_{k=1}^{\infty} \mu(H_k)$$
    como $H_1 \supset H_2 \supset \dots \supset H_\infty$, se $\mu(H_\infty) > \eps > 0$, então para todo $k$, $\mu(H_k) > \eps$. Teríamos por fim que $\sum_{k=1}^{\infty} \mu(H_k) = \infty$.

    Eu acho a solução do Rudin mais elegante, pois - ao menos para mim - foi trabalhoso estabelecer a igualdade entre os somatórios. Assim como antes, construa 
    $$f_N = \sum_{k=1}^{N} \mathds{1}_{E_k}$$
    Temos que $f_N$ é uma sequência crescente de funções que tende a $\sum_{k = 1}^{\infty} \mathds{1}_{E_k}$. Pelo teorema da convergência monótona [\ref{trm:conv_mon}]
    \begin{equation}
        \lim_{n\to\infty} \int_X f_n d\mu = \int_X f d\mu
    \end{equation}
    O termo da esquerda é precisamente $\sum_{k=1}^{\infty} \mu(E_k)$ que estamos supondo ser $ < \infty$. Agora, se a medida do conjunto $\{x : f(x) = \infty\}$ (os $x$'s que aparecem em infinitos $E_k$'s) 
    fosse positiva então estaríamos integrando infinito sobre um conjunto de medida não nula, e a integral da direita seria infinito. Simbólicamente:
    $$\int_X f d\mu \geq \int_{f^{-1}(\infty)} fd\mu = \mu(f^{-1}(\infty)) \cdot \infty = \infty$$
    O que contradiz (8).
\end{proof}

\begin{problem}
    \label{prob:l2:7}
\end{problem}
\begin{proof}
    Eu não sei exatamente quanto queremos mostrar nessa questão, provamos em aula que para $f,g \in L^1(\mu)$ e $\alpha,\beta \in \mathbb{C}$, 
    a função $\alpha f + \beta g$ é mensurável (onde está bem definida). Seja $A = f^{-1}(\infty) \cap g^{-1}(\infty)$, $A$ é interseção de mensuráveis
    e portanto mensurável, definimos exatamente como no problema \ref{prob:l1:2} a função mensurável 
    $$h = \mathds{1}_A + \mathds{1}_{A^c}f - \mathds{1}_{A^c}g$$
    $h(z)$ é $0$ se e somente se $f(z)$ e $g(z)$ não são infinitas e $f(z) = g(z)$. Usando $h$, o conjunto 
    $$\{z : f(z) = g(z)\} = h^{-1}(0) \cup A$$
    é mensurável.
\end{proof}

%LISTA 3
\section{Lista 3 (28/08/2025)}

Listagem de problemas:
\begin{enumerate}
    \item Exercício \ref{prob:l3:1} : \Frowny
    \item Exercício \ref{prob:l3:2} : \Frowny
    \item Exercício \ref{prob:l3:3} : \checkmark
    \item Exercício \ref{prob:l3:4} : \checkmark
    \item Exercício \ref{prob:l3:5} : \checkmark
    \item Exercício \ref{prob:l3:6} : \checkmark
\end{enumerate}

\begin{problem}
    \label{prob:l3:1}
\end{problem}
\begin{proof}
    A resposta dessa pergunta é positiva, mas eu penei um pouco para chegar nessa conclusão. Lembremos 
    que para mostrar que $f$ é Borel mensurável, basta mostrar que, para todo $c \in \R$, a pré-imagem 
    $f^{-1}((c, +\infty))$ é mensurável. Vamos mostrar que essa pré-imagem é uma união enumerável de conjuntos
    Borel mensuráveis em $\R$.

    Seja $A = f^{-1}((c, +\infty))$ e tome $a \in A$, i.e $f(a) > c$, pelas condições de continuidade em $f$, temos três casos possíveis:
    \begin{enumerate}
        \item $f$ é contínua em $a$, logo $\exists \delta_a > 0$ tal que $(a - \delta_a, a + \delta_a) \subset A$
        \item $f$ é contínua a esquerda em $a$, logo $\exists \delta_a > 0$ tal que $(a - \delta_a, a] \subset A$
        \item $f$ é contínua a direita em $a$, logo $\exists \delta_a > 0$ tal que $[a, a + \delta_a) \subset A$
    \end{enumerate}
    Vamos mostrar que $A$ é a união enumerável de seus componentes conexos, como os componentes conexos são intervalos
    da reta, eles são borelianos e portanto $A$ será boreliano. Para isso, basta notar que em cada componente conexo 
    há um racional que determina ele completamente - tome $x$ de um componente, olhe para 
    um racional no intervalinho associado a $x$. Como os racionais são enumeráveis, esses componentes são enumeráveis.
\end{proof}

\begin{problem}
    \label{prob:l3:2}
\end{problem}
\begin{proof}
    \textbf{RESOLVER PARTE POSITIVA E NEGATIVA, ESQUECI DISSO}

    Basta lembrar bem da definição da integral de Riemann para perceber que a de Lebesgue generaliza ela.
    Por Riemann, toda função contínua num compacto é integrável e suas somas inferiores e superiores convergem. Temos
    $$\int_a^b f(x) dx = \lim_{|P|\to 0} L(P,f)$$
    onde $P$ é um pontilhamento do compacto $[a,b]$, $L(f,P)$ é a soma inferior de $f$ por $P$ e $|P|$ é 
    o tamanho do maior intervalo do pontilhamento. Podemos expressar
    $L(P,f)$ como uma soma, se $P$ é $(a = t_0, \dots t_n = b)$, temos 
    $$L(P,f) = \sum_{i = 1}^{n} m_i (t_i - t_{i-1}) = \sum_{i = 1}^{n} m_i \mu([t_{i-1},t_i))$$
    onde $m_i = \inf\{f(x); x \in [t_{i-1}, t_{i})\}$.

    Olhando para essa fórmula é claro perceber que cada pontilhamento $P$ está associado com uma 
    função simples menor ou igual a $f$. A ideia da prova é escolher uma sequência de pontilhamento $(P_n)_n$ (diádicos por exemplo)
    cujo módulo $|P_n|$  tende a 0 e cada pontilhamento é um 
    refinamento do anterior. Dessa forma, eles definirão uma sequência crescentes de funções que convergem para $f$, 
    aplicando o Teorema da Convergência Monotona [\ref{trm:conv_mon}], teremos o resultado.

    Seja $P_0 = \{a,b\}$, definiremos indutivamente uma sequência de refinamentos (os diádicos).
    Dado $P_n = \{t_0 < t_1 < \dots < t_m\}$, cortamos cada intervalo no meio, i.e.
    $$P_{n+1} = P_n \cup \bigg\{\frac{t_i + t_{i+1}}{2} : 0 \leq i < m\bigg\}$$
    Claramente, $|P_n| = (b-a)/2^n \to 0$ e, portanto, pelos teoremas da integral de Riemann,
    \begin{equation}
        \lim_{n \to \infty} L(P_n,f) \to \int_a^b fdx
    \end{equation}
    Agora definimos uma função step $s_n$ associada ao pontilhamento $P_n = \{t_0 < t_1 < \dots < t_m\}$,
    $$s_n(x) = \begin{cases}
        \inf\{f(a): a \in [t_i, t_{i+1})\} & \text{se } x \in [t_i, t_{i+1}]\; \text{para } i < m\\
        \inf\{f(a): a \in [t_{m-1}, t_{m}]\} & \text{se } x \in [t_{m-1}, t_{m}]\\
    \end{cases}$$
    Separar o último caso não é necessário, coloquei somente por clareza. Da forma que estão definidos,
    (1) os $s_n$ são funções simples. Como os $P_n$ são refinamentos, (2) $s_{n} \leq s_{n+1}$ e, além do mais,
    sendo $f$ uniformemente contínua em $[a,b]$, temos que (3) $s_n \to f$ uniformemente. Por [\ref{trm:conv_mon}],
    \begin{equation}
        \lim_{n\to \infty} \int_{[a,b]} s_n d\mu= \int_{[a,b]} f d\mu
    \end{equation}
    Mas por serem funções simples,
    \begin{equation}
        \int_{[a,b]} s_n d\mu = L(P_n, f)
    \end{equation}
    Juntando as equações 9, 10 e 11, obtemos o resultado.
    $$\int_a^b fdx = \lim_{n \to \infty} L(P_n, f) = \lim_{n\to\infty }\int_{[a,b]} s_n d\mu = \int_{[a,b]} fd\mu$$
\end{proof}

\begin{problem}
    \label{prob:l3:3}
\end{problem}
\begin{proof}
    Esse exercício é similar ao \ref{prob:l2:5} da lista anterior, pelo menos a resolução do João. Sejam 
    \begin{align*}
        A_n &= \{x : |f_{n+1}(x) - f_n(x)| \geq \eps_n\}\\
        B_n &= \bigcup_{m \geq n} A_m
    \end{align*}
    Note que, sendo $\mu^*$ uma medida exterior,
    $$\mu^*(B_N) \leq \sum_{m = N}^{\infty} \mu^*(A_m) \to 0$$
    quando $N \to \infty$. Como os $B_n$ são encaixados, isso é o mesmo que dizer $\mu^*(\bigcap_n B_n) = 0$.
    Agora seja $x \not \in \bigcap_n^\infty B_n$, portanto existe $n_0$ tal que 
    $$ x \not \in B_{n_0}, B_{n_0 + 1}, \dots $$
    e, da mesma forma, 
    $$ x \not \in A_{n_0}, A_{n_0 + 1}, \dots$$
    Isso significa que para todo $m > n_0$, $|f_{m+1}(x) - f_{m}(x)| \leq \eps_m$. Como
    $\sum_n \eps_n < \infty$, pelo $M$-teste de Weierstrass, $f_n(x)$ converge. Ou seja, mostramos 
    que $(f_n(x))$ converge em quase todo ponto.
\end{proof}

\begin{problem}
    \label{prob:l3:4}
\end{problem}
Achei que esse problema era um pouco mais fácil do que realmente é. A ideia principal será 
aproximar $f$ por funções borelianas por baixo cujo limite das integrais é a integral de $f$.
Para formalizar isso, precisamos de um lema.

\textbf{Nota posterior:} Só percebi agora que $f$ é supostamente complexa, mas como vimos anteriormente
isso não faz a menor diferença, escreva $f = u^{+} - u^{-} + iv^{+} - iv^{-}$ e aplique o resultado
para cada parte. Vamos supor de agora em diante que $f : \R \to \R^+$.

\begin{lemma}
    \label{lemm:sn_ae}
    Seja $s_n$ uma sequência funções simples, todas menores igual a uma $f: X \to \R^+$ mensurável,
    as quais vale que
    $$\lim_{n \to \infty} \int_X s_n d\mu = \int_X f d\mu$$
    (Tal sequência sempre pode ser obtida pela definição da integral de $f$ e pela aproximação feita em aula). 
    Se definimos $S = \sup_n s_n$ mensurável, então $\mu(\{x : S(x) \neq f(x)\}) = 0$.
\end{lemma}
\begin{proof}
    Sabemos que o conjunto $\{x : S(x) \neq f(x)\}$ é mensurável, então a conclusão do lema faz sentido.
    Além disso, por definição, $S(x) \leq f(x)$. Defina $A_n = \{x : S(x) < f(x) - 1/n\}$, note que
    $$\bigcup_n A_n = \{x : S(x) \neq f(x)\}$$
    Portanto, se para todo $m$, $\mu(A_m) = 0$, valerá que $\mu(\{x : S(x) \neq f(x)\}) = 0$. Agora isso é 
    quase que imediato, já que, se $\mu(A_m) > 0$, então, para todo $n$
    $$\int_{X} f - s_n d\mu > \frac{\mu(A_m)}{m} > 0$$
    Mas isso é absurdo, pois $$\int_X f - s_n d\mu \to 0$$
\end{proof}
Essa é a ferramenta principal. Agora, para a questão, basta procurar uma sequência da mesma forma composta por funções borelianas 
- essa parte é mais difícil. Para isso, vou invocar o Teorema 2.17 do livro do Rudin (adaptado a $\R$).
\begin{theorem}
    Sejam $\mu$ e $M$ a medida e a $\sigma$-álgebra de Lebesgue da reta, elas satisfazem as seguintes propriedades:
    \begin{enumerate}[label=(\alph*)]
        \item Se $E \in M$ e $\eps > 0$, existe um fechado $F$ e um aberto $V$ tal que $F \subset E \subset V$ e $\mu(V - F) < \eps$. 
        \item $\mu$ é uma medida regular de Borel
        \item Se $E \in M$, existem $A$ e $B$ tal que $A$ é $F_\sigma$, $B$ é $G_\delta$, $A \subset E \subset B$, e $\mu(B-A) = 0$.
    \end{enumerate}
\end{theorem}
Agora estamos prontos para resolver o problema.
\begin{proof}
    \textbf{Prova do exercício.} Seja $(s_n)$ uma sequência de funções Lebesgue simples que aproximam
    a integral de $f$ por baixo. Cada $s_n$ pode ser escrita da forma
    $$s_n = \sum_{m=1}^{k} a_m \mathds{1}_{E_m}$$
    Onde os $E_m$ são conjuntos de Lebesgue, pelo Teorema, podemos aproximá-los por borelianos $A_m$ de mesma medida,
    de forma que a função simples boreliana 
    $$\tilde{s}_n = \sum_{m=1}^{k} a_m \mathds{1}_{A_m}$$
    é igual a $s_n$ a.e. Como
    $$\int_X \tilde{s}_n d\mu = \int_X s_n d\mu$$
    então temos que
    $$\lim_{n\to\infty} \int_X \tilde{s}_n d\mu = \int_X fd\mu$$
    Claramente, $\tilde{s}_n \leq s_n \leq f$, logo pelo lema \ref{lemm:sn_ae} $S = \sup \tilde{s}_n$ é idêntica
    a $f$ a.e. 
\end{proof}

\begin{problem}
    \label{prob:l3:5}
\end{problem}
Essa questão é a mais simples e vou tentar transcrever o desenho que soluciona ela em palavras.
A ideia aqui é que escrevamos uma sequência de funções trapezoidais em $[0,1]$ que vão ficando cada vez mais fininhas,
de forma que integrar sobre elas tenda a $0$, mas que cada ponto de $[0,1]$ chegue a valer $0$ e $1$ infinitas vezes.
\begin{proof}
    Vamos construir uma sequência de funções contínuas espertas $(F_n)_n$ em $[0,1]$.
    % $$F_{2^k}(x) = \begin{cases}
    %     1 &\text{se } x \leq 2^{-k}\\
    %     1 - 2^k(x - 2^{-k}) & \text{se } 2^{-k} \leq x \leq 2^{-k + 1}\\
    %     0 & \text{se }  x \geq 2^{-k + 1}
    % \end{cases}$$
    % Semelhantemente, definimos (para a última função do bloco)
    % $$F_{2^{k+1} - 1}(x) = \begin{cases}
    %     1 &\text{se } x \geq  1 - 2^{-k}\\
    %     2^k(x - (1 - 2^{-k})) & \text{se } 1 - 2^{-k} \geq x \geq 1 - 2^{-k + 1}\\
    %     0 & \text{se }  x \leq 1 - 2^{-k + 1}
    % \end{cases}$$
    % Ou mais econômicamente, $F_{2^{k+1} - 1}(x) = F_{2^k}(1 - x)$.
    Para $m \in \{2^k, 2^k + 1, \dots, 2^{k+1} - 1\}$, seja $n = m - 2^k$, definimos
    $$F_{m}(x) = \begin{cases}
        0 & \text{se }  x \leq (n - 1)2^{-k}\\
        2^k(x - (n-1)2^{-k}) & \text{se } (n-1)2^{-k} \leq x \leq n2^{-k}\\
        1 &\text{se } n2^{-k} \leq x \leq (n+1)2^{-k}\\
        1 - 2^k(x - (n+1)2^{-k}) & \text{se } (n+1)2^{-k} \leq x \leq (n+2)2^{-k}\\
        0 & \text{se }  x \geq (n+2)2^{-k}
    \end{cases}$$
    Onde óbiviamente $F_m$ está definida definida dessa forma quando os casos fazem sentido.
    Por exemplo quando $m = 2^k$, $n=0$ o primeiro e o segundo caso não aparecem. Quando 
    $m = 2^{k+1} - 1$, $n = 2^{k} - 1$, o quarto e o último não aparecem. Essas funções são claramente contínuas, 
    são trapézios que vão ficando cada vez menos espessos. Se $2^k \leq m < 2^{k+1}$, uma soma simples 
    sobre funções afins mostra que 
    $$\int_0^1 F_m dx \leq 2^{-k + 1} \to 0$$
    No entanto, é também fácil perceber que se $k > 2$, para qualquer $x \in [0,1]$, existem $2^k \leq N,M < 2^{k+1}$
    tais que $F_N(x) = 0$ e $F_M(x) = 1$. Portanto $F_m(x)$ não converge para nenhum ponto, mesmo que as integrais convirjam. 
\end{proof}

\begin{problem}
    \label{prob:l3:6}
\end{problem}
Esse é o problema mais legal, não acredito que consiguiria fazê-lo sem uma dica da professora Cynthia.
A única função não mensurável que conhecemos até agora é a característica de um conjunto não mensurável,
a ideia é tentar formar essa característica somente no liminf. Vamos fazer isso removendo pontualmente 
o complementar de um conjunto não mensurável infinitas vezes. \textbf{Obs:} A escolha esquista
de $(0,1]$ nos conjuntos a seguir é para facilitar a colagem que precisaremos fazer para construir a função $f$.
\begin{proof}
    Seja $T$ um conjunto não mensurável de $(0,1]$ e $T' = (0,1] - T$ seu complementar em $(0,1]$, note que $T'$ também é não mensurável.
    Vamos definir uma função $g(x,t) : \R\times(0,1] \to \{0,1\}$ que será a nossa ferramenta principal para construir  $f$.
    
    Dado $t$ fixo, se $t \in T'$, seja $A_t = (0,1] - \{t\}$, então definimos 
    $$g(x,t) \begin{cases}
        \mathds{1}_{A_t}(x) & \text{se } t \in T'\\
        \mathds{1}_{(0,1]}(x) & \text{se } t \not \in T'
    \end{cases}$$
    Note que, trivialmente, para todo $t \in (0,1]$, a função $g(t,x)$ -
    com a variável em $x$ - é mensurável, além do mais, sua integral sobre $x$ é claramente $1$. 
    
    Agora a  ideia é de alguma forma colar infinitas cópias de $g$ uma acima da outra.
    Separe $(0,1]$ na união disjunta:
    $$(0,1] = \bigcup_{n=1}^{\infty} (2^{-n}, 2^{-n + 1}]$$
    Definiremos $g_n(x,t) : \R \times (2^{-n},2^{-n + 1}] \to \{0,1\}$ da seguinte forma:
    $$g_n(x, t) = g(x, t2^n - 1)$$
    Por fim, defina $f :\R^2 \to \{0,1\}$ colando as $g_n$.
    $$f(x,t) = \begin{cases}
        \mathds{1}_{(0,1]}(x) & \text{se } t \leq 0 \ \lor \ t > 1\\
        g_n(x,t) & \text{se } t \in (2^{-n}, 2^{-n + 1}] 
    \end{cases}$$
    Afirmo que $f$ satisfaz as propriedades do exercício. Claramente a integral de 
    Lebesgue $\int_\R f(x,t) dx = 1$ para todo $t \in \R$, pois, fixando $t$, nossa 
    função é sempre a indicadora de $(0,1]$ salvo as vezes um único ponto. Para a segunda propriedade,
    vamos querer verificar que $h(x) = \liminf_{t \to 0} f(x,t)$ é justamente $\mathds{1}_T(x)$ que não é mensurável.
    Para isso note que se $x \in T \subset [0,1]$, então $g(x,t) = 1$ para todo $t$ e portanto, $g_n(x,t) = 1$ para qualquer $t$ também.
    Logo $f(x,t) = 1$ para todo $t$ e $h(x) = 1$. Se $x \in (0,1]^c$ então também, trivialmente $f(x,t) = 0$ para todo $t$
    e $h(x) = 0$. Agora, se $x \in T'$, para todo $n$, vale que 
    $$g_n\bigg(x, \frac{x + 1}{2^n}\bigg) = \mathds{1}_{A_x}(x) = 0$$
    e, para qualquer outro $t \in (2^{-n}, 2^{-n + 1}]$,
    $$g_n(x, t) = 1$$
    Em particular, sendo colagem desses valores, vale que para valores arbitrariamente pequenos de $t$ atingimos
    $f(x,t) = 0$ e portanto $h(x) = \liminf_{t \to 0} f(x,t) = 0$. Segue dos casos anteriores que $\liminf_{t \to 0} f(x,t)= \mathds{1}_T(x)$
    que não é mensurável.
\end{proof}



%LISTA 4
\clearpage
\section{Lista 4 (04/09/2025)}

Listagem de problemas:
\begin{enumerate}
    \item Exercício \ref{prob:l4:1} : \checkmark
    \item Exercício \ref{prob:l4:2} : \checkmark
    \item Exercício \ref{prob:l4:3} : \checkmark
    \item Exercício \ref{prob:l4:4} : \checkmark
    \item Exercício \ref{prob:l4:5} : \checkmark/\Frowny\,  (Se sortear pode corrigir)
    \begin{enumerate}[label=(\alph*)]
        \item \checkmark
        \item \checkmark
        \item \checkmark
        \item \Frowny /\checkmark (Acho que não está suficientemente detalhada - alguns passos omissos)
        \item \Frowny/\checkmark (Não acredito que seja uma prova de verdade)
    \end{enumerate}
\end{enumerate}

\begin{problem}
    \label{prob:l4:1}
\end{problem}
\begin{proof}
    Sabemos que para qualquer $E \subset \R$, $\lambda^*(E) \leq \lambda^*(E \cap A) + \lambda^*(E - A)$.
    Queremos mostrar então que a outra desigualdade vale, i.e. $\lambda^*(E) \geq \lambda^*(E \cap A) + \lambda^*(E - A)$.
    
    Se $\lambda^*(E) = \infty$, não há nada a fazer. Suponha que $\lambda^*(E) < \infty$ e dado $\eps > 0$
    encontre um aberto $V$ com $E \subset V$ tal que 
    $$\lambda^*(V) < \lambda^*(E) + \eps .$$
    Note crucialmente que a condição que $A$ satisfaz significa que $A \in M_F$
    e sendo $V$ aberto, então $V \in M_F$ também. Logo, $\lambda^*(V) = \lambda^*(V \cap A) + \lambda^*(V - A)$.
    Portanto, temos 
    $$\lambda^*(E) + \eps > \lambda^*(V) = \lambda^*(V \cap A) + \lambda^*(V - A) \geq \lambda^*(E \cap A) + \lambda^*(E - A).$$
    Fazendo $\eps \to 0$, terminamos a demonstração.
\end{proof}

\begin{problem}
    \label{prob:l4:2}
\end{problem}
\begin{proof}
    Vamos seguir a dica do Rudin para esse exercício. Seja $\{K_\alpha\}$ a coleção de todos 
    os subconjuntos compactos de $X$ com $\mu(K_\alpha) = 1$. Essa coleção não é vazia pois $X$ está nela. 
    Defina o compacto (interseção de compactos)
    $$K = \bigcap_\alpha K_\alpha.$$
    Vamos mostrar que $K$ satisfaz as propriedades exigidas. 
    
    Naturalmente, se houvesse subcompacto próprio  $H \subsetneq K$
    com $\mu(H) = 1$, então teriamos que $K \subset H \subset X$ e $K = H$, absurdo. Como $\mu(K) \leq \mu(X) = 1$, 
    falta só mostrar que $\mu(K) \geq 1$.
    % ,faremos isso provando que se $V \supset K$ é um aberto, então $\mu(V) \geq 1$.
    Seja $V$ aberto com $K \subset V$, então $X - V$ é um compacto e em particular
    $$X - V \subset X - K = X \cap \bigg(\bigcup_\alpha K_\alpha^c\bigg) \subset \bigcup_\alpha K_\alpha^c.$$
    Tomando uma subcobertura finita, notamos que
    $$X - V \subset X \cap \bigcup_{i=1}^{n} K_{\alpha_i}^c = \bigcup_{i=1}^{n} (X - K_{\alpha_i})$$
    Portanto, 
    $$\mu(X - V) \leq \sum_{i=1}^{n} \mu(X - K_{\alpha_i}) = 0$$
    e temos $1 = \mu(X \cap V) \leq \mu(V)$. Como isso vale para qualquer $V$ aberto que contém $K$, tomando
    ínfimos, temos que $\mu(K) \geq 1$.
\end{proof}

\begin{problem}
    \label{prob:l4:3}
\end{problem}
Eu tinha uma resposta super complicada para essa pergunta, por sorte o Bruno - aluno de Doutorado - 
viu a questão e respondeu de maneira muito mais simples.
\begin{proof}
    Vamos mostrar que esses são os fechos de abertos limitados da reta. Uma inclusão 
    é óbvia, pois para toda $f \in C_c(\R)$, $\text{supp } f = \overline{f^{-1}(\R - \{0\})}$
    é o fecho do aberto limitado $f^{-1}(\R - \{0\})$. Agora para todo aberto $A$ limitado em $\R$, vamos mostrar 
    que existe uma função contínua $f$ com $A = f^{-1}(\R - \{0\})$, seguirá que $\overline{A} = \text{supp }f$.
    Como estamos na reta, escreva $A$ como união enumerável de intervalos disjuntos (suas componentes conexas)
    $$A = \bigcup_{n=1} (a_n, b_n)$$
    sendo $A$ limitado, $\inf a_n > \infty$ e $\sup b_n < \infty$. Então, defina $f$ contínua sendo
    $$f(x) = \begin{cases}
        0 & \text{se } x \not \in A\\
        (b_n - x)(x - a_n) & \text{se } x \in (a_n, b_n)
    \end{cases}$$
    Claramente, como $A$ é limitado, o suporte de $f$ é compacto e $f^{-1}(\R - \{0\}) = A$, pois $f(x) \neq 0$ sse $x \in A$.
\end{proof}

\begin{problem}
    \label{prob:l4:4}
\end{problem}
Essa questão é fácil mas é bem longa, dá uma preguiça miserável escrevê-la. Vamos seperá-la em partes.
\begin{prop}
    $\rho : \R^2 \times \R^2 \to \R^+$ definida por
    $$\rho((x_1, y_1) , (x_2,y_2)) = \begin{cases}
        |y_1 - y_2| & \text{se } x_1 = x_2\\
        1 + |y_1 - y_2| & \text{se } x_1 \neq x_2\\
    \end{cases}$$
    é uma métrica de $\R^2$.
\end{prop}
\begin{proof}
    Temos que verificar as propriedades usuais - todas são claras, mas farei a fim de completude. 
    Não negatividade e separabilidade segue de que 
    $$\rho((a,b),(x,y)) = 0 \iff a = x \land b = y$$
    e $\rho$ é positiva. Simetria segue diretamente da definição e de que $|y_1 - y_2| = |y_2 - y_1|$. Para desigualdade 
    triangular, temos que analisar dois casos, sejam $p_1 = (x_1,y_1)$, $p_2 = (x_2,y_2)$, $p_3 = (x_3,y_3)$ três pontos de $\R^2$, então
    \begin{enumerate}
        \item Se $x_1 = x_3$, então $\rho(p_1,p_3) = |y_1 - y_3| \leq |y_1 - y_2| + |y_2 - y_3| \leq \rho(p_1,p_2) + \rho(p_2,p_3)$.
        \item Se $x_1 \neq x_3$, então já temos que $x_1 \neq x_2$ ou $x_2 \neq x_3$, em ambos os casos
        $$\rho(p_1,p_3) = 1 + |y_1 - y_3| \leq 1 + |y_1 - y_2| + |y_2 - y_3| \leq \rho(p_1,p_2) + \rho(p_2,p_3).$$ 
    \end{enumerate}
\end{proof}
\begin{prop}
    O espaço $(\R^2, \rho)$ é localmente compacto.
\end{prop}
\begin{proof}
    Para todo ponto $p = (x,y)$, a bola aberta de raio $1/2$ ao redor de $p$ é $B_{(p,1/2)} = \{x\} \times (y-1/2, y+1/2)$ é aberto. Vamos mostrar que
    seu fecho, $B_{[p,1/2]} = \{x\} \times [y-1/2, y+1/2]$, é compacto. Para isso, vamos usar somente que $[y-1/2,y+1/2]$ é compacto em $\R$.
    Seja $\{U_\alpha\}_\alpha$ uma cobertura aberta de $B_{[p,1/2]}$.
    Então
        $$B_{[p,1/2]} \subset (\{x\}\times \R) \cap \bigcup_\alpha U_\alpha$$
    Fixando o eixo $e_1 = x$ e tomando projeções $\pi_2$ na segunda coordenada, vemos que
    $$
        B_{[p,1/2]} \subset \bigcup_\alpha U_\alpha \iff [y-1/2,y+1/2] \subset \bigcup_\alpha \pi_2((\{x\} \times \R) \cap U_\alpha).
    $$
    Mas é óbvio pela definição de $\rho$ (que condiz com a métrica usual de $\R$ se a primeira coordenada está fixa) 
    que $\pi_2((\{x\} \times \R) \cap U_\alpha)$ são conjuntos abertos em $\R$. Logo, como $[y-1/2, y+1/2]$ é compacto,
    obtemos um conjunto finito de indices $\{\alpha_i\}_{i=1}^{N}$, tal que
    $$
    [y-1/2,y+1/2] \subset \bigcup_{i=1}^N \pi_2((\{x\} \times \R) \cap U_{\alpha_i})
    $$
    Claramente, $B_{[p,1/2]} \subset \bigcup_{i=1}^N U_{\alpha_i}$. Como a cobertura aberta foi 
    tomada arbitráriamente, provamos que $B_{[p,1/2]}$ é compacto.
\end{proof}

\begin{prop}
    \label{prob:l4:finite_compact}
    Para $f \in C_c((\R^2,\rho))$, existem no máximo finitos $x_1, \dots x_n$ tais quais $f(x,y) \neq 0$ para algum $y$. 
\end{prop}
\begin{proof}
    Antes de provar o resultado, perceba que, dado $x \in \R$ fixo, o conjunto 
    $$\{x\} \times \R = \bigcup_{y \in \R} B_{((x,y) , 1/2)}$$
    é união de abertos e portanto aberto do nosso espaço métrico.
    Agora o resultado segue quase que imediatamente por contradição, 
    suponha que $f$ de suporte compacto não satisfaz a condição. Então 
    existem infinitos $x_\alpha$ com $f(x_\alpha,y)$ diferente de $0$ para algum $y$.
    Em particular, vale a seguinte inclusão natural
    \begin{equation}
        \text{supp f} \subset \bigcup_\alpha \{x_\alpha\} \times \R
    \end{equation}
    pois, se $x_\beta \not \in \{x_\alpha\}_\alpha$, então, para qualquer $y,z \in \R$ e $\alpha$, $\rho( (x_\beta, y) , (x_\alpha, z)) >= 1$. 
    Logo, $(x_\beta, y)$ não é valor de aderência de nenhuma sequência de $f^{-1}(\R - \{0\})$. Note 
    que a cobertura em (12) óbviamente não admite subcobertura finita, absurdo.
\end{proof}

\begin{prop}
    (Exercício) Defina, para $f \in C_c((\R^2,\rho))$ o funcional linear
    $$\Lambda(f) = \sum_{j=1}^{n} \int_\R f(x_j,y) dy.$$
    Seja $\mu$ a medida recuperada pelo TRR. Se $E$ é o eixo-$x$ então $\mu(E) = \infty$
    e $\mu(K) = 0$ para todo compacto $K \subset E$.
\end{prop}

\begin{proof}
    Primeiramente temos que mostrar que o conjunto é mensurável. Vamos fazer algo melhor e mostrar que ele 
    é boreliano. Note que ele é interseção enumerável de abertos
    $$\R \times \{0\} = \bigcap_{n \in \N} \bigcup_{x \in \R} B_{((x,0), 1/n)}.$$
    Portanto, faz sentido tomar $\mu$ sobre ele. Vamos mostrar que para qualquer aberto $V$ que contém o eixo $x$,
    $\mu(V) = \infty$, para isso, por definição, basta exibir uma sequência de funções $f_n \prec V$ com $\Lambda(f_n) \to \infty$.

    Falta uma ideia boa para essa parte, mostrar que $V$ é gordo em infinitos pontos. Defina os conjuntos $A_n$ onde $V$ é um pouco espesso
    $$A_n = \{x \in \R ; \{x\} \times (-1/n, 1/n) \subset V\}$$
    Note que, como $V$ é aberto e contém $\R \times \{0\}$, claramente todo ponto $x \in \R$ pertence a algum $A_n$, i.e
    $$\R \subset \bigcup_{n \in \N} A_n$$
    Como $\R$ é não enumerável e temos na direita uma união enumerável, precisamos que algum $A_n$ seja não enumerável e portanto
    infinito. Seja $A_M$ então um dos conjuntos infinitos e escolha infinitos pontos  distintos $(x_k)_k \in A_M$.
    Agora estamos prontos para definir nossa sequência de funções de suporte compacto em $V$. Seja $G$ função real 
    com $[-1/2M, 1/2M] \prec G \prec (-1/M, 1/M)$, naturalmente
    $$\int_\R G dy \geq \frac{1}{M}$$
    Defina $f_n$ sendo
    $$f_n(x,y) = \begin{cases}
        G(y) & \text{se } x = x_k \text{ para algum } k \leq n\\ 
        0 & \text{c.c}
    \end{cases}.$$
    Claramente $f_n \prec V$, no entanto, 
    $$\Lambda(f_n) = \sum_{i=1}^{n} \int_\R f_n(x_m, y) dy = \sum_{i=1}^{n} \int_\R G(y) dy \geq \frac{n}{M} \to \infty$$
    quando $n \to \infty$. Como consequência, $\mu(V) = \infty$. Portanto para qualquer aberto $V \supset E$, $\mu(V) = \infty$,
    logo $\mu(E) = \infty$.

    Para mostrar que para todo compacto $K \subset E$, $\mu(K) = 0$, vamos construir outra sequência $(f_n)$, dessa vez
    com $K \prec f_n$, mas que $\Lambda(f_n) \to 0$. Lembrando da prova da proposição \ref{prob:l4:finite_compact} e notamos 
    que $K$ é finito. Escrevendo $K = \{(x_1,0), \dots, (x_M,0)\}$ fica claro o que devemos fazer. Seja $g_n$ uma função real contínua 
    com $[-1/2n, 1/2n] \prec g_n \prec (-1/n,-1/n)$, análogamente ao caso anterior definimos 
    $$f_n(x,y) = \begin{cases}
        g_n(y) & \text{se } x = x_k \text{ para algum } k\\
        0 & \text{c.c}
    \end{cases}.$$
    É imediato que $K \prec f_n$, mas note que 
    $$\Lambda(f_n) = \sum_{k = 1}^{M} \int_\R f_n(x_k, y)dy = \sum_{k=1}^{M} \int_\R g_n(y)dy \leq \frac{2M}{n} \to 0$$
    quando $n \to 0$, portanto $\mu(K) \leq 0$.
\end{proof}

\begin{problem}
    \label{prob:l4:5}
\end{problem}
\begin{enumerate}[label=(\alph*)]
    \item \begin{proof}
        Fazendo duas iterações temos 
        $$C_1 = [0,1/3] \cup [2/3, 1]$$
        $$C_2 = [0,1/9] \cup [2/9, 1/3] \cup [2/3,7/9] \cup [8/9,1]$$
        Indutivamente, se $C_n = \bigcup_m^{M} [a_m, b_m]$ é união finita de intervalos fechados
        disjuntos, então, por definição
        $$C_{n+1} = \bigcup_m^{M} \bigg[a_m, \frac{2a_m + b_m}{3}\bigg] \cup \bigg[ \frac{a_m + 2b_m}{3}, b_m \bigg]$$
        também é união finita de intervalos fechados disjuntos.

        Claramente os $C_n$ são compactos encaixados não vazios, portanto $C = \bigcap_n C_n$ é um compacto não vazio. 
    \end{proof}

    \item \begin{proof}
        Vamos mostrar por indução que depois da $n$-ésima iteração, todos intervalos tem tamanho igual a $3^{-n}$. Note que isso 
        implica o resultado, pois, sendo $C$ interseção de todos os $C_n$, não contém nenhum intervalo de medida positiva.
        É claro para $n = 0$. Como na $n$-ésima iteração removemos 
        o terço do meio de todos os intervalos que sobraram em $C_{n-1}$ (todos de tamanho $3^{-(n-1)}$), só 
        sobram intervalos de tamanho $3^{-n}$, o que demonstra o passo indutivo. 
    \end{proof}

    \item \begin{proof}
        Basta notar que ao remover os terços do intervalos fechados de $C_n$, temos que $\mu(C_{n+1}) = (2/3)\mu(C_n)$, 
        pois de cada intervalo fechado conexo, deixamos somente $2/3$ dele sobrando.
        Portanto, $\mu(C_n) = (2/3)^n \mu(C_0) = (2/3)^n$ e temos 
        $$\forall n \quad \mu(C) \leq \mu(C_n) \leq (2/3)^n$$
        ou seja, $\mu(C) = 0$.
    \end{proof}

    \item \begin{proof}
        Há inúmeras formas de fazer isso, vou fazer a que acredito ser a mais intuitiva - mas não acho que é a mais fácil.
        Note que em cada iteração da construção, nunca removemos os pontos extremos dos intervalos fechados,
        sempre removemos abertos propriamente contidos no meio dos intervalos. Isso quer dizer 
        que em qualquer momento da construção, se $C_n$ é a união disjunta
        $$C_n = \bigcup_m [a_m, a_m + 3^{-n}]$$
        então, para cada $m$, $a_m, 3^{-n} \in C$. Além do mais, em cada iteração, cada intervalo é dividido 
        em um intervalo da "esquerda" e um intervalo da "direita", como visto na letra (a).
        
        Seja $E$ e $D$ símbolos para esquerda e direita, a ideia será construir uma injeção de $\{E,D\}^\N$ para $C$
        usando sequências de pontos extremais dos $C_n$ - em cada passo da construção de $C$ 
        decidimos se escolhemos um ponto esquerdo ou direito. Dada uma sequência $(a_0, a_1, \dots) \in \{E,D\}^\N$, 
        definimos uma sequência $f((a_n)_n) = (c_0, c_1, \dots)$ em $C$  iterativamente. Para o primeiro elemento, temos
        $$c_0 = \begin{cases}
            0 & \text{se } a_0 = E\\
            1 & \text{se } a_0 = D
        \end{cases}$$
        Agora, suponha que já construímos até $c_n$. Se $a_{n} = E$, então no $n$-ésimo passo escolhemos um ponto extremal esquerdo e portanto
        em $C_{n+1}$ temos um conjunto do tipo $[c_{n}, c_{n} + 3^{-n-1}]$. Agora definimos 
        $$c_{n+1} = \begin{cases}
            c_{n} & \text{se } a_{n+1} = E\\
            c_{n} + 3^{-n - 1} & \text{se } a_{n+1} = D\\
        \end{cases}.$$
        Semelhantemente, se $a_{n} = D$, então no passo anterior escolhemos $c_n$ extremal direito e, portanto, 
        em $C_{n+1}$ temos um conjunto do tipo $[c_n - 3^{-n - 1}, c_n]$. Definimos
        $$c_{n+1} = \begin{cases}
            c_{n} - 3^{-n - 1} & \text{se } a_{n+1} = E\\
            c_{n} & \text{se } a_{n+1} = D\\
        \end{cases}$$

        A afirmação é que cada sequência assim definida é convergente em $C$ e sequências distintas convergem em pontos distintos.
        Para mostrar que $(c_1, c_2, \dots)$ converge em $C$, note que todos os elementos pertencem a $C$ e para cada $n$,
        $|c_{n+1} - c_n| \leq 3^{-n - 1}$, portanto 
        $$\sum_{n=0}^{\infty} |c_{n+1} - c_n| \leq \sum_{n=0}^{\infty} 3^{-n - 1} < \infty$$
        Pelo M-teste de Weierstrass, $(c_n)$ converge, como $C$ é fechado, $(c_n)$ converge em um ponto de $C$. 
        
        Vamos mostrar  que sequências diferentes de $E,D$ geram pontos diferentes de $C$. Sejam $a = (a_n)_n \neq b = (b_n)_n$ ambas em $\{E,D\}^\N$,
        seja $n_0$ o primeiro natural tal que $a_{n_0} \neq b_{n_0}$, então no $n_0$-passo, uma sequência escolheu ir para a esquerda 
        e a outra escolheu ir para a direita. Suponha sem perda de generalidade que $a_{n_0} = E$ e $b_{n_0} = D$, então
        se $n_0 = 0$, fica claro que para todo $n > 0$, $f(a)_n \in [0,1/3]$ e $f(b)_n \in [2/3,1]$, logo as sequências associadas não podem convergir no mesmo ponto.
        Semelhantemente, se $n_0 > 0$, então sem perda de generalidade, suponha que $a_{n_0 - 1} = b_{n_0 - 1} = E$, temos que $[f(a)_{n_0} = c_{n_0 - 1}, c_{n_0 - 1} + 3^{-n_0} = f(b)_{n_0}] \subset C_{n_0}$
        e desse momento adiante, para $n > n_0$, 
        $$f(a)_n \in [f(a)_{n_0 + 1}, f(a)_{n_0 + 1} + 3^{-n_0 - 1}]$$
        mas
        $$f(b)_n \in [f(b)_{n_0 + 1} - 3^{-n_0 - 1}, f(b)_{n_0 + 1}]$$
        e $f(b)_{n_0} - f(a)_{n_0} = 3^{- n_0}$, portanto são sempre disjuntos por uma distância $3^{-n_0 - 1}$, isso é as sequências $f(a)_n$ e $f(b)_n$ não podem convergir no mesmo ponto.
        Isso prova a injetividade de $f$ e a não enumerabilidade de $C$.
    \end{proof}
    
    \item
    Antes de fato solucionar a questão, vamos olhar para o que acontece em $C_1$. Podemos escrever $x \in [0,1]$ como 
    $$x = \frac{x_1}{3} + \frac{x_2}{3^2} + \dots + \frac{x_n}{3^n} + \dots$$
    onde cada $x_n$ é igual a $0$, $1$ ou $2$. Na primeira etapa $C_1$, ao remover a terça parte,
    estamos removendo justamente os números $x \in [0,1]$ cuja representação ternária tem $x_1 = 1$, com a exceção
    de $1/3 = 0.1 = 0.0222\dots = 0.0\bar{2}$ que permanece. Na segunda etapa, removemos dos que sobraram os que tem $x_2 = 1$, com exceção
    daqueles cujo último digito não $0$ é $x_2 = 1$ que podem ser escritos como $0.x_1 1 = 0.x_10\bar{2}$.
    Podemos sempre substituir o algarismo $1$ final pela sequência $0222\dots$ e obter o mesmo número, então 
    de forma geral, em cada etapa $n$, estamos removendo os números restantes que em ternário tem $x_n = 1$. Sendo a interseção 
    de todos esses $C_n$, $C$ não possui $1$ em sua representação em base 3.  Vamos tentar formalizar por indução.
    
    \begin{prop}
        $C_n$ é o conjunto de pontos de $ 0.x_1x_2x_3\cdots x_n\dots \in [0,1]$ com $x_1 \dots x_n$ diferentes de $1$.
        Além do mais, $C_n$ pode ser expresso como uma união disjunta
        $$C_n = \bigcup_m^{2^n} [a_m, a_m + 3^{-n}]$$
        onde cada $a_m$ é da forma $0.x_1x_2\dots x_n\bar{0}$ com cada $x_i \in \{0,2\}$ - por isso $2^n$ deles.
    \end{prop}
    \begin{proof}
        A prova, como quase todas as anteriores, será por indução. Claramente $C_1$ satisfaz isso,
        $$C_1 = [0,0.1 = 0.0\bar{2}] \cup [0.2, 1 = 0.\bar{2}]$$
        Suponha que vale para $C_n$, vamos provar que vale para $C_{n+1}$. Basta usar a identidade que vimos 
        na letra (a), precisamente, na etapa $n+1$, teremos que o conjunto
        $$[a_m, a_m+3^{-n}]$$
        se tornará 
        $$[a_m, a_m + 3^{-n - 1}] \cup [a_m + 2\cdot3^{-n - 1}, a_m + 3^{-n}]$$
        Se $a_m$ era da forma $0.x_1\dots x_n0\bar{0}$, conseguimos em $C_{n+1}$, além de $a_m$, um novo $a'_m = 0.x_1\dots x_n2\bar{0}$
        que representa o intervalo direito. Como isso vale para todo $a_m$, na etapa $n+1$ estamos de fato 
        construindo toda as $n+1$ sequências de $0,2$ até o $(n+1)$ digito. Segue naturalmente, olhando para esses intervalos, 
        que removemos exatamente os números contendo o digito $1$ na $(n+1)$ casa decimal, provando a hipótese de indução.
    \end{proof}
    \begin{corollary}
        (Exercício) $C$ é exatamente os números de $[0,1]$ que não contém $1$ em sua expansão ternária.
    \end{corollary}
    \begin{proof}
        Segue do fato de $C$ ser a interseção de todos os $C_n$.  
    \end{proof}
\end{enumerate}

%LISTA 5
\section{Lista 5 (11/09/2025)}

Listagem de problemas:
\begin{enumerate}
    \item Exercício \ref{prob:l5:1} : \checkmark
    \item Exercício \ref{prob:l5:2} : \Frowny
    \item Exercício \ref{prob:l5:3} : \Frowny
    \item Exercício \ref{prob:l5:4} : \Frowny
    \item Exercício \ref{prob:l5:5} : \Frowny
\end{enumerate}

\begin{problem}
    \label{prob:l5:1}
\end{problem}
Para não que eu não me confunda, vamos definir novamente funções semicontínuas.
\begin{definition}
    Uma função $f: X \to \R$ é dita semicontínua inferior (SCI) se para todo $\alpha \in \R$,
    $f^{-1}( (\alpha, \infty) )$ é aberto. Similarmente, $f$ é dita semicontinua superior (SCS)
    se $f^{-1}((-\infty, \beta))$ é aberto para todo $\beta \in \R$.
\end{definition}

Vamos mostrar que as proposições (a), (b) e (d) são verdadeiras, mas (c) é falso.
\begin{enumerate}[label=(\alph*)]
    \item \begin{proof}
        Sejam $f_1, f_2$ SCS e seja $\beta \in \R$ qualquer, queremos mostrar que $(f_1 + f_2)^{-1}((-\infty, \beta))$ é aberto.
        Podemos escrever esse conjunto como a seguinte união aberta
        $$(f_1 + f_2)^{-1}((-\infty, \beta)) = \bigcup_{x + y \leq \beta} [f_1^{-1}((-\infty, x)) \cap f_2^{-1}((-\infty, y))] $$
        Portanto, a pre-imagem é aberta e $f_1+ f_2$ é SCS. Note que aqui não usamos nada sobre a positividade de $f_1$ e $f_2$, então essa hipótese não é necessária.
    \end{proof}
    \item \begin{proof}
        Análogo à (a), se $f_1,f_2$ são SCI e $\alpha \in \R$, então 
        $$(f_1 + f_2)^{-1}((\alpha, \infty)) = \bigcup_{x + y \geq \alpha} [f_1^{-1}((x,\infty)) \cap f_2^{-1}((y,\infty))]$$
        é aberto. Como isso vale para todo $\alpha$, $f_1 + f_2$ é SCI. Novamente não precisamos da hipótese de positividade.
    \end{proof}
\end{enumerate}
Antes de mostrar que (c) é falsa usando um contra-exemplo, vamos mostrar que (d) é verdadeira. Precisamos de um lema - (e aqui sem perda de generalidade, 
vamos supor que nosso contra-domínio é a reta estendida)
\begin{lemma}
    Seja $(f_n)_{n \in \N}: X \to \overline{\R}$ uma sequência de funções SCI, então $\sup_{n \in \N} f_n$ é SCI.
\end{lemma}
\begin{proof}
    Vamos notar que para $\alpha \in \R$, $\sup_{n \in \N} f (x) > \alpha$ se somente se existe $n$ com $f_n(x) > \alpha$. Então podemos escrever
    $$(\sup_{n \in \N} f)^{-1}((\alpha, \infty)) = \bigcup_{n \in \N} f_n((\alpha, \infty)).$$
    Como para qualquer $\alpha$, $(\sup_{n \in \N} f)^{-1}((\alpha, \infty))$ é união de abertos, então $\sup_n f$ é SCI.
\end{proof}
Agora conseguimos provar (d).
\begin{enumerate}[label=(\alph*)]
    \addtocounter{enumi}{3}
    \item \begin{proof}
        Seja $F_n = \sum_{i=1}^{n} f_i$, por (a), todos os $F_n$ são SCI. Como $f_i$ são positivas, os $F_n$ são crescentes, portanto
        $$\sup_{n \in \N} F_n = \sum_{i=1}^{\infty} f_i$$
        pelo lema anterior, esse somatório é SCI. Aqui usamos fortemente a hipótese que as $f_i$ são positivas.
    \end{proof}
\end{enumerate}
% Se trocarmos $\sup$ por $\inf$, o lema anterior claramente funciona para sequências SCS também, isso é só uma observação e não será usado.
Vamos mostrar que (c) é falso.
\begin{enumerate}[label=(\alph*)]
    \addtocounter{enumi}{2}
    \item \begin{proof}
        Vimos em aula que se $F$ é fechado, $\mathds{1}_F$ é SCS. Considere a série de SCS's $\R \to \R$
        $$F(x) = \sum_{i=1}^{\infty} \mathds{1}\bigg[\frac{1}{2n+1}, \frac{1}{2n}\bigg](x).$$
        Claramente, $F(0) = 0$, logo $ 0 \in F^{1}((-\infty, 1/2))$, mas para valores arbitrariamente próximos - $\frac{1}{2n}$ -  não 
        pertencem a $F^{1}((-\infty, 1/2))$, logo esse conjunto não pode ser aberto ao redor de $0$. 
    \end{proof}
\end{enumerate}
Vimos que (a), (b) e (d) independem do espaço topológico do domínio. Sobre a positividade, só a usamos em (d) e aqui foi necessário para assegurar que 
a sequência das somas finitas era crescente, é fácil ver no entanto que essa hipótese é completamente necessária.
\begin{prop}
    Existe uma sequência de funções SCI's $(f_n)_n : \R \to \R$ com 
    $$F(x) = \sum_{i=1}^{\infty} f_i$$
    completamente descontínua. 
\end{prop}
\begin{proof}
    Como antes, se $F$ é fechado, $\mathds{1}_F$ é SCS e portanto, $-\mathds{1}_F$ é SCI. Agora basta considerar
    $$G(x) = \sum_{q \in \mathbb{Q}} - \mathds{1}[\{q\}] (x)$$
    a função que vale $-1$ nos racionais e $0$ nos irracionais. Óbviamente $G$ não é nem SCI, nem SCS e ela serve de contra-exemplo.

\end{proof}

\begin{problem}
    \label{prob:l5:2}
\end{problem}
Eu gostei muito desse problema - até porque fui eu quem propus \Smiley.
Vamos dividir a questão em etapas, a primeira sobre a existência, a segunda sobre as aproximações por S.C.I's.
Será necessária uma observação que eu não vou provar.
\begin{remark}
    Existe uma sequência de reais positivos $(a_n)_n$ menores que $1$, tal que 
    $$\prod_{n \in \N} (1 - a_n) > 0.$$
    Uma que funciona é a \href{https://en.wikipedia.org/wiki/Vi%C3%A8te%27s_formula}{fórmula de Viète}, por exemplo
\end{remark}

Agora podemos construir nosso conjunto.
\begin{proof}
    A ideia é criar um conjunto de Cantor gordo removendo frações de tamanhos diferentes em cada etapa. 
    Seja $\alpha_n$ uma sequência de números positivos menores que $1$ tal que
    $$ 1 > \prod_{n = 0}^{\infty} (1 - \alpha_n) = \alpha > 0.$$
    A ideia será na $n$-ésima iteração da construção de Cantor, remover uma $\alpha_n$ fração do conjunto restante em abertos de maneira esperta -
    sem deixar um intervalo de tamanho $1/n$.
    Sobrará, no final de todos os passos, uma $\alpha$-fração do intervalo [0,1] que terá medida positiva e nenhum intervalo de medida positiva.

    Seja $C_0 = [0,1]$. Suponha que seguimos as intruções anteriores até $n-1$, então temos 
    $$C_{n-1} = \bigcup_m [a_m, b_m]$$
    de forma que $b_m - a_m < 1/(n-1)$ e 
    $$\mu(C_{n-1}) = \prod_{k = 1}^{n-1} (1 - a_k)$$
    Para cada $m$, divida $[a_m, b_m]$ em $n$ intervalos de medida igual
    $$[a_m, b_m] = [a_m, t_1] \cup \bigcup_{i = 2}^{n-1} (t_i, t_{i+1}] \cup (t_n, b_m]$$
    Como $b_m - a_m \leq 1$, cada um deles certamente tem medida menor ou igual que $1/n$. Separe
    de cada subintervalo, uma fração aberta $\alpha_n$ do centro deles, por exemplo,
    $$(x_i,y_i) \subset (t_i, t_{i+1})$$
    onde $t_i < x_i,y_i < t_{i+1}$ e $y - x = \alpha_n (t_{i+1} - t_i)$.
    Removendo de $[a_m, b_m]$ a união desses intervalinhos $(x_i, y_i)$,
    estaremos claramente removendo uma fração $\alpha_n$ de $[a_m,b_m]$. Fazendo isso 
    para cada $m$, teremos removido em intervalos abertos, uma fração $\alpha_n$ de $C_{n-1}$, obtendo $C_n$.
    Pela forma que construimos, $C_n$ não contém intervalos de tamanho maior que $1/n$ e claramente,
    $$\mu(C_n) = (1-\alpha_n) \mu(C_{n-1}) = \prod_{i=1}^{n} (1-\alpha_n).$$
    
    Por construção, os $C_n$ são compactos encaixados e sua interseção forma um conjunto de Cantor $K$ de medida positiva sem 
    qualquer intervalo positivo. Por não ter nenhum intervalo, seu interior é vazio, logo na reta, o conjunto é totalmente desconexo.
\end{proof}
Vamos mostrar que a indicadora do conjunto $K$ construído, não pode ser aproximada por baixo por funções S.C.I.
\begin{proof}
    Seja $v : \R \to \R$ função S.C.I com $v \leq \mathds{1}_K$, vamos mostrar que $v \leq 0$ e, portanto
    $$\int_\R  (\mathds{1}_K - v) d\mu > \mu(K) $$

    Suponha que exista $x$ com $v(x) = c > 0$, em particular, teríamos um aberto não vazio $v^{-1}((c/2,\infty)) \subset K$,
    pois $v(x) > 0 \implies \mathds{1}_K(x) > 0$, mas $K$ como construído era totalmente desconexo - de interior vazio - absurdo.
\end{proof}

\begin{problem}
    \label{prob:l5:3}
\end{problem}
\begin{proof}
    Talvez não é a mais intuitiva, mas a prova do Rudin parece ser a mais simples.
    
    Para essa questão eu acho mais útil usar a definição geométrica de convexidade. 
    Uma função $\varphi : \R \to \R$ é convexa se para todo ponto $t$, $\varphi$ está acima da reta tangente 
    a $\varphi$ no ponto $t$. 
    
    Em termos analíticos, se definirmos
    \begin{align*}
        \alpha = \sup_{x < t} \frac{\varphi(t) - \varphi(x)}{t - x}\\
        \beta = \inf_{y > t} \frac{\varphi(y) - \varphi(t)}{y - t}
    \end{align*}
    as tangentes esquerda e direita no ponto $t$, então a proposição se expressa como 
    \begin{equation}
        \varphi(z) \geq \varphi(t) + \max(\alpha,\beta)(z - t)
    \end{equation}
    para todo $z$ em $\R$.

    Para provar Jensen, basta integrar sobre essa desigualdade com $z = f(x)$ e $t$ sendo o valor médio da função.
    Formalizando, seja 
    $$t = \int_{\Omega}f(x) d\mu$$
    Note que, como $a < f < b$, temos 
    $$ a = \int_{\Omega} a d\mu < \int_{\Omega}f(x) d\mu < \int_{\Omega} b d\mu = b$$
    logo $t \in (a,b)$. Fazendo a substituição em (13) e lembrando que $f(x) \in \R$, temos 
    $$\varphi(f(x)) \geq \varphi\bigg(\int_\Omega f d\mu\bigg) + \max(\alpha,\beta)\bigg(f(x) - \int_\Omega f d\mu\bigg)$$
    Integrando sobre $x$, o termo da direita cancela e ficamos com a desigualdade de Jensen.
    $$\int_\Omega \varphi \circ f d\mu \geq \varphi\bigg(\int_\Omega f d\mu\bigg)$$
\end{proof}

\begin{problem}
    \label{prob:l5:4}
\end{problem}

\begin{problem}
    \label{prob:l5:5}
\end{problem}
\begin{proof}
    %REPETIR PROVA DO ROBERTO
\end{proof}



\end{document}
