\clearpage
\section{Lista 8 (30/10/2025)}


\begin{problem}
    \label{prob:l8:1}
    Uma função $f$ é dita $L^1_{\text{loc}}(\R^n)$ se para toda bola $B$, $f\cdot\mathds{1}_B \in L^1(\R^n)$.
    Mostre que o Teorema de Diferenciação de Lebesgue ainda vale para $L^1_{\text{loc}}(\R^n)$.
\end{problem}

\begin{proof}
    Seja $B$ bola aberta arbitrária. Como $f \cdot\mathds{1}_B \in L^1(\R^n)$, pelo teorema da diferenciação de Lebesgue,
    para quase todo $x \in B$,
    $$f(x)\mathds{1}_B(x) = \lim_{r \to 0} \frac{1}{m(B_r)} \int_{B_r(x)}f(z)\mathds{1}_B(z) dz.$$
    Como $x \in B$, $f(x)\mathds{1}_B = f(x)$ e, para $r$ suficientemente pequeno, temos
    $$\frac{1}{m(B_r)} \int_{B_r(x)}f(z)\mathds{1}_B(z) dz = \frac{1}{m(B_r)} \int_{B_r(x)}f(z) dz.$$
    Portanto, para quase todo $x \in B$, vale
    $$f(x) =  \lim_{r \to 0} \frac{1}{m(B_r)} \int_{B_r(x)}f(z) dz.$$
    Como $\R^n = \bigcup_{n=1}^{\infty} B_n(0)$ é união enumerável de bolas abertas, 
    vale que quase todo ponto é de Lebesgue em $\R^n$.
\end{proof}


\begin{problem}
    \label{prob:l8:2}
    Seja $\varphi$ uma função Lebesgue mensurável em $\R^n$, que satisfaza seguinte propriedade:
    para todo retângulo $n$-dimensinal $Q$
    $$\bigg| \int_Q \varphi(x) dx\bigg| \leq \frac{Mm(Q)}{1 + m(Q)}$$
    para alguma constante $M$. Mostre que para todo $f \in L^1(\R^n)$
    $$\lim_{k\to\infty} \int_{\R^n} \varphi(kx)f(x)dx = 0.$$
\end{problem}
\begin{lemma}
    \label{lemma:nicely_shrinking}
    Seja $x$ um ponto de Lebesgue de $f \in L^1_{\text{loc}}(\R^n)$. Seja $\{E_i\}$ uma sequência
    de conjuntos borelianos que contém $x$, se existe $\alpha > 0$ e sequência de bolas $B_i(x,r_i)$
    satisfazendo
    \begin{enumerate}
        \item $E_i \subset B_i(x, r_i)$
        \item $m(E_i) > \alpha \cdot m(B_i(x,r_i))$
        \item $\lim_{i \to \infty} r_i = 0$
    \end{enumerate}
    então 
    $$\lim_{i\to\infty} \frac{1}{m(E_i)}\int_{E_i} |f - f(x)| dm  = 0$$
\end{lemma}
\begin{proof}
    Usando os dados do enunciado,
    $$\frac{1}{m(E_i)}\int_{E_i} |f - f(x)| dm \leq \frac{1}{\alpha m(B_i(x,r_i))}\int_{B_i(x,r_i)} |f - f(x)|dm$$
    tomando $i \to \infty$, pelo teorema da diferenciação de Lebesgue, o lado direito tende a $0$.
\end{proof}
\begin{lemma}
    \label{lemma:step_retangular_densa}
    Seja $S_Q$ o conjunto de funções simples de $\R^n$ cujas pré-imagens por um valor são sempre um retângulo. Isso é, se $s \in S$,
    então
    $$s = \sum_{j=1}^{N} c_j \mathds{1}_{Q_j}$$
    onde os $c_j$ são complexos e $Q_j$ são retângulos.

    Se $1 \leq p < \infty$, então $S_Q$ é denso em $L^p(\R^n)$.
\end{lemma}
\begin{proof}
    Basta notar que $S_Q$ é denso em $C_c(\R^n)$ na norma $||\cdot||_\infty$ que por sua vez é denso em $L^p(\R^n)$ para $ 1 \leq p < \infty$.
    Mas isso segue fácilmente do fato que as funções em $C_c(\R^n)$ são uniformemente contínuas. Dado $f \in C_c(\R^n)$,
    tomamos $Q$ um cubo grande tal que $\text{supp}(f) \subset Q$, dado qualquer $\eps > 0$, escolhemos $\delta > 0$
    de forma que $|x - y| < \delta$ implica $|f(x) - f(y)| < \eps$. Dividindo $Q$ em cubos menores $\{Q_j\}_{j=1}^{N}$ de diâmetro menor que $\delta$,
    escolha $x_j$ em qualquer um desses cubos e defina 
    $$s = \sum_{j=1}^{N} f(x_j) \mathds{1}_{Q_j}.$$
    Segue que 
    $$\int_{Q}{|f - s|^p}dm \leq \int_{Q} ||f - s||_\infty^p dm \leq \int_Q \eps^p dm = \eps^p m(Q)$$
    fazendo $\eps \to 0$, segue $S_Q$ é denso em $C_c(\R^n)$ em qualquer $L^p$.
\end{proof}

\begin{proof}
    (Do problema [\ref{prob:l8:2}]) Como $\varphi$ é integrável em qualquer retangulo $Q$, ela pertence a 
    $L^1_{\text{loc}}(\R^n)$. Em $\R^n$, qualquer bola $B$ de raio $r$ centrada em $x$, contém um cubo $Q$ centrado em $x$
    de lado $2r/\sqrt{n}$ e satisfaz que 
    $$\frac{m(Q)}{m(B)} \geq \frac{2^n r^n}{D n^{n/2} r^n} > C > 0$$
    Portanto, sempre podemos formar conjuntos bonitinhos, com respeito ao Lema [\ref{lemma:nicely_shrinking}] para usar Diferenciação
    de Lebesgue. Sai do Lema que em todo ponto de Lebesgue de $\varphi$ (e portanto quase todo ponto)
    $$|\varphi(x)| = \lim_{i \to \infty} \frac{1}{m(Q_i)} \bigg|\int_{Q_i} \varphi(x)dm\bigg| \leq \lim_{i \to \infty} \frac{M}{(1 + m(Q_i))} = M$$
    onde $Q_i$ é uma sequência retângulos que decresce adequadamente com as bolas, que podemos sempre tomar pela observação anterior. Em particular, vale que $||\varphi||_\infty \leq M$.

    Agora, dado um retângulo $Q \subset \R^n$, vamos calcular o limite para $f = \mathds{1}_Q$. Temos 
    $$\lim_{k \to \infty} \bigg|\int_{\R^n} \varphi(kx)\mathds{1}_Q(x) dm(x)\bigg| = \lim_{k \to \infty} \bigg|\int_{Q} \varphi(kx) dm(x)\bigg|$$
    fazendo a substituição $u = kx$ em $\R^n$, temos 
    $$\bigg| \int_{Q} \varphi(kx) dm(x)\bigg| = \frac{1}{k^n} \bigg|\int_{kQ} \varphi(u) dm(u)\bigg| \leq \frac{M m(kQ)}{k^n(1 + m(kQ))} \leq \frac{M}{k^n}$$
    que tende para $0$ quando $k \to \infty$, ou seja 
    $$\lim_{k \to \infty} \bigg|\int_{\R^n} \varphi(kx)\mathds{1}_Q(x) dm(x)\bigg| = 0.$$

    Por linearidade da integral, para qualquer função $s \in S_Q$, como definida no Lema [\ref{lemma:step_retangular_densa}],
    vale que 
    $$\lim_{k \to \infty} \bigg| \int_{\R^n} \varphi(kx)s(x) dm(x)\bigg| = 0.$$

    Agora para provar o resultado, dado $f \in L^1(\R^n)$ e $\eps > 0$, pelo Lema [\ref{lemma:step_retangular_densa}], tome $s \in S_Q$
    com 
    $$||f - s||_1 < \eps$$
    e seja $h = f - s$, segue que
    \begin{align*}
        \lim_{k \to \infty} \bigg|\int_{\R^n} \varphi(kx)f(x) dm(x)\bigg| &\leq \lim_{k \to \infty} \bigg|\int_{\R^n} \varphi(kx)s(x) dm(x)\bigg| + \bigg|\int_{\R^n} \varphi(kx)h(x) dm(x)\bigg|\\
        &\leq 0 + \int_{\R^n} ||\varphi||_\infty |h(x)| dm(x)\\
        &\leq M||h||_1 = M\eps
    \end{align*}
    Como isso vale para qualquer $\eps$, segue que 
    $$\lim_{k \to \infty} \int_{\R^n} \varphi(kx)f(x) dm(x) = 0.$$
\end{proof}

\begin{problem}
    
\end{problem}


\begin{problem}
    \label{prob:l8:4}
    Seja $E$ um conjunto de Lebesgue em $\R$, os limites superiores e inferiores dos quocientes
    $$\frac{m(E \cap (x-\delta, x+\delta))}{2\delta}$$
    são chamados das densidades superiores e inferiores de $E$ em $x$. Se esses são iguais, seu valor em comum
    $D_E(x)$ é a densidade de $E$ em $x$. Se $D_E(x) = 1$, $x$ é um ponto de densidade de $E$. Prove que $D_E(x) = 1$ para 
    quase todo ponto $x \in E$ e que $D_E(x) = 0$ para quase todo ponto $x \not \in E$. 
\end{problem}
\begin{proof}
    Basta notar que a função $\mathds{1}_E$ pertence a $L^1_{\text{loc}}(\R)$ uma vez que, para qualquer bola $B$,
    $$\int_{\R} \mathds{1}_B(x)\mathds{1}_E(x) dx \leq m(B) < \infty.$$
    Por [\ref{prob:l8:1}], quase todo ponto de $\R$ é de Lebesgue para $\mathds{1}_E$,
    ou seja, para quase todo $x$,
    $$\mathds{1}_E(x) = \lim_{\delta \to 0} \frac{1}{m(B_\delta)} \int_{B_\delta(x)} \mathds{1}_E dm = 
    \lim_{\delta \to 0} \frac{m(E \cap (x-\delta, x+\delta))}{2\delta}.$$
    E temos que $D_E(x) = 1$ para quase todo $x \in E$ e $D_E(x) = 0$ para quase todo ponto $x \not \in E$. 
\end{proof}


\begin{problem}
    \label{prob:l8:5}
    Seja $A,B \subset \R$, seja $A + B = \{a + b; a \in A, b \in B\}$. Suponha que $m(A) > 0$ e $m(B) > 0$, 
    prove que $A + B$ contém um intervalo.
\end{problem}
\begin{proof}
    Vou seguir o outline do Rudin. Sejam $a_0 \in A$ e $b_0 \in B$ pontos de densidade [Prob. \ref{prob:l8:4}], 
    mostrarei que existe um intervalo ao redor de $c_0 = a_0 + b_0$. Defina, para $E$ mensurável, $x \in \R$ e $\delta > 0$
    $$d_E(x, \delta) = \frac{m(E \cap (x-\delta, x+\delta))}{2\delta}.$$
    Temos que $\lim_{\delta \to 0} d_A(a_0, \delta) = 1$ e $\lim_{\delta \to 0} d_B(b_0, \delta) = 1.$
    
    Tome $\delta$ pequeno o suficiente tal que para todo $\delta' < \delta$, $d_A(a_0, \delta') > 2/3$ e 
    $d_B(b_0, \delta') > 2/3$. Agora, para cada $\eps \in \R$ positivo ou negativo defina 
    $B_\eps = \{c_0 + \eps - b; \,|b - b_0| < \delta/2,\,b \in B\}$. Crucialmente,
    $$B_\eps \subset (a_0 + \eps - \delta/2, a_0 + \eps + \delta/2) \subset (a_0 - |\eps| - \delta/2, a_0 + |\eps| + \delta/2).$$
    Além disso, sendo uma translação de uma vizinhança de $b_0$ em $B$ (multiplicada por $-1$),
    vale que 
    \begin{equation}
        \label{eq:l8:5:1}
        m(B_\eps) = m(B \cap (b_0 - \delta/2, b_0 + \delta/2)) > 2\delta/3.
    \end{equation}
    Se $|\eps| < \delta/2$, vale que $d_A(a_0, \delta/2 + |\eps|) > 2/3$, ou seja
    \begin{equation}
        \label{eq:l8:5:2}
        m(A \cap (a_0 - |\eps| - \delta/2, a_0 + |\eps| + \delta/2)) > \frac{2\delta + 4|\eps|}{3}.
    \end{equation}
    Agora, notamos que 
    \begin{align*}
        % \label{eq:l8:5:3}
        m(A \cap (a_0 - |\eps| - \delta/2, a_0 + |\eps| + \delta/2)) + m(B_\eps) &\\
        &>\frac{2\delta}{3}+ \frac{2\delta + 4|\eps|}{3}\\
        &> \delta + 2|\eps|\\
        &=m((a_0 - |\eps| - \delta/2, a_0 + |\eps| + \delta/2))
    \end{align*}
    Onde usamos [\ref{eq:l8:5:1}] e [\ref{eq:l8:5:2}] na primeira desigualdade e $\delta > 2|\eps|$ na segunda.
    Segue que $m(A \cap B_\eps) > 0$ e portanto $A \cap B_\eps \neq \varnothing$. Mas isso significa que para algum $b \in B$,
    $c_0 + \eps - b \in A$ e, portanto, somando $b$, $c_0 + \eps \in A + B$. Como isso vale para todo $\eps$ com $|\eps| < \delta/2$,
    segue que $(c_0 - \delta/2, c_0 + \delta/2) \subset A + B$.
    
    
\end{proof}