\section{Lista 5 (11/09/2025)}

Listagem de problemas:
\begin{enumerate}
    \item Exercício \ref{prob:l5:1} : \checkmark
    \item Exercício \ref{prob:l5:2} : \checkmark
    \item Exercício \ref{prob:l5:3} : \checkmark
    \item Exercício \ref{prob:l5:4} : \Frowny/\checkmark\, (a menos do Lema.)
    \item Exercício \ref{prob:l5:5} : \checkmark
\end{enumerate}

\begin{problem}
    \label{prob:l5:1}
\end{problem}
Para não que eu não me confunda, vamos definir novamente funções semicontínuas.
\begin{definition}
    Uma função $f: X \to \R$ é dita semicontínua inferior (SCI) se para todo $\alpha \in \R$,
    $f^{-1}( (\alpha, \infty) )$ é aberto. Similarmente, $f$ é dita semicontinua superior (SCS)
    se $f^{-1}((-\infty, \beta))$ é aberto para todo $\beta \in \R$.
\end{definition}

Vamos mostrar que as proposições (a), (b) e (d) são verdadeiras, mas (c) é falso.
\begin{enumerate}[label=(\alph*)]
    \item \begin{proof}
        Sejam $f_1, f_2$ SCS e seja $\beta \in \R$ qualquer, queremos mostrar que $(f_1 + f_2)^{-1}((-\infty, \beta))$ é aberto.
        Podemos escrever esse conjunto como a seguinte união aberta
        $$(f_1 + f_2)^{-1}((-\infty, \beta)) = \bigcup_{x + y \leq \beta} [f_1^{-1}((-\infty, x)) \cap f_2^{-1}((-\infty, y))] $$
        Portanto, a pre-imagem é aberta e $f_1+ f_2$ é SCS. Note que aqui não usamos nada sobre a positividade de $f_1$ e $f_2$, então essa hipótese não é necessária.
    \end{proof}
    \item \begin{proof}
        Análogo à (a), se $f_1,f_2$ são SCI e $\alpha \in \R$, então 
        $$(f_1 + f_2)^{-1}((\alpha, \infty)) = \bigcup_{x + y \geq \alpha} [f_1^{-1}((x,\infty)) \cap f_2^{-1}((y,\infty))]$$
        é aberto. Como isso vale para todo $\alpha$, $f_1 + f_2$ é SCI. Novamente não precisamos da hipótese de positividade.
    \end{proof}
\end{enumerate}
Antes de mostrar que (c) é falsa usando um contra-exemplo, vamos mostrar que (d) é verdadeira. Precisamos de um lema - (e aqui sem perda de generalidade, 
vamos supor que nosso contra-domínio é a reta estendida)
\begin{lemma}
    Seja $(f_n)_{n \in \N}: X \to \overline{\R}$ uma sequência de funções SCI, então $\sup_{n \in \N} f_n$ é SCI.
\end{lemma}
\begin{proof}
    Vamos notar que para $\alpha \in \R$, $\sup_{n \in \N} f (x) > \alpha$ se somente se existe $n$ com $f_n(x) > \alpha$. Então podemos escrever
    $$(\sup_{n \in \N} f)^{-1}((\alpha, \infty)) = \bigcup_{n \in \N} f_n((\alpha, \infty)).$$
    Como para qualquer $\alpha$, $(\sup_{n \in \N} f)^{-1}((\alpha, \infty))$ é união de abertos, então $\sup_n f$ é SCI.
\end{proof}
Agora conseguimos provar (d).
\begin{enumerate}[label=(\alph*)]
    \addtocounter{enumi}{3}
    \item \begin{proof}
        Seja $F_n = \sum_{i=1}^{n} f_i$, por (a), todos os $F_n$ são SCI. Como $f_i$ são positivas, os $F_n$ são crescentes, portanto
        $$\sup_{n \in \N} F_n = \sum_{i=1}^{\infty} f_i$$
        pelo lema anterior, esse somatório é SCI. Aqui usamos fortemente a hipótese que as $f_i$ são positivas.
    \end{proof}
\end{enumerate}
% Se trocarmos $\sup$ por $\inf$, o lema anterior claramente funciona para sequências SCS também, isso é só uma observação e não será usado.
Vamos mostrar que (c) é falso.
\begin{enumerate}[label=(\alph*)]
    \addtocounter{enumi}{2}
    \item \begin{proof}
        Vimos em aula que se $F$ é fechado, $\mathds{1}_F$ é SCS. Considere a série de SCS's $\R \to \R$
        $$F(x) = \sum_{i=1}^{\infty} \mathds{1}\bigg[\frac{1}{2n+1}, \frac{1}{2n}\bigg](x).$$
        Claramente, $F(0) = 0$, logo $ 0 \in F^{1}((-\infty, 1/2))$, mas para valores arbitrariamente próximos - $\frac{1}{2n}$ -  não 
        pertencem a $F^{1}((-\infty, 1/2))$, logo esse conjunto não pode ser aberto ao redor de $0$. 
    \end{proof}
\end{enumerate}
Vimos que (a), (b) e (d) independem do espaço topológico do domínio. Sobre a positividade, só a usamos em (d) e aqui foi necessário para assegurar que 
a sequência das somas finitas era crescente, é fácil ver no entanto que essa hipótese é completamente necessária.
\begin{prop}
    Existe uma sequência de funções SCI's $(f_n)_n : \R \to \R$ com 
    $$F(x) = \sum_{i=1}^{\infty} f_i$$
    completamente descontínua. 
\end{prop}
\begin{proof}
    Como antes, se $F$ é fechado, $\mathds{1}_F$ é SCS e portanto, $-\mathds{1}_F$ é SCI. Agora basta considerar
    $$G(x) = \sum_{q \in \mathbb{Q}} - \mathds{1}[\{q\}] (x)$$
    a função que vale $-1$ nos racionais e $0$ nos irracionais. Óbviamente $G$ não é nem SCI, nem SCS e ela serve de contra-exemplo.

\end{proof}

\begin{problem}
    \label{prob:l5:2}
\end{problem}
Eu gostei muito desse problema - até porque fui eu quem propus \Smiley.
Vamos dividir a questão em etapas, a primeira sobre a existência, a segunda sobre as aproximações por S.C.I's.
Será necessária uma observação que eu não vou provar.
\begin{remark}
    Existe uma sequência de reais positivos $(a_n)_n$ menores que $1$, tal que 
    $$\prod_{n \in \N} (1 - a_n) > 0.$$
    Uma que funciona é a \href{https://en.wikipedia.org/wiki/Vi%C3%A8te%27s_formula}{fórmula de Viète}, por exemplo
\end{remark}

Agora podemos construir nosso conjunto.
\begin{proof}
    A ideia é criar um conjunto de Cantor gordo removendo frações de tamanhos diferentes em cada etapa. 
    Seja $\alpha_n$ uma sequência de números positivos menores que $1$ tal que
    $$ 1 > \prod_{n = 0}^{\infty} (1 - \alpha_n) = \alpha > 0.$$
    A ideia será na $n$-ésima iteração da construção de Cantor, remover uma $\alpha_n$ fração do conjunto restante em abertos de maneira esperta -
    sem deixar um intervalo de tamanho $1/n$.
    Sobrará, no final de todos os passos, uma $\alpha$-fração do intervalo [0,1] que terá medida positiva e nenhum intervalo de medida positiva.

    Seja $C_0 = [0,1]$. Suponha que seguimos as intruções anteriores até $n-1$, então temos 
    $$C_{n-1} = \bigcup_m [a_m, b_m]$$
    de forma que $b_m - a_m < 1/(n-1)$ e 
    $$\mu(C_{n-1}) = \prod_{k = 1}^{n-1} (1 - a_k)$$
    Para cada $m$, divida $[a_m, b_m]$ em $n$ intervalos de medida igual
    $$[a_m, b_m] = [a_m, t_1] \cup \bigcup_{i = 2}^{n-1} (t_i, t_{i+1}] \cup (t_n, b_m]$$
    Como $b_m - a_m \leq 1$, cada um deles certamente tem medida menor ou igual que $1/n$. Separe
    de cada subintervalo, uma fração aberta $\alpha_n$ do centro deles, por exemplo,
    $$(x_i,y_i) \subset (t_i, t_{i+1})$$
    onde $t_i < x_i,y_i < t_{i+1}$ e $y - x = \alpha_n (t_{i+1} - t_i)$.
    Removendo de $[a_m, b_m]$ a união desses intervalinhos $(x_i, y_i)$,
    estaremos claramente removendo uma fração $\alpha_n$ de $[a_m,b_m]$. Fazendo isso 
    para cada $m$, teremos removido em intervalos abertos, uma fração $\alpha_n$ de $C_{n-1}$, obtendo $C_n$.
    Pela forma que construimos, $C_n$ não contém intervalos de tamanho maior que $1/n$ e claramente,
    $$\mu(C_n) = (1-\alpha_n) \mu(C_{n-1}) = \prod_{i=1}^{n} (1-\alpha_n).$$
    
    Por construção, os $C_n$ são compactos encaixados e sua interseção forma um conjunto de Cantor $K$ de medida positiva sem 
    qualquer intervalo positivo. Por não ter nenhum intervalo, seu interior é vazio, logo na reta, o conjunto é totalmente desconexo.
\end{proof}
Vamos mostrar que a indicadora do conjunto $K$ construído, não pode ser aproximada por baixo por funções S.C.I.
\begin{proof}
    Seja $v : \R \to \R$ função S.C.I com $v \leq \mathds{1}_K$, vamos mostrar que $v \leq 0$ e, portanto
    $$\int_\R  (\mathds{1}_K - v) d\mu > \mu(K) $$

    Suponha que exista $x$ com $v(x) = c > 0$, em particular, teríamos um aberto não vazio $v^{-1}((c/2,\infty)) \subset K$,
    pois $v(x) > 0 \implies \mathds{1}_K(x) > 0$, mas $K$ como construído era totalmente desconexo - de interior vazio - absurdo.
\end{proof}

\begin{problem}
    \label{prob:l5:3}
\end{problem}
Antes de começar, precisamos de um leminha que eu havia esquecido.
\begin{lemma}
    Seja $(X,\mu)$ espaço de medida positiva finita. Se $1 \leq q \leq p \leq \infty$, 
    então $L^p(\mu) \subset L^q(\mu)$.
\end{lemma}
\begin{proof}
    Seja $f \in L^p(\mu)$, queremos mostrar que $f \in L^q(\mu)$.
    O caso $p = \infty$ é óbvio. Se $p < \infty$, tome $r > q$ tal que 
    $$\frac{1}{p} + \frac{1}{r} = \frac{1}{q}$$
    Em particular,
    $$\frac{q}{p} + \frac{q}{r} = 1$$
    Podemos aplicar Hölder com $p/q$ e $r/q$ em $|f|^q$ para obter
    $$\int_X |f|^q d\mu = \int_X 1 \cdot |f|^q d\mu \leq \bigg(\int_X (|f|^q)^{p/q}d\mu \bigg)^{q/p} \bigg(\int_X 1^{r/q}d\mu\bigg)^{q/r} = (||f||_p)^q (\mu(X))^{q/r} < \infty$$
\end{proof}
Portanto, para a questão, como $f \in L^2(\mu)$, sabemos que está em $L^1(\mu)$ também.

\begin{proof} \textbf{(Do Exercício)}
    Talvez não é a mais intuitiva, mas a prova do Rudin parece ser a mais simples.
    
    Para essa questão eu acho mais útil usar a definição geométrica de convexidade. 
    Uma função $\varphi : \R \to \R$ é convexa se para todo ponto $t$, $\varphi$ está acima da reta tangente 
    a $\varphi$ no ponto $t$. 
    
    Em termos analíticos, se definirmos
    \begin{align*}
        \alpha = \sup_{x < t} \frac{\varphi(t) - \varphi(x)}{t - x}\\
        \beta = \inf_{y > t} \frac{\varphi(y) - \varphi(t)}{y - t}
    \end{align*}
    as tangentes esquerda e direita no ponto $t$, então a proposição se expressa como 
    \begin{equation}
        \varphi(z) \geq \varphi(t) + \max(\alpha,\beta)(z - t)
    \end{equation}
    para todo $z$ em $\R$.

    Para provar Jensen, basta integrar sobre essa desigualdade com $z = f(x)$ e $t$ sendo o valor médio da função.
    Formalizando, seja 
    $$t = \int_{\Omega}f(x) d\mu$$
    Note que, como $a < f < b$, temos 
    $$ a = \int_{\Omega} a d\mu < \int_{\Omega}f(x) d\mu < \int_{\Omega} b d\mu = b$$
    logo $t \in (a,b)$. Fazendo a substituição em (13) e lembrando que $f(x) \in \R$, temos 
    $$\varphi(f(x)) \geq \varphi\bigg(\int_\Omega f d\mu\bigg) + \max(\alpha,\beta)\bigg(f(x) - \int_\Omega f d\mu\bigg)$$
    Integrando sobre $x$, o termo da direita cancela e ficamos com a desigualdade de Jensen.
    $$\int_\Omega \varphi \circ f d\mu \geq \varphi\bigg(\int_\Omega f d\mu\bigg)$$
\end{proof}

\begin{problem}
    \label{prob:l5:4}
\end{problem}
\Frowny\;
Fiquei longe de resolver esse exercício. Um Henrique do futuro pode encontrar a solução \href{https://www.research-collection.ethz.ch/server/api/core/bitstreams/8ce179c7-da1f-4862-9404-cefb49d8e64e/content}{aqui},
na seção 2.1.5.

Como fui mal na prova \Frowny, vou escrever a solução a menos do lema necessário - que não provarei.
\begin{lemma}
    Seja $\{a_n\} \subset \R^+$ uma sequência tal que
    $$\sum_{n=1}^{\infty} \frac{a_n}{n} < \infty.$$
    Existe sequência de índices $\{n_j\} \subset \N$ satisfazendo 
    $$\sum_{j=1}^{\infty} a_{n_j} < \infty \quad \land \quad \lim_{j \to \infty} \frac{n_{j+1}}{n_j} = 1.$$ 
\end{lemma}

\label{lista5:eq:limite_int}
\begin{proof}
    \textbf{Do Exercício.} Aplique o lema anterior na sequência
    $$a_N = \int_X \bigg|\sum_{n=1}^{N} \frac{e(mf_n(x))}{N} \bigg|^p d\mu$$
    Conseguimos uma subsequência $a_{n_j}$ satisfazendo que 
    $$\sum_{j=1}^{\infty} a_{n_j} < \infty \quad \land \quad \lim_{j \to \infty} \frac{n_{j+1}}{n_j} = 1$$
    Em particular
    \begin{equation}
        \label{lista5:eq:finite_sum}
        \sum_{j=1}^{\infty} \int_X \bigg|\sum_{n=1}^{N_j} \frac{e(mf_n(x))}{N_j} \bigg|^p d\mu < \infty.
    \end{equation}
    Vamos mostrar que em quase todo ponto $x$, 
    \begin{equation}
        \label{lista5:eq:lim_to_zero} 
        \sum_{n=1}^{N_j} \frac{e(mf_n(x))}{N_j} \to 0.
    \end{equation}
    Suponha que o limite não é $0$ para algum conjunto $E$ de medida positiva, escrevemos $E$ como a seguinte união
    $$E = \bigcup_{M \in \N} \bigg\{ x : \limsup_{j\to\infty} \bigg|\sum_{n=1}^{N_j} \frac{e(mf_n(x))}{N_j}\bigg| > \frac{1}{M} \bigg\}.$$
    Por $E$ ser uma união enumerável de medida positiva, existe algum $M$, tal que o $M$-ésimo conjunto dessa união tem medida
    positiva, seja esse conjunto $E_M$. Se $x \in E_M$, então, para infinitos $j$'s
    $$\bigg|\sum_{n=1}^{N_j} \frac{e(mf_n(x))}{N_j}\bigg| > \frac{1}{M}$$
    e portanto
    $$\sum_{j=1}^{\infty} \bigg|\sum_{n=1}^{N_j} \frac{e(mf_n(x))}{N_j}\bigg| = \infty.$$
    Mas então, substituindo $E_M$ no somatório (\ref{lista5:eq:finite_sum}),
    $$\sum_{j=1}^{\infty} \int_{E_M} \bigg|\sum_{n=1}^{N_j} \frac{e(mf_n(x))}{N_j} \bigg|^p = \int_{E_M} \sum_{j=1}^{\infty} \bigg|\sum_{n=1}^{N_j} \frac{e(mf_n(x))}{N_j} \bigg|^p d\mu = \int_{E_M} \infty d\mu = \infty $$
    O que contradiz (\ref{lista5:eq:finite_sum}) ser finito.

    Agora concluimos que se (\ref{lista5:eq:lim_to_zero}) vale a.e, então equidistribuição vale a.e.
    Note que para $N$ suficientemente grande, existem índices da subsequência $N_j$ e $N_{j+1}$
    com $N_j \leq N < N_{j+1}$ e $j \to \infty$ quando $N \to \infty$. Podemos portanto escrever
    \begin{align*}
        \bigg|\sum_{n=1}^{N} \frac{e(mf_n(x))}{N}\bigg| &= \bigg|\sum_{n=1}^{N_j} \frac{e(mf_n(x))}{N} + \sum_{n=N_j + 1}^{N} \frac{e(mf_n(x))}{N} \bigg|\\
        &\leq \bigg|\sum_{n=1}^{N_j} \frac{e(mf_n(x))}{N}\bigg| + \frac{N - N_j}{N}\\
        &\leq \bigg|\sum_{n=1}^{N_j} \frac{e(mf_n(x))}{N}\bigg| + \frac{N_{j+1} - N_j}{N_j}\\
        &\leq \bigg|\sum_{n=1}^{N_j} \frac{e(mf_n(x))}{N}\bigg| + \frac{N_{j+1}}{N_j} - 1 \to 0
    \end{align*}
    quando $j \to \infty$. Isso completa a demonstração.

\end{proof}

\begin{problem}
    \label{prob:l5:5}
\end{problem}
Há duas formas que eu conheço de resolver esse problema. Uma ideia é mostrar a desigualdade reversa de Hölder e 
repetir a prova de Minkowski com as desigualdades invertidas. A outra, que é bem mais rápida, é repetir a prova do Prof. Roberto
em seu livro de Análise do Rn - vou fazer essa.
\begin{proof}
    Estenda a definição de $||f||_p$ para $p \in (0,1)$ sendo justamente
    $$||f||_p = \bigg( \int_X |f|^p \bigg)^{1/p}$$
    Queremos mostrar que para $f,g \geq 0$,
    \begin{equation}
        \label{lista5:eq:minkowski}
        ||f + g||_p \geq ||f||_p + ||g||_p
    \end{equation}
    
    
    Vamos analisar primeiramente o caso em que $0 < ||f||_p, ||g||_p < \infty$. Note que, sob essa hipótese, (\ref{lista5:eq:minkowski}) é equivalente a
    \begin{equation}
        \label{lista5:eq:normalization}
        \frac{||f + g||_p}{||f||_p + ||g||_p} = \bigg|\bigg|\frac{f + g}{||f||_p + ||g||_p}\bigg|\bigg|_p \geq 1
    \end{equation}
    O truque do professor Roberto é perceber que se $||f||_p$ e $||g||_p$ são positivas,
    podemos escrever
    $$\frac{f + g}{||f||_p + ||g||_p} =  \frac{\lambda f}{||f||_p} + \frac{(1-\lambda)g}{||g||_p}$$
    onde
    $$\lambda = \frac{||f||_p}{||f||_p + ||g||_p} \in (0,1).$$
    Elevando os dois lados de (\ref{lista5:eq:normalization}) por $p$ e lembrando da positividade de $f$ e $g$, notamos que
    $$\bigg|\bigg|\frac{f + g}{||f||_p + ||g||_p}\bigg|\bigg|_p \geq 1 \iff \int_X \bigg(\frac{f + g}{||f||_p + ||g||_p}\bigg)^p d\mu \geq 1$$
    Usando a concavidade de $x^p$ e  a expansão com o $\lambda$ anterior, temos
    \begin{align*}
        \int_X \bigg(\frac{f + g}{||f||_p + ||g||_p}\bigg)^p d\mu &\geq \int_X \lambda \bigg(\frac{f}{||f||_p}\bigg) d\mu + \int_X (1-\lambda) \bigg(\frac{g}{||g||_p}\bigg) d\mu\\
        &= \lambda + 1 - \lambda = 1.
    \end{align*}
    O que demonstra a desigualdade.

    Falta o caso em que alguma das duas funções tem "norma" $0$ ou $\infty$. Suponha sem perda de generalidade que  $||f||_p = 0$,
    então definindo $E = \{x : f(x) > 0\}$, sabemos que 
    $$\mu(E) = 0$$
    Portanto,
    $$\int_X (f+g)^p d\mu= \int_{X - E} (f+g)^p d\mu= \int_{X - E} g^p d\mu= \int_X g^p d\mu$$
    e vale a igualdade $||f + g||_p = ||g||_p$. No outro caso, se $||f||_p = \infty$, 
    então naturalmente,
    $$\int_X (f + g)^p d\mu \geq \int_X f^p d\mu = \infty$$
    e a desigualdade vale trivialmente.
\end{proof}
